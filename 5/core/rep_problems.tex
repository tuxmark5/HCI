\xbug{1}{picture}
{
  Mygtukas „Rodyti“ yra ne savo vietoje (turi būti arčiau intervalo
  pasirinkimo laukų):
}
{
  Taip nuspręsta dėl to, kad mygtuką rodyti reikia paspausti tik tuomet, kai yra
  pakeičiamas rodomas intervalas. Keičiant intervalą ar jo granuliarumą kitais mygtukais,
  rezultatai yra atnaujinami automatiškai.
}

\xbugN{2}
{
  Pakeitus rezultatų skaičiavimo granuliarumą rezultatai neatsinaujina.
}
{
  Norint atnaujinti rezultatus, reikia dar papildomai paspausti „Rodyti“, o tai yra papildomas
  nereikalingas žingsnis, mažinantis interfeiso efektyvumą.
}

\xbug{3}{Pradinė būsena, kurioje visų padalinių/IS rodymo rėžimai sutampa}
{
  Pakeitus skaičiavimo rėžimą, rezultatų rodymo rėžimas atsistato į pradinį.
}
{
  Rezultatų lange kiekvieno padalinio/IS vaizdavimo rėžimą galima reguliuoti keliais skirtingais
  parametrais. Pagal savo poreikius sukonfigūravus kelis padalinius/IS ir pakeitus skaičiavimo rėžimą pvz. iš IS į padalinių apkrovų skaičiavimą, sitema „pamiršta“, koks vaizdavimo rėžimas
  buvo parinktas kiekvienam padaliniui/IS. Dėl to labai stipriai nukenčia efektyvumas, nes norint
  atgal susigrąžinti seną konfigūraciją, reikia sugaišti nemažai laiko.
}

\xbug{4}{Mygtukas „Atstatyti“ šalia „Išsaugoti“ ir „Mažiausios apkrovos intervalai“}
{
  Netyčia paspaudus „Atstatyti“, naudotojas gali prarasti visus pakeitimus
  atliktus po paskutinio išsaugojimo.
}
{
  Mygtukas „Atstatyti“ yra negrįžtama operacija, kuri pašalina visus padarytus pakeitimus.
  Kadangi tai yra tokia destruktyvi operacija, kurią nesunkiai galima iškviesti per klaidą,
  tai vertėtų įdėti modalinį dialogą, prašantį patvirtinti vartotojo veiksmą.
}

\xbug{5}{Pataisyta apžvalgos sąsaja su tinkama antrašte}
{
  Apžvalgos sąsajoje nesimatė, koks rodymo rėžimas dabar yra aktyvus.
}
{
  Kadangi apžvalgos interfeise yra pateikiama viena lentelė/grafikas, kuriame vaizduojami
  apibendrinti duomenys, tai vartotojui pirmą kartą įėjus į šį interfeisą (arba sugrįžus po
  pertraukos) gali būti neaišku, kas konkrečiai toje lentelėje/grafike yra vaizduojama.
}

\xbugN{6}
{
  Ištrynus įrašus iš bet kurios lentelės, rezultatai automatiškai neatsinaujina.
}
{
  Keičiant duomenis, arba įdedant naujus įrašus, sistema automatiškai atnaujina rezultatus
  pagal vartotojo atliktus veiksmus. Tačiau pašalinus vieną arba daugiau įrašų sistema
  rezultatų neperskaičiuoja ir interfeiso neatnaujina.
}

\xbug{7}{Orginalus ir pakoreguotas intervalai savaitiniame granuliarume}
{
  Pasirinkus savaitinį granuliarumą sistema pati modifikuoja intervalo galus,
  nepaaiškindama, kodėl taip daro.
}
{
  Tam kad supaprastinti skaičiavimus ir padidinti jų tikslumą, sistema automatiškai
  pagoreguoja intervalo galus taip, kad šie atitiktu turimus duomenis, prasidėtų ties 
  granuliaraus vieneto pradžia (jei pasirinktas savaitinis rodymas - tai pirmadieniu,
  jei mėnėsinis - tai pirmąja mėnesio diena ir t.t), tačiau vartotojui gali neišku, dėl
  ko intervalai savaime pajuda. Dėl to reikėtų įdėti pranešimą, kuriame yra paaiškinama,
  kodėl intervalo galai buvo paslinkti.
}

\xbug{8}{Ne iš eilės pažymėtos eilutės}
{
  Šalinant ne iš eilės pažymėtas eilutes, sistema arba išmeta ne išmetimui
  skirtas eilutes, arba nulūžta.
}
{
  Tai yra paprasčiausia algoritmavimo klaida kode, atsakingame už elementų pašalinimą
  iš lentelių.
}

\xbug{9}{Užblokuota prieiga prie sistemų rėžimo}
{
  Administratoriaus sąsajoje naudotojui įjungus prieigą prie sistemų
  rėžimo, šis vis tiek naudotojui netampa prieinamu.
}
{
  Tai vėl yra dar viena algoritmavimo klaida, šį kartą dėl sumaišytų privilegijų deskriptorių.
}

\xbugN{10}
{
  Į pranešimų žurnalą keliauja ne visi pranešimai.
}
{
  Kadangi pranešimų žurnalo funkcija yra kaupti pranešimus, tai šis turėtu tai atlikti
  visiems sistemos pateikiamiems pranešimais. Kadangi dabar taip nėra, tai nukenčia
  nuspėjamumas.
}

\xbug{11}{Ištrauka iš pranešimų žurnalo apie įvykusias importo klaidas}
{
  Pašalinus viską iš duomenų bazės ir importuojant duomenis iš failo,
  sistema skundžiasi, jog tokie elementai jau yra, nors ką tik buvo pašalinti.
}
{
  Ši problema vėl susijusi su algoritmavimo klaida kode, atsakingame už elementų pašalinimą
  iš konteinerių.
}

\xbugN{12}
{
  Importuojant failą, sistema neatsimena aplanko, kuriame paskutinį
  kartą buvo vykdytas importavimas.
}
{
  Gana smulki klaida, nes duomenys į sistemą įkeliami yra gana retai. Tačiau norint
  padidinti našumą, galima įsiminti paskutinę direktoriją, iš kurios failas importuotas.
}

\xbugN{13}
{
  Naudotojų administravimo sąsaja ne visose platformose reaguoja į
  aktyvaus naudotojo pasikeitimus.
}
{
  Tai problema, kuri pasireiškia tik kai kuriose platformose/OS, potencialiai susijusi
  su galima klaida Qt biblitekos kode, atsakingame už pranešimą apie paspaudimus ant 
  lentelės.
}

\xbug{14}{Neigiamos apkrovos}
{
  Kai kuriais atvejais dėl panaudoto apkrovų skaičiavimo algoritmo,
  apkrovos gaunasi neigiamos.
}
{
  Tam, kad duomenys rezultatuose atrodytų tolydžiau, sistema naudoja ekstrapoliaciją
  apkrovų skaičiavimo algoritme, tačiau dėl jo kai kuriais atvejais gaunasi neigiamos
  apkrovos. Norint šią klaidą ištaisyti reikėtų kode įdėti minimalios apkrovos apribojimą ir
  taip išvengti neigiamų apkrovų. Taip pat pravartu būtų galimybė laikinai išjungti
  ekstrapoliaciją, jei to vartotojas nepageidauja.
}

\xbug{15}{Ištaisytas interfeisas prognozes žymi žaliai}
{
  Grafikuose aiškiai nesimato, ties kuria vieta prasideda prognozės
}
{
  Vieta ties kuria sistema pradeda skaičiuoti apkrovas remdamasi fiktyviais prognozuojamais
  duomenimis turėtų būti aiškiai išryškinta tam, kad palengvinti naudotojo orientaciją
  duomenyse.
}

\xbug{16}{Privilegijos pasirinkimas administratoriaus interfeise}
{
  Iš karto nėra aišku, ką reiškia pasirinkimas „Leisti ir rašyti“.
}
{
  Kiekvienas vartotojas turi tam tikras privilegijas prie lentelių duomenų bazėje. Viso 
  sistemoje gali būti trys privilegijos lygiai:
  \begin{itemize}
    \item Slėpti - nerodyti minimo interfeiso visai.
    \item Leisti tik skaityti - leisti tik peržiūrėti duomenis.
    \item Leisti ir rašyti - leisti ir skaityti, ir rašyti duomenis.
  \end{itemize}
  Kad būtų aiškiau, reikia paskutinį privilegijos lygį pervadinti.
}

\xbug{17}{Užblokuotas administratoriaus interfeisas, nes nepasirinktas joks vartotojas}
{
  Nėra aišku, kad norint, jog kortelės „Naud. adm.“ turinys taptų aktyvus,
  reikia kortelėje „Naudotojai“ paspausti ant kurio nors iš naudotojų.
}
{
  Reikėtų uždėti pranešimą, skatinantį pasirinkti aktyvų naudotoją.
}

\xbugN{18}
{
  Naudotojui pačiam sau pasikeitus, jog (ne)rodytų apžvalgos kortelę, ji
  atsiranda tik iš naujo prisijungus.
}
{
  Šios klaidos pataisymas greičiausiai realizuotas nebus, nes tik administratorius
  turi teisę keisti privilegijas, o tokio pakankamai nereikalingo pataisymo realizacija
  yra pakankamai sudėtinga. Kaip alternatyvą, galima administratoriui pateikti pranešimą,
  jog norint išvysti pakeitimus reikėtų prisijungti per naujo prie sistemos.
}

\xbugN{19}
{
  Prisijungiant prašo „Vartotojo vardas“, o naudotojų administravimui
  yra skirta kortelė „Naudotojai“.
}
{
  Norint ištaisyti šią klaidą, pakanka suvienodinti pavadinimus.
}

\xbug{20}{Pataisytas interfeisas: dabar rodomas atitinkamas klaidos pranešimas}
{
  Įvedimo laukas sistemų administravimo interfeise leidžia įvesti bet kokį skaičių, 
  bet jei jis nėra „1“, tai jokio klaidos pranešimo neparodo ir pakeičia į tuščią.
}
{
  Pataisymas trivialus - paprasčiausiai parodyti klaidos pranešimą.
}

\xbug{21}{Papildomi menu punktai}
{
  Kaskart atsijungus ir vėl prisijungus, meniu „Rodyti“ atsiranda papildomas
  pasirinkimas „Pateikti visus rezultatus“.
}
{
  Tai yra algoritmavimo klaida.
}

\xbugN{22}
{
  Renkantis vietinės duomenų bazės failą, sistema neatsimena aplanko,
  kuriame paskutinį kartą buvo vykdytas failo pasirinkimas.
}
{
  Pataisymas analogiškas klaidai nr. 12.
}

\xbug{23}{Failų pasirinkimo dialoge nefiltruojami netinkami failai}
{
  Renkantis vietinės duomenų bazės failą, sistema leidžia pasirinkti
  bet kokio formato failą ir neparodo jokio pranešimo nei naudotojo sąsajoje,
  nei pranešimų žurnale.
}
{
  Reikia įdėti failų filtravimą pagal plėtinį.
}

\xbugN{24}
{
  Pasirinkus netinkamo formato vietinės duomenų bazės failą ir mėginant
  išsaugoti, sistema parodo pranešimą „Duomenys išsaugoti“, tačiau iš tikro sistemai
  nepavyksta to padaryti.
}
{
  Ši klaida yra susijusi su klaida nr. 23. Kolkas dar vietinės duomenų bazės
  interfeisas nėra pilnai realizuotas, užtat testavimo metu pasimato vietos,
  kurių realizavimas dar neįvykdytas.
}

\xbugN{25}
{
  Įvedus neegzistuojančios priemonės pavadinimą lentelėje „Planuojami kiekiai“,
  sistema lūžta neparodžiusi jokio klaidos pranešimo.
}
{
  Programiniu požiūriu tai pati sudėtingiausia klaida, nes ji buvo ne sistemos viduje,
  o C++ kompiliatoriuje, naudojamame sistemai kompiliuoti. Į sistemą buvo integruotas
  pakeitimas, išvengiantis šios kompiliatoriaus klaidos.
}
