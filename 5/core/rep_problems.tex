\xbug{1}{picture}
{
  Mygtukas „Rodyti“ yra ne savo vietoje (turi būti arčiau intervalo
  pasirinkimo laukų):
}
{
  Taip nuspręsta dėl to, kad mygtuką „Rodyti“ reikia paspausti tik tuomet, kai yra
  pakeičiamas rodomas intervalas. Keičiant intervalą ar jo granuliarumą kitais mygtukais,
  rezultatai yra atnaujinami automatiškai.
}

\xbugN{2}
{
  Pakeitus rezultatų skaičiavimo granuliarumą rezultatai neatsinaujina.
}
{
  Norint atnaujinti rezultatus, reikia dar papildomai paspausti „Rodyti“, o tai yra papildomas
  nereikalingas žingsnis, mažinantis naudotojo sąsajos efektyvumą.
}

\xbug{3}{Pradinė būsena, kurioje visų padalinių/IS rodymo režimai sutampa}
{
  Pakeitus skaičiavimo režimą, rezultatų rodymo režimas atsistato į pradinį.
}
{
  Rezultatų lange kiekvieno padalinio/IS vaizdavimo režimą galima reguliuoti
  keliais skirtingais parametrais. Pagal savo poreikius sukonfigūravus kelis
  padalinius/IS ir pakeitus skaičiavimo režimą, pvz. iš IS į padalinių
  apkrovų skaičiavimą, sistema „pamiršta“, koks vaizdavimo režimas buvo
  parinktas kiekvienam padaliniui/IS. Dėl to labai stipriai nukenčia
  efektyvumas, nes norint susigrąžinti seną konfigūraciją, reikia sugaišti
  nemažai laiko.
}

\xbug{4}{Mygtukas „Atstatyti“ šalia „Išsaugoti“ ir „Mažiausios apkrovos intervalai“}
{
  Netyčia paspaudus „Atstatyti“, naudotojas gali prarasti visus pakeitimus,
  atliktus po paskutinio išsaugojimo.
}
{
  Mygtukas „Atstatyti“ yra negrįžtama operacija, kuri pašalina visus
  padarytus pakeitimus. Kadangi tai yra tokia destruktyvi operacija, kurią
  nesunkiai galima iškviesti per klaidą, tai vertėtų įdėti modalinį dialogą,
  prašantį patvirtinti naudotojo veiksmą.
}

\xbug{5}{Pataisyta apžvalgos sąsaja su tinkama antrašte}
{
  Apžvalgos sąsajoje nesimatė, koks rodymo režimas dabar yra aktyvus.
}
{
  Kadangi apžvalgos sąsajoje yra pateikiama viena lentelė/grafikas, kuriame
  vaizduojami apibendrinti duomenys, tai naudotojui pirmą kartą įėjus į šią
  sąsają (arba sugrįžus po pertraukos) gali būti neaišku, kas konkrečiai
  toje lentelėje/grafike yra vaizduojama.
}

\xbugN{6}
{
  Ištrynus įrašus iš bet kurios lentelės, rezultatai automatiškai neatsinaujina.
}
{
  Keičiant duomenis arba įdedant naujus įrašus, sistema automatiškai
  atnaujina rezultatus pagal naudotojo atliktus veiksmus. Tačiau pašalinus
  vieną arba daugiau įrašų sistema rezultatų neperskaičiuoja ir naudotojo
  sąsajos neatnaujina.
}

\xbug{7}{Originalus ir pakoreguotas intervalai savaitiniame granuliarume}
{
  Pasirinkus savaitinį granuliarumą, sistema pati modifikuoja intervalo
  galus, nepaaiškindama, kodėl taip daro.
}
{
  Tam, kad būtų supaprastinti skaičiavimai ir padidintas jų tikslumas,
  sistema automatiškai pakoreguoja intervalo galus taip, kad šie atitiktu
  turimus duomenis, prasidėtų ties granuliaraus vieneto pradžia (jei
  pasirinktas savaitinis rodymas – tai pirmadieniu, jei mėnesinis – tai
  pirma mėnesio diena ir t.~t.), tačiau naudotojui gali būti neaišku, dėl
  ko intervalai savaime pakinta. Dėl to reikėtų įdėti pranešimą, kuriame tai
  paaiškinama.
}

\xbug{8}{Ne iš eilės pažymėtos eilutės}
{
  Šalinant ne iš eilės pažymėtas eilutes, sistema arba išmeta ne išmetimui
  skirtas eilutes, arba nulūžta.
}
{
  Tai yra paprasčiausia algoritmavimo klaida kode, atsakingame už elementų
  pašalinimą iš lentelių.
}

\xbug{9}{Užblokuota prieiga prie sistemų režimo}
{
  Administratoriaus sąsajoje naudotojui įjungus prieigą prie sistemų
  režimo, šis vis tiek naudotojui netampa prieinamu.
}
{
  Tai yra dar viena algoritmavimo klaida, šį kartą dėl sumaišytų privilegijų
  deskriptorių.
}

\xbugN{10}
{
  Į pranešimų žurnalą keliauja ne visi pranešimai.
}
{
  Kadangi pranešimų žurnalo funkcija yra kaupti pranešimus, tai šis turėtų tai atlikti
  visiems sistemos pateikiamiems pranešimams. Kadangi dabar taip nėra, tai nukenčia
  nuspėjamumas.
}

\xbug{11}{Ištrauka iš pranešimų žurnalo apie įvykusias importo klaidas}
{
  Pašalinus viską iš duomenų bazės ir importuojant duomenis iš failo,
  sistema skundžiasi, jog tokie elementai jau yra, nors ką tik buvo pašalinti.
}
{
  Ši problema vėl susijusi su algoritmavimo klaida kode, atsakingame už
  elementų pašalinimą iš konteinerių.
}

\xbugN{12}
{
  Importuojant failą, sistema neatsimena aplanko, kuriame paskutinį
  kartą buvo vykdytas importavimas.
}
{
  Gana smulki klaida, nes duomenys į sistemą įkeliami yra gana retai. Tačiau norint
  padidinti našumą, galima įsiminti paskutinį aplanką, iš kurio failas importuotas.
}

\xbugN{13}
{
  Naudotojų administravimo sąsaja ne visose platformose reaguoja į
  aktyvaus naudotojo pasikeitimus.
}
{
  Tai problema, kuri pasireiškia tik kai kuriose platformose/OS, potencialiai susijusi
  su galima klaida Qt biblitekos kode, atsakingame už pranešimus apie paspaudimus ant 
  lentelės.
}

\xbug{14}{Neigiamos apkrovos}
{
  Kai kuriais atvejais dėl panaudoto apkrovų skaičiavimo algoritmo,
  apkrovos gaunasi neigiamos.
}
{
  Tam, kad duomenys rezultatuose atrodytų tolydžiau, sistema naudoja
  ekstrapoliaciją apkrovų skaičiavimo algoritme, tačiau dėl to kai kuriais
  atvejais gaunasi neigiamos apkrovos. Norint šią klaidą ištaisyti, reikia
  padaryti, kad intervaluose, kuriuose pritaikius ekstrapoliaciją gaunamos
  neigiamos reikšmės, ji nebūtų nenaudojama. Taip pat praverstų galimybė
  laikinai išvis išjungti ekstrapoliaciją, jei jos naudotojas nepageidauja.
}

\xbug{15}{Ištaisyta naudotojo sąsaja prognozes žymi žaliai}
{
  Grafikuose aiškiai nesimato, ties kuria vieta prasideda prognozės
}
{
  Vieta, ties kuria sistema pradeda skaičiuoti apkrovas remdamasi fiktyviais
  prognozuojamais duomenimis, turėtų būti aiškiai išryškinta tam, kad
  palengvintų naudotojo orientaciją duomenyse.
}

\xbug{16}{Privilegijos pasirinkimas administratoriaus sąsajoje}
{
  Iš karto nėra aišku, ką reiškia pasirinkimas „Leisti ir rašyti“.
}
{
  Kiekvienas naudotojas turi tam tikras privilegijas duomenų bazių lentelėms.
  Iš viso sistemoje yra trys privilegijų lygiai:
  \begin{itemize}
    \item Slėpti – nerodyti minimos sąsajos.
    \item Leisti tik skaityti – leisti tik peržiūrėti duomenis.
    \item Leisti ir rašyti – leisti ir skaityti, ir rašyti duomenis.
  \end{itemize}
  Kad būtų aiškiau, reikia paskutinį privilegijos lygį pervadinti.
}

\xbug{17}{Užblokuota administratoriaus sąsaja, nes nepasirinktas joks naudotojas}
{
  Nėra aišku, kad norint, jog kortelės „Naud. adm.“ turinys taptų aktyvus,
  reikia kortelėje „Naudotojai“ paspausti ant kurio nors iš naudotojų.
}
{
  Reikėtų uždėti pranešimą, skatinantį pasirinkti aktyvų naudotoją.
}

\xbugN{18}
{
  Naudotojui pačiam sau pasikeitus, jog (ne)rodytų apžvalgos kortelę, ji
  atsiranda tik iš naujo prisijungus.
}
{
  Šios klaidos pataisymas greičiausiai realizuotas nebus, nes tik administratorius
  turi teisę keisti privilegijas, o tokio menkai nereikalingo pataisymo realizacija
  yra pakankamai sudėtinga. Kaip alternatyvą, galima administratoriui pateikti pranešimą,
  jog norint išvysti pakeitimus reikia prie sistemos prisijungti iš naujo.
}

\xbugN{19}
{
  Prisijungiant prašo „Vartotojo vardas“, o naudotojų administravimui
  yra skirta kortelė „Naudotojai“.
}
{
  Norint ištaisyti šią klaidą, pakanka suvienodinti pavadinimus.
}

\xbug{20}{Pataisyta sąsaja: dabar rodomas atitinkamas klaidos pranešimas}
{
  Įvedimo laukas sistemų administravimo sąsajoje leidžia įvesti bet kokį
  skaičių, bet jei jis nėra „1“, tai jokio klaidos pranešimo neparodo ir lauką
  pakeičia į tuščią.
}
{
  Pataisymas trivialus – paprasčiausiai parodyti klaidos pranešimą.
}

\xbug{21}{Papildomi meniu punktai}
{
  Kaskart atsijungus ir vėl prisijungus, meniu „Rodyti“ atsiranda papildomas
  pasirinkimas „Pateikti visus rezultatus“.
}
{
  Tai yra algoritmavimo klaida.
}

\xbugN{22}
{
  Renkantis vietinės duomenų bazės failą, sistema neatsimena aplanko,
  kuriame paskutinį kartą buvo vykdytas failo pasirinkimas.
}
{
  Pataisymas analogiškas klaidai Nr. 12.
}

\xbug{23}{Failų pasirinkimo dialoge nefiltruojami netinkami failai}
{
  Renkantis vietinės duomenų bazės failą, sistema leidžia pasirinkti
  bet kokio formato failą ir neparodo jokio pranešimo nei naudotojo sąsajoje,
  nei pranešimų žurnale.
}
{
  Reikia įdėti failų filtravimą pagal plėtinį.
}

\xbugN{24}
{
  Pasirinkus netinkamo formato vietinės duomenų bazės failą ir mėginant
  išsaugoti, sistema parodo pranešimą „Duomenys išsaugoti“, tačiau iš tikrųjų
  sistemai nepavyksta to padaryti.
}
{
  Ši klaida yra susijusi su klaida Nr. 23. Kol kas dar vietinės duomenų bazės
  sąsaja nėra pilnai realizuota, todėl testavimo metu pasimato vietos, kurių
  realizavimas dar neįvykdytas.
}

\xbugN{25}
{
  Įvedus neegzistuojančios priemonės pavadinimą lentelėje „Planuojami kiekiai“,
  sistema lūžta neparodžiusi jokio klaidos pranešimo.
}
{
  Programiniu požiūriu tai pati sudėtingiausia klaida, nes ji buvo ne sistemos viduje,
  o C++ kompiliatoriuje, naudojamame sistemai kompiliuoti. Į sistemą buvo integruotas
  pakeitimas, išvengiantis šios kompiliatoriaus klaidos.
}
