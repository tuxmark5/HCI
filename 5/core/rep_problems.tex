\xbug{1}{picture}
{
  Mygtukas „Rodyti“ yra ne savo vietoje (turi būti arčiau intervalo
  pasirinkimo laukų):
}
{
  Taip nuspręsta dėl to, kad mygtuką rodyti reikia paspausti tik tuomet, kai yra
  pakeičiamas rodomas intervalas. Keičiant intervalą ar jo granuliarumą kitais mygtukais,
  rezultatai yra atnaujinami automatiškai.
}

\xbugN{2}
{
  Pakeitus rezultatų skaičiavimo granuliarumą rezultatai neatsinaujina.
}
{
  Norint atnaujinti rezultatus, reikia dar papildomai paspausti „Rodyti“, o tai yra papildomas
  nereikalingas žingsnis, mažinantis interfeiso efektyvumą.
}

\xbug{3}{Pradinė būsena, kurioje visų padalinių/IS rodymo rėžimai sutampa}
{
  Pakeitus skaičiavimo rėžimą, rezultatų rodymo rėžimas atsistato į pradinį.
}
{
  Rezultatų lange kiekvieno padalinio/IS vaizdavimo rėžimą galima reguliuoti keliais skirtingais
  parametrais. Pagal savo poreikius sukonfigūravus kelis padalinius/IS ir pakeitus skaičiavimo rėžimą pvz. iš IS į padalinių apkrovų skaičiavimą, sitema „pamiršta“, koks vaizdavimo rėžimas
  buvo parinktas kiekvienam padaliniui/IS. Dėl to labai stipriai nukenčia efektyvumas, nes norint
  atgal susigrąžinti seną konfigūraciją, reikia sugaišti nemažai laiko.
}

\xbug{4}{Mygtukas „Atstatyti“ šalia „Išsaugoti“ ir „Mažiausios apkrovos intervalai“}
{
  Netyčia paspaudus „Atstatyti“, naudotojas gali prarasti visus pakeitimus
  atliktus po paskutinio išsaugojimo.
}
{
  Mygtukas „Atstatyti“ yra negrįžtama operacija, kuri pašalina visus padarytus pakeitimus.
  Kadangi tai yra tokia destruktyvi operacija, kurią nesunkiai galima iškviesti per klaidą,
  tai vertėtų įdėti modalinį dialogą, prašantį patvirtinti vartotojo veiksmą.
}

\xbug{5}{Pataisyta apžvalgos sąsaja su tinkama antrašte}
{
  Apžvalgos sąsajoje nesimatė, koks rodymo rėžimas dabar yra aktyvus.
}
{
  Kadangi apžvalgos interfeise yra pateikiama viena lentelė/grafikas, kuriame vaizduojami
  apibendrinti duomenys, tai vartotojui pirmą kartą įėjus į šį interfeisą (arba sugrįžus po
  pertraukos) gali būti neaišku, kas konkrečiai toje lentelėje/grafike yra vaizduojama.
}

\xbugN{6}
{
  Ištrynus įrašus iš bet kurios lentelės, rezultatai automatiškai neatsinaujina.
}
{
  Keičiant duomenis, arba įdedant naujus įrašus, sistema automatiškai atnaujina rezultatus
  pagal vartotojo atliktus veiksmus. Tačiau pašalinus vieną arba daugiau įrašų sistema
  rezultatų neperskaičiuoja ir interfeiso neatnaujina.
}
