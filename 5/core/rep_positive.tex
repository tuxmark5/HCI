\begin{itemize}
  \item Vienu metu galima matyti ir pradinius duomenis, ir rezultatus
  (naudotojo kontrolė ir laisvė):
  \ximageS{s:two_panes}{Pradiniai duomenys ir rezultatai}{screens/two_panes.png}  

  \item Galimybė suskaičiuoti senų ir naujų apkrovų, gautų įvykdžius pradinių
  duomenų pakeitimus, skirtumus (naudotojo kontrolė ir laisvė):
  \ximageS{s:deltas}{Dinaminis skirtumų skaičiavimas}{screens/deltas.png}

  \item Galima paslėpti pradinius duomenis ir taip padidinti rezultatų užimamą
  plotą ekrane:
  \ximageS{s:hide_input}{Pradinių duomenų paslėpimas}{screens/hide_input.png}

  \item RSys neturi modalinių dialogų, visi pranešimai yra netrikdantys, o praėjus
    nuo pranešimo ilgio priklausančiam laiko tarpu, jie patys dingsta
    (lankstumas ir efektyvumas):
  \ximageS{s:uninvasive_messages}{Įvedamų duomenų validavimas}{screens/uninvasive_messages.png}

  \item Mažiausios apkrovos intervalo paieška taip pat yra realizuota nemodaliniu principu
  (vientisumas):
  \ximageS{s:interval_search}{Mažiausios apkrovos paieška}{screens/interval_search.png}

  \item Įvedami duomenys yra realiu laiku validuojami, o įvykus klaidai yra iškart parodomas
  atitinkamas pranešimas (klaidų vengimas):
  \ximageS{s:input_validation}{Įvedamų duomenų validavimas}{screens/input_validation.png}

  \item Galima palyginti arba visų, arba tik dominančių padalinių apkrovas nurodytame intervale
  (lankstumas ir efektyvumas):
  \ximageS{s:summary_bars}{Apžvalgos interfeisas}{screens/summary_bars.png}

  \item Suteikiama galimybė dirbti ir prisijungus prie centrinės duomenų bazės, ir vietiniame
  kompiuteryje (lankstumas ir efektyvumas):
  \ximageS{s:login_remote}{Prisijungimo lango dalis}{screens/login_remote.png}

  \item
    Importuojant duomenis iš failo, sistema ne tik pati atspėja, kas
    yra importuojama, bet ir leidžia naudotojui pasirinkti, ką
    importuoti, o ką – ne (lankstumas ir efektyvumas):
  \ximageS{s:import_settings}{Importavimo nustatymai}{screens/import_settings.png}

  \item Atlikus duomenų importavimą, yra tiksliai parodoma ko ir kiek buvo importuota
  (būsenos matomumas, palengvinta klaidų diagnozė):
  \ximageS{s:import_feedback}{Importavimo pabaigos pranešimas}{screens/import_feedback.png}

  \item Darbo metu (pavyzdžiui, importuojant duomenis) įvykus klaidoms, detalią informaciją
    galima rasti pranešimų žurnale (būsenos matomumas, palengvinta klaidų diagnozė):
  \ximageS{s:log}{Pranešimų žurnalas}{screens/log.png}
\end{itemize}
