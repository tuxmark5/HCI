Vertintojai ir jų indentifikatoriai:
\xtableu
{
  w [ 1 | 3 ]
  a [ p | p ]
  h [ Kodas | Vertintojas ]
  e [ A | Audrius Šaikūnas ]
  e [ E | Egidijus Lukauskas ]
  e [ J | Justinas Jusevičius ]
  e [ M | Martynas Budriūnas ]
  e [ V | Vytautas Astrauskas ]
}

Klaidų indentifikatorių paaiškinimai:
\xtableu
{
  w [ 1 | 3 ]
  a [ p | p ]
  h [ Sunkumas | Reikšmė ]
  e [ 4 | Katastrofinė problema ]
  e [ 3 | Rimta problema ]
  e [ 2 | Smulki problema ]
  e [ 1 | Kosmetinė problema ]
  e [ 0 | Nėra problemos ]
  e [ Vid. | Įverčių vidurkis ]
}

\xtableu
{
  a [ p   | p       | p | p | p | p | p | p | p | p | p | p | p    | p    ]
  w [ 2   | 20      | 2 | 2 | 2 | 2 | 2 | 2 | 2 | 2 | 2 | 2 | 2    | 8    ]
spec[ Nr. | Aprašas | Rado              | Sunkumas                 | Lok. ]
  h [     |         | A | E | J | M | V | A | E | J | M | V | Vid. |      ]
  e [ 1   | Mygtukas "Rodyti" yra ne savo vietoje (turi būti arčiau intervalo 
  pasirinkimo laukų)
                    | + |   |   |   |   | 2 |   |   |   |   |      | Pagr. sąsaja ]
  e [ 2   | Pakeitus rezultatų skaičiaivmo granuliarumą rezultatai neatsinaujina
                    | + |   |   |   |   | 3 |   |   |   |   |      | Rezult. sąsaja ]
  e [ 3   | Pakeitus skaičiavimo rėžimą, rezultatų rodymo rėžimas atsistato
  į pradinį.
                    | + |   |   |   |   | 3 |   |   |   |   |      | Pagr. sąsaja ]
  e [ 4   | Netyčia paspaudus "Atstatyti", naudotojas gali prarasti savo visus duomenis
  iki paskutinio išsaugojimo. 
                    | + |   |   |   |   | 2 |   |   |   |   |      | Pagr. sąsaja ]
  e [ 5   | Apžvalgos interfeise nesimato, koks rodymo rėžimas dabar yra aktyvus.
                    | + |   |   |   |   | 2 |   |   |   |   |      | Pagr. sąsaja ]
  e [ 6   | Ištrynus įrašus iš bet kurios lentelės, rezultatai automatiškai neatsinaujina.
                    | + |   | + |   |   | 3 |   |   |   |   |      | Pagr. sąsaja ]
  e [ 7   | Pasirinkus savaitinį granuliaruma sistema pati modifikuoja intervalo galus,
  nepaaiškindama kodėl taip daro.
                    | + |   | + |   |   | 4 |   |   |   |   |      | Pagr. sąsaja ]
  e [ 8   | Šalinant ne iš eilės pažymėtas eilutes, sistema arba išmeta ne išmetimui
  skirtas eilutes, arba nulūžta.  
                    |   |   | + |   |   | 3 |   |   |   |   |      | Priem. sąsaja ]
  e [ 9   | Administratoriaus interfeise vartotojui įjungus prieeigą prie sistemų
  rėžimo, šis vis tiek vartotojui netampa prieinamu.     
                    |   |   | + |   |   | 2 |   |   |   |   |      | Admin. sąsaja ]
  e [ 10  | Į pranešimų žurnalą keliauja ne visi pranešimai     
                    |   |   | + |   |   | 3 |   |   |   |   |      | Pranešimų žurnalas ]
  e [ 11  | Pašalinus viską iš duomenų bazės ir importuojant duomenis iš failo,
  sistema skundžiasi, jog tokie elementai jau yra, nors tiką buvo pašalinti.     
                    |   |   | + |   |   | 1 |   |   |   |   |      | Pagr. sąsaja ]
  e [ 12  | Importuojant failą sistema neatsimena direktorijos, kur paskutinį
  kartą buvo vykdytas importavimas.   
                    |   |   | + |   |   | 3 |   |   |   |   |      | Import. sąsaja ]
  e [ 13  | Vartotojų administravimo interfeisas ne visose platformose reaguoja į
  aktyvaus naudotojo pasikeitimus  
                    |   |   |   | + |   | 3 |   |   |   |   |      | Adm. sąsaja ]
  e [ 14  | Kai kuriais atvejais dėl panaudoto apkrovų skaičiavimo algoritmo,
  apkrovos gaunasi neigiamos.
                    | + |   | + | + |   | 2 |   |   |   |   |      | Rezult. sąsaja ]
  e [ 15 | Grafikuose aiškiai nesimato, ties kuria vieta prasideda prognozės
                    | + |   |   |   | + |   |   |   |   |   |      | Adm. sąsaja ]
}
