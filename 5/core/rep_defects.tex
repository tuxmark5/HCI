Vertintojai ir jų indentifikatoriai:
\xtableu
{
  w [ 1 | 3 ]
  a [ p | p ]
  h [ Kodas | Vertintojas ]
  e [ A | Audrius Šaikūnas ]
  e [ E | Egidijus Lukauskas ]
  e [ J | Justinas Jusevičius ]
  e [ M | Martynas Budriūnas ]
  e [ V | Vytautas Astrauskas ]
}

Klaidų indentifikatorių paaiškinimai:
\xtableu
{
  w [ 1 | 3 ]
  a [ p | p ]
  h [ Sunkumas | Reikšmė ]
  e [ 4 | Katastrofinė problema ]
  e [ 3 | Rimta problema ]
  e [ 2 | Smulki problema ]
  e [ 1 | Kosmetinė problema ]
  e [ 0 | Nėra problemos ]
  e [ Vid. | Įverčių vidurkis ]
}

\xtableu
{
  a [ p   | p       | p | p | p | p | p | p | p | p | p | p | p    | p    ]
  w [ 2   | 20      | 2 | 2 | 2 | 2 | 2 | 2 | 2 | 2 | 2 | 2 | 2    | 8    ]
spec[ Nr. | Aprašas | Rado              | Sunkumas                 | Lok. ]
  h [     |         | A | E | J | M | V | A | E | J | M | V | Vid. |      ]
  e [ 1   | Mygtukas „Rodyti“ yra ne savo vietoje (turi būti arčiau intervalo
  pasirinkimo laukų).
                    | + |   |   |   |   | 2 | 1 | 1 | 2 | 1 | 1.4  | Pagr. sąsaja ]
  e [ 2   | Pakeitus rezultatų skaičiavimo granuliarumą rezultatai neatsinaujina.
                    | + |   |   |   |   | 2 | 2 | 3 | 2 | 3 | 2.4  | Rezult. sąsaja ]
  e [ 3   | Pakeitus skaičiavimo režimą, rezultatų rodymo režimas atsistato
  į pradinį.
                    | + |   |   |   |   | 3 | 3 | 3 | 3 | 3 | 3.0  | Pagr. sąsaja ]
  e [ 4   | Netyčia paspaudus „Atstatyti“, naudotojas gali prarasti visus pakeitimus,
  atliktus po paskutinio išsaugojimo.
                    | + |   |   |   |   | 3 | 2 | 2 | 3 | 3 | 2.6  | Pagr. sąsaja ]
  e [ 5   | Apžvalgos sąsajoje nesimato, koks rodymo režimas dabar yra aktyvus.
                    | + |   |   |   |   | 2 | 2 | 2 | 2 | 2 | 2.0  | Pagr. sąsaja ]
  e [ 6   | Ištrynus įrašus iš bet kurios lentelės, rezultatai automatiškai neatsinaujina.
                    | + |   | + |   |   | 3 | 3 | 3 | 3 | 3 | 3.0  | Pagr. sąsaja ]
  e [ 7   | Pasirinkus savaitinį granuliarumą sistema pati modifikuoja intervalo galus,
  nepaaiškindama, kodėl taip daro.
                    | + |   | + |   |   | 3 | 4 | 4 | 2 | 3 | 3.2  | Pagr. sąsaja ]
  e [ 8   | Šalinant ne iš eilės pažymėtas eilutes, sistema arba išmeta ne išmetimui
  skirtas eilutes, arba nulūžta.
                    |   |   | + |   |   | 4 | 4 | 3 | 4 | 4 | 3.8  | Priem. sąsaja ]
  e [ 9   | Administratoriaus sąsajoje naudotojui įjungus prieigą prie sistemų
  režimo, šis vis tiek naudotojui netampa prieinamu.
                    |   |   | + |   |   | 3 | 3 | 4 | 4 | 3 | 3.4  | Admin. sąsaja ]
  e [ 10  | Į pranešimų žurnalą keliauja ne visi pranešimai
                    |   | + | + | + |   | 2 | 2 | 2 | 3 | 2 | 2.2  | Pranešimų žurnalas ]
  e [ 11  | Pašalinus viską iš duomenų bazės ir importuojant duomenis iš failo,
  sistema skundžiasi, jog tokie elementai jau yra, nors ką tik buvo pašalinti.
                    |   |   | + |   |   | 3 | 3 | 4 | 4 | 4 | 3.6  | Pagr. sąsaja ]
  e [ 12  | Importuojant failą, sistema neatsimena aplanko, kuriame paskutinį
  kartą buvo vykdytas importavimas.
                    |   |   | + |   | + | 1 | 1 | 2 | 2 | 1 | 1.4  | Import. sąsaja ]
  e [ 13  | Naudotojų administravimo sąsaja ne visose platformose reaguoja į
  aktyvaus naudotojo pasikeitimus.
                    |   |   |   | + |   | 4 | 4 | 4 | 4 | 4 | 4.0  | Adm. sąsaja ]
  e [ 14  | Kai kuriais atvejais dėl panaudoto apkrovų skaičiavimo algoritmo,
  apkrovos gaunasi neigiamos.
                    | + |   | + | + |   | 3 | 3 | 4 | 4 | 4 | 3.6  | Rezult. sąsaja ]
  e [ 15 | Grafikuose aiškiai nesimato, ties kuria vieta prasideda prognozės
                    | + |   |   |   | + | 2 | 2 | 3 | 3 | 3 | 2.6  | Adm. sąsaja ]
  e [ 16 | Iš karto nėra aišku, ką reiškia pasirinkimas „Leisti ir rašyti“.
                    |   |   |   |   | + | 1 | 1 | 1 | 1 | 2 | 1.2  | Adm. sąsaja ]
% Galimas sprendimas: Pakeisti į „Skaityti ir rašyti“. Kitus
% atitinkamai į „Tik skaityti“, „Nerodyti“ bei sugrūsti viską
% į rėmelį su antrašte „Leidimai“.
  e [ 17 | Nėra aišku, kad norint, jog kortelės „Naud. adm.“ turinys taptų aktyvus,
  reikia kortelėje „Naudotojai“ paspausti ant kurio nors iš naudotojų.
                    |   |   |   |   | + | 2 | 3 | 2 | 2 | 3 | 2.4  | Adm. sąsaja ]
% Galimas sprendimas: „Naud. adm.“ kortelės nerodyti tarp pasirinkimų,
% o ją automatiškai parodyti, kai naudotojas pasirenka „Naudotojai“.
  e [ 18 | Naudotojui pačiam sau pasikeitus, jog (ne)rodytų apžvalgos kortelę, ji
  atsiranda tik iš naujo prisijungus.
                    |   |   |   | + | + | 0 | 3 | 2 | 2 | 3 | 2.0  | Adm. sąsaja ]
% Galimas sprendimas: Parodyti pranešimą, jog pakeitimai įsigalios tik
% prisijungus iš naujo.
  e [ 19 | Prisijungiant prašo „Vartotojo vardas“, o naudotojų administravimui
  yra skirta kortelė „Naudotojai“.
                    |   |   |   |   | + | 1 | 1 | 1 | 1 | 1 | 1.0  | Adm. sąsaja ]
% Galimas sprendimas: „Vartotojo vardas“ → „Naudotojo vardas“.
  e [ 20 | Įvedimo laukas leidžia įvesti bet kokį skaičių, bet jei jis nėra
  „1“, tai jokio klaidos pranešimo neparodo ir pakeičia į tuščią.
                    |   |   |   |   | + | 2 | 2 | 2 | 2 | 2 | 2.0  | IS adm. sąsaja ]
% Galimas sprendimas: Parodyti pranešimą, kad „1“ – taip, visa kita – „ne“.
  e [ 21 | Kaskart atsijungus ir vėl prisijungus, meniu „Rodyti“ atsiranda papildomas
  pasirinkimas „Pateikti visus rezultatus“.
                    |   | + |   |   |   | 3 | 2 | 3 | 2 | 2 | 2.4  | Pagr. sąsaja ]
  e [ 22 | Renkantis vietinės duomenų bazės failą, sistema neatsimena aplanko,
  kuriame paskutinį kartą buvo vykdytas failo pasirinkimas.
                    |   | + |   |   |   | 1 | 1 | 1 | 1 | 1 | 1.0  | Pagr. sąsaja ]
  e [ 23 | Renkantis vietinės duomenų bazės failą, sistema leidžia pasirinkti
  bet kokio formato failą ir neparodo jokio pranešimo nei naudotojo sąsajoje,
  nei pranešimų žurnale.
                    |   | + |   |   |   | 3 | 3 | 3 | 3 | 3 | 3.0  | Pagr. sąsaja ]
  e [ 24 | Pasirinkus netinkamo formato vietinės duomenų bazės failą ir mėginant
  išsaugoti, sistema parodo pranešimą „Duomenys išsaugoti“, tačiau iš tikro sistemai
  nepavyksta to padaryti.
                    |   | + |   |   |   | 3 | 3 | 4 | 4 | 4 | 3.6  | Pagr. sąsaja ]
  e [ 25 | Įvedus neegzistuojančios priemonės pavadinimą lentelėje „Planuojami kiekiai“,
  sistema lūžta neparodžiusi jokio klaidos pranešimo.
                    |   | + |   |   |   | 4 | 3 | 4 | 4 | 4 | 4.75 | Pagr. sąsaja ]
}
