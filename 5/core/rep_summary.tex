Naudotojo sąsajos euristinį vertinimą atliko visi komandos nariai jiems patogiu
laiku. Kadangi kuriama perkeliama darbalaukinė informacinė sistema, tai ji buvo
testuojama keliose operacinėse sistemose bei platformose vienu metu
(daugiausiai Ubuntu 11.10 64-bit ir Windows XP). Buvo atrasta ir klaidų,
būdingų tam tikrai OS (konkrečiai – Windows).

Testavimas buvo vykdomas trimis etapais: pirmajame etape testuotojai atliko vieną iš
pagrindinių užduočių (apkrovų testavimą). Kadangi visi testuotojai jau yra gerai susipažinę
su sistema, tai šiame etape kėblumų nekilo. Antrajame etape testuotojai bandė taip paveikti
sistemos elgseną, kad šios elgsena sutriktų – bandant įvesti nelogiškus ar nevalidžius duomenis
ar sukuriant kokį kitą mažai tikėtiną vykdymo scenarijų, kuriame sistemos elsena nukrypsta
nuo apibrėžtų normos ribų. Atliekant šią „grubaus“ testavimo stadiją, pavyko rasti nemažai
klaidų ir jas ištaisyti. Galiausiai trečiajame etape buvo bandoma pereiti per visą sistemos
teikiamą funkcionalumą (paspausti kiekvieną mygtuką ir pan.) ir taip atrasti galimus defektus
naudotojo sąsajoje arba su juo susijusioje elgsenoje.

Iš visų rastų problemų, šios laikomos pačiomis svarbiausiomis:
\xtableu
{
  a [ p | p | p | p ]
  w [ 1 | 4 | 1 | 2 ]
  h [ Nr. | Aprašas | Rado | Pažeistas principas ]
  e [  7  | Pasirinkus savaitinį granuliarumą sistema pati modifikuoja intervalo galus,
  nepaaiškindama, kodėl taip daro. | J | Nuspėjamumas ]
  e [  8  | Šalinant ne iš eilės pažymėtas eilutes, sistema arba išmeta ne išmetimui
  skirtas eilutes, arba nulūžta.  | J | Nuspėjamumas ]
  e [ 11  | Pašalinus viską iš duomenų bazės ir importuojant duomenis iš failo,
  sistema skundžiasi, jog tokie elementai jau yra, nors ką tik buvo pašalinti. | J | Nuspėjamumas ]
  e [ 13  | Naudotojų administravimo sąsaja ne visose platformose reaguoja į
  aktyvaus naudotojo pasikeitimus. | M | Nuspėjamumas, darna ]
  e [ 14  | Kai kuriais atvejais, dėl panaudoto apkrovų skaičiavimo algoritmo,
  apkrovos gaunasi neigiamos. | J | Nuspėjamumas ]
  e [ 21  | Kaskart atsijungus ir vėl prisijungus, meniu „Rodyti“ atsiranda papildomas
    pasirinkimas „Pateikti visus rezultatus“. | E | Darna ]
  e [ 25  | Įvedus neegzistuojančios priemonės pavadinimą lentelėje „Planuojami kiekiai“,
  sistema lūžta neparodžiusi jokio klaidos pranešimo. | E | Nuspėjamumas ]
}

Kadangi dauguma klaidų savo principu yra panašios, tai kiekvienai klaidai jos taisymo
rekomendacijos nebus pateikiamos, tačiau vietoj jos galima daryti apibendrintas klaidų
taisymo rekomendacijas:
\begin{itemize}
  \item Dauguma klaidų susijusios su nuspėjamumu kyla dėl naudotojo sąsajos
  programavimo klaidos. Tokių klaidų ištaisymas apima naudotojo sąsajos kode
  esančios klaidos pataisymą.
  \item Kita dalis klaidų, susijusi su nuspėjamumu, atsiranda dėl naudotojo užduočių
  koregavimo/automatizavimo. Pvz.: sistema automatiškai pareguliuodavo pateikiamus intervalus
  taip, kad šie tilptų į duomenis ir sudarytų sveiką savaičių/mėnesių/ketvirčių skaičių.
  Kadangi toks sistemos elgesys naudotojui gali sukelti nuostabą, į sistemą buvo
  įdėta atitinkama žinutė su paaiškinimu, kuri pasirodo, jei sistema koreguoja intervalo galus.
  \item Yra keletas klaidų, susijusių su darnos nebuvimu (pvz.: tas pats objektas yra vadinamas
  skirtingai skirtingose sąsajose). Tokios klaidos ištaisomos suvienodinant sąvokas.
  \item Taip pat yra keletas klaidų, susijusių su klaidų prevencijų nebuvimu.
  Pvz.: sistema netikrino, ar pateikiamas failas yra duomenų bazė, ar ne,
  todėl be jokio klaidos pranešimo galėjo atidaryti iš esmės bet kokį
  pateiktą failą, nepriklausomai nuo jo formato, kas, žinoma, yra klaidingas
  elgesys. Tokios klaidos taip pat yra taisomos įdedant papildomus funkcinius
  patikrinimus ir pranešimus.
\end{itemize}

\vspace{1cm}
Taip pat atrasta nemažai teigiamų dalykų, padidinančių panaudojamumą. Pavyzdžiui:
\begin{itemize}
  \item Klaidų žurnalas, kuriame pateikiama detali informacija apie įvykusias klaidas.
  \item Netrikdantys pranešimai, kurie patys užsidaro po atitinkamo laiko.
  \item ir kt.
\end{itemize}

Plačiau apie teigiamus sistemos aspektus žr. skyrelį „teigiami įspūdžiai“.
