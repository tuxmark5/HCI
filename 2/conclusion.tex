\xchapter{IŠVADOS}
Kaip matyti iš dokumente aprašytų esamų sąsajų nepatogumų, dažniausiai būna
pažeidžiami keli panaudojamumo principai iš karto viename interfeise. Tai
reiškia, jog yra kuriami tik tokie interfeisai, kurie yra estetiškai patrauklūs arba
kuriuos yra nesunku realizuoti programiškai, tačiau yra visiškai neatsižvelgiama į
panaudojamumo principus. Dar labiau situaciją pablogina faktas, jog dauguma
interfeisų trūkumų buvo rasti gerai žinomuose, brangiuose ir plačiai naudojamuose produktuose, 
kurie yra sukurti finansiškai turtingų kompanijų, kurios tiesiog vengia skirti išteklių
interfeisų tobulinimui.
