% Dokumento versijos numeris.
\newcommand\docversion{0.1}

% Komandos pavadinimas.
\newcommand\docname{Naudotojų poreikiai}


\newcounter{xreqCounter0}
\newcounter{xreqCounter1}

% Tools
\newcommand\xreq[2]
{
  \addtocounter{xreqCounter0}{1}
  \setcounter{xreqCounter1}{0}
  \label{#1}
  \colorbox{gray}{ \textbf { Reikalavimas R\arabic{xreqCounter0}. } } #2
  \par
}

\newcommand\yreq[2]
{
  \addtocounter{xreqCounter1}{1}
  \label{#1}
  \hspace{0.7cm}
  \colorbox{gray}{ \textbf { R\arabic{xreqCounter0}.\arabic{xreqCounter1}. } } #2
  \par
}

\newcommand\xchars[1]
{
  \xtable
  {
    w [ 2 | 6 ]
    a [ p | p ]
    #1
  }
}

\newcommand\xcharSystems[1]{ e [ Programų sistemos ir aparatūra, kuriomis moka naudotis | \unexpanded{#1} ] }
\newcommand\xcharSkills[1]{ e [ Naudotojų įgūdžiai ir motyvacija | \unexpanded{#1} ] }
\newcommand\xcharEnv[1]{ e [ Aplinka | \unexpanded{#1} ] }

\newcommand\xtasks[1]
{
  \xtable
  {
    w [ 6 | 2 | 2 ]
    a [ p | p | p ]
    h [ Užduotis | Trukmė | Dažnis ]
    #1
  }
}

\newcommand\xtask[3]
{
  e [ \unexpanded{#1}
    | \unexpanded{#2}
    | \unexpanded{#3}
    ]
}

\renewcommand\labelenumi{\arabic{enumi}.}
\renewcommand\labelenumii{\theenumi.\arabic{enumii}.}
\renewcommand\labelenumiii{\theenumi.\theenumii.\arabic{enumiii}.}
\renewcommand\labelenumiv{\theenumi.\theenumii.\theenumiii.\arabic{enumiv}.}
