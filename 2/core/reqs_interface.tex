\subsection{Dalykinės srities metaforos reikalavimai}
\xreq{r:metafora}
{
  Naudotojo sąsajų užduotys turi būti formuluojamos naudojantis darbo
  su skaičiuokle metafora. Naudotojui informacija pateikiama lentelių
  ir grafikų pavidalu.
  % FIXME DĖSTYTOJA Ar tinkamai suformuluota metafora? Ar galima
  % naudoti tokią metaforą?
  % FIXME Kas būtent turi būti apibrėžiama metaforų žodyne?

  Sistemoje naudojamų savokų paaiškinimai:
  \xtable
  {
    w [ 2 | 7 ]
    a [ p | p ]
    h [ Metafora | Paaiškinimas ]
    e [ Informacinė sistema |
      Sistema skirta informacijai apdoroti, formuoti (kurti) ir skleisti
      (siųsti ir gauti).]
    e [ IS | Termino „informacinė sistema“ trumpinys. ]
    e [ Padalinys | Organizacijos struktūrinis vienetas. ]
    e [ Paramos priemonė |
      Metodika, pagal kurią organizacijos klientai gali gauti paramą. ]
  }
}

\subsection{Formuluojamos užduotys}
\xreq{r:iface}{Sistemoje turi būti šie sąsajos:}
  \yreq{r:iface_base}{Bazinė – iš jos galima pasiekti visas kitas sąsajas
  bei atsijungti nuo sistemos.}
  \yreq{r:iface_pa}{Padalinių įvedimo – naudojantis ja galima įvesti
  informaciją apie padalinius.}
  \yreq{r:iface_pp}{Paramos priemonių įvedimo – naudojantis ja gali įvesti
  informaciją apie paramos priemones.}
  \yreq{r:iface_is}{IS įvedimo – naudojantis ja galima įvesti informaciją
  apie IS.}
  \yreq{r:iface_pa_is}{IS-Padalinių įvedimo – naudojantis ja galima
  nurodyti, kokie padaliniai kokias IS naudoja.}
  \yreq{r:iface_pa_pp}{Paramos administravimo kaštų įvedimo – naudojantis
  ja galima nurodyti, kiek kuriam padaliniui kainuoja konkrečios
  paramos priemonės vieno vieneto apdorojimas.}
  \yreq{r:iface_data}{Istorinių duomenų įvedimo – naudojantis juo
  galima įvesti istorinę informaciją apie apdorotas paraiškas.}
  \yreq{r:iface_visual}{Apkrovų vizualizacijos – jame galima vizualiai
  pamatyti kokios apkrovos tenka padaliniams ir sistemoms.}
  \yreq{r:iface_login}{Prisijungimo – per jį prisijungiama prie sistemos.}

\subsection{Užduočių formulavimo kalbos reikalavimai}
\xreq{r:ui_components}
{Visos užduotys turi būti formuluojamos naudojantis LIMP (langas, ikona,
meniu ir pelė) sąsajos priemonėmis.}

\xreq{r:ui_kbd_mouse}
{Duomenys sistemai turi būti pateikiami naudojantis pele ir klaviatūra.}

\xreq{r:ui_opt_mouse:1}
{Turi būti galimybė bent 80 procentų funkcijų, kurios nėra skirtos
sistemos parametrų keitimui, pasiekti naudojantis vien tik pele.}

\xreq{r:ui_opt_mouse:2}
{Apkrovų vizualizacijos sąsajai turi būti galimybė pateikti laiko
intervalą, kurio duomenis norima vizualizuoti.}

\subsection{Užduočių formulavimo būdo (protokolo) reikalavimai}
% FIXME Citata iš Čaplinsko: „Šiame poskyryje kiekvienai naudotojo
% sąsajai pateikiama atitinkama UML sekų diagrama, vaizduojanti
% naudotojo ir sistemos sąveiką formuluojant užduotis, ir tą diagramą
% paaiškinantis tekstas.

\xreq{r:ui_composition}
{Kiekviena iš pagrindinių sistemos naudotojo sąsajų turi būti
pateikiama vienoje arba keliose bazinės sąsajos kortelėse.}

\xreq{r:ui_base}{Bazinio interfeiso reikalavimai:}
  % FIXME DĖSTYTOJA Ar čia nėra per daug detalu? Bent jau man, po šito tai 
  % net ir eskizo nebereikia…
  \yreq{r:ui_base:mainwnd}{Pagrindiniame langą turi sudaryti menu, įrankių juosta(-os), 
  kortelių komponentas ir būsenos juosta. }
  \yreq{r:ui_base:messages}{Bazinis interfeisas yra atsakingas už įvairių pranešimų pateikimą 
  naudotojui.}
  
\xreq{r:ui_tables}{Visose naudotojo sąsajos lentelėse turi būti
galimybė keisti stulpelių  plotį.}

\subsection{Interfeiso darnos ir standartizavimo reikalavimai}
\xreq{r:ui_os_look}{Naudotojo sąsaja turi atitikti tos operacinės
sistemos sistemos išvaizdą, kurioje dirba taikomoji aplikacija.}

\xreq{r:ui_unicode}{Grafinėje naudotojo sąsajoje pateikiamas tekstas
ir pranešimai turi būti UTF-8 arba UTF-16 koduotėje.}

\xreq{r:ui_cua}{Teksto įvedimo laukai turi palaikyti CUA 
% FIXME Reikia paaiškinimo kas yra CUA. Vikipedijoje paieška pagal
% Common User Access duoda ne tai ko reikia.
įvedimo modelį.}

\xreq{r:ui_passwords}{Jokie konfidencialūs duomenys
% FIXME Gal tiesiog naudotojo slaptažodžiai negali būti rodomi naudotojo
% sąsajoje? Nes dabar galima suprasti, kad apskritai jokia konfidenciali
% informacija, tai yra ir aptarnautų padalinių kiekis, negali būti
% rodomas.
neturi būti rodomi naudotojo sąsajoje.}

\subsection{Pranešimų formulavimo reikalavimai}
\xreq{r:ui_messages:groups}{Pateikiami pranešimai turi būti
suskirstyti į 4 grupes: informacinio pobūdžio, perspėjimo, klaidų
ir kritiniai.
% FIXME Trūksta paaiškinimo, kuo skiriasi klaidų pranešimai, nuo kritinių.
% Ir apskritai, turbūt derėtų apibrėžti kiekvieną iš rūšių.
}

\xreq{r:ui_messages:icons}{
Skirtingiems tipams priklausantys pranešimai turi būti pažymėti
skirtingomis ikonėlėmis.}

\xreq{r:ui_messages:error}{Klaidos pobūdžio pranešimuose turi būti pateikta detali klaidos
priežastis.}

\xreq{r:ui_messages:critical}{Kritinio pobūdžio pranešimuose turi
būti galimybė peržiūrėti techninę klaidos priežasties
informaciją, kurios pagalba sistemos administratorius galėtų
identifikuoti ir pašalinti įvykusią problemą.}

\xreq{r:ui_messages:log}{Visi naudotojui pateikiami pranešimai kartu su
jų pateikimo laiku ir kontekstu turi būti saugomi pranešimų žurnale.}

\xreq{r:ui_messages:id}{Kiekvienai galimai klaidai turi būti suteiktas
unikalus numeris. Tokio paties pobūdžio klaidos yra laikomos
skirtingomis, jei jų atsiradimo priežastys yra skirtingos arba klaidos
pasirodo skirtingame kontekste. Pavyzdžiui, apibendrinta klaida „nepavyko
atidaryti failo“ turi būti išskirstyta į „nepavyko atidaryti
duomenų bazės failo“, „nepavyko atidaryti nustatymų failo“.}

\subsection{Sąsajos individualizavimo reikalavimai}
\xreq{r:tool_dock}{Turi būti galimybė paslėpti užduočių juostą.}
