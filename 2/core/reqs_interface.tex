\subsection{Dalykinės srities metaforos reikalavimai}
\xreq{r:metafora}
{
  Naudotojo sąsajų užduotys turi būti formuluojamos naudojantis darbo
  su skaičiuokle metafora. Naudotojui informacija pateikiama lentelių
  ir grafikų pavidalu.
  % FIXME Ar tinkamai suformuluota metafora? Ar galima naudoti tokią
  % metaforą?
  % FIXME Kas būtent turi būti apibrėžiama metaforų žodyne?
  
  Sistemoje naudojamų savokų paaiškinimai:
  \xtable
  {
    w [ 2 | 7 ]
    a [ p | p ]
    h [ Metafora | Paaiškinimas ]
    e [ Informacinė sistema |
      Sistema skirta informacijai apdoroti, formuoti (kurti) ir skleisti
      (siųsti ir gauti).]
    e [ IS | Termino „informacinė sistema“ trumpinys. ]
    e [ Padalinys | Organizacijos struktūrinis vienetas. ]
    e [ Paramos priemonė |
      Metodika, pagal kurią organizacijos klientai gali gauti paramą. ]
  }
}

\subsection{Formuluojamos užduotys}
\xreq{r:interfeisai}{Sistemoje turi būti šie interfeisai:}
  \yreq{r:iface_base}{Bazinis interfeisas.}
  \yreq{r:iface_pa}{Padalinių įvedimo interfeisas.}
  \yreq{r:iface_pp}{Paramos priemonių įvedimo interfeisas.}
  \yreq{r:iface_is}{IS įvedimo interfeisas.}
  \yreq{r:iface_pa_is}{IS-Padalinių įvedimo interfeisas.}
  \yreq{r:iface_pa_pp}{Paramos administravimo kaštų įvedimo interfeisas.}
  \yreq{r:iface_data}{Istorinių duomenų įvedimo interfeisas.}
  \yreq{r:iface_visual}{Apkrovų vizualizacijos interfeisas.}
  \yreq{r:iface_login}{Prisijungimo interfeisas.}

\subsection{Užduočių formulavimo kalbos reikalavimai}
\xreq{r:ui_components}
{Visos užduotys turi būti formuluojamos standartiniais grafinio vartotojo interfeiso 
komponentais.}

\xreq{r:ui_kbd_mouse}
{Duomenys į sistemą turi būti pateikiami naudojantis pele ir klaviatūra.}

\xreq{r:ui_opt_mouse:1}
{Turi būti galimybė daugumą duomenų sistemai pateikti vien tik klaviatūra, nesinaudojant pele.}

\xreq{r:ui_opt_mouse:2}
{Apkrovų vizualizacijos interfeisui turi būti galimybė pateikti laiko intervalą, kurio duomenis 
  norima vizualizuoti.}

\subsection{Užduočių formulavimo būdo (protokolo) reikalavimai}
\xreq{r:ui_composition}
{Kiekvienas iš pagrindinių sistemos vartotojo interfeisų turi būti pateikiamas vienoje arba
  keliose bazinio interfeiso kortelėse.}

\xreq{r:ui_base}{Bazinio interfeiso reikalavimai:}
  \yreq{r:ui_base:mainwnd}{Pagrindiniame langą turi sudaryti menu, įrankių juosta(-os), 
  kortelių komponentas ir būsenos juosta. }
  \yreq{r:ui_base:messages}{Bazinis interfeisas yra atsakingas už įvairių pranešimų pateikimą 
  naudotojui.}
  
\xreq{r:ui_tables}{Visose vartotojo interfeiso lentelėse turi būti galimybė keisti stulpelių 
plotį.}

\subsection{Interfeiso darnos ir standartizavimo reikalavimai}
\xreq{r:ui_os_look}{Vartotojo interfeisas turi atitikti tos operacinės sistemos sistemos
išvaizdą, kurioje dirba taikomoji varotojo aplikacija.}

\xreq{r:ui_unicode}{Grafiniame vartotojo interfeise pateikiamas tekstas ir pranešimai turi
būti UTF-8 arba UTF-16 koduotėje.}

\xreq{r:ui_cua}{Teksto įvedimo laukai turi palaikyti CUA įvedimo modelį.}

\xreq{r:ui_passwords}{Jokie konfidencialūs duomenys neturi būti pateikiami vartotojo
interfeise.}

\subsection{Pranešimų formulavimo reikalavimai}
\xreq{r:ui_messages:groups}{Pateikiami pranešimai turi būti suskirstyti į 4 grupes: 
informacinio pobūdžio, perspėjimo, klaidų ir kritiniai.}

\xreq{r:ui_messages:icons}{Kiekvienas iš 4 pranešimų tipų turi būti paženklintas 
atitinkama ikona.}

\xreq{r:ui_messages:error}{Klaidos pobūdžio pranešimuose turi būti pateikta detali klaidos
priežastis.}

\xreq{r:ui_messages:critical}{Kritinio pobūdžio pranešimuose turi būti galimybė peržiūrėti 
techninę klaidos priežasties informaciją, kurios pagalba sistemos administratorius galėtų 
identifikuoti ir pašalinti įvykusią problemą.}

\xreq{r:ui_messages:log}{Visi vartotojui pateikiami pranešimai kartu su jų pateikimo 
laiku ir kontekstu turi būti saugomi pranešimų žurnale.}

\xreq{r:ui_messages:id}{Kiekvienai galimai klaidai turi būti suteiktas unikalus numeris. Tokio
paties pobūdžio klaidos yra laikomos skirtingomis, jei jų atsiradimo priežastys yra skirtingos
arba klaidos pasirodo skirtingame kontekste. Pvz.: apibendrinta klaida „nepavyko atidaryti failo“
turi būti išskirstyta į „nepavyko atidaryti duomenų bazės failo“, „nepavyko atidaryti nustatymų
failo“.}

\subsection{Interfeiso individualizavimo reikalavimai}
\xreq{r:tool_dock}{Turi būti galimybė slėpti ir judinti skirtingus vartotojo interfeiso
elementus (pvz.: užduočių juostas, dock langus).}
