Ši persona yra išvestinė iš pirmos.
\subsection{Personos aprašas}
\xtableu
{
  a [ p | p ]
  w [ 1 | 3 ]
  %%%%%%%%%%%%%%%%%%%%%%%%%%%%%%%%%%%%%%%%%%%%%%%%%%%%%%%%%%%%%%%%
  hh [ Pagrindinė informacija ]
  %%%%%%%%%%%%%%%%%%%%%%%%%%%%%%%%%%%%%%%%%%%%%%%%%%%%%%%%%%%%%%%%
  e [ Trumpas asmenybės aprašas, veiklų siekiai 
  | 
    43 metų IT vadovas, Žadgailas. Siekia kuo pigiau užtikrinti
    organizacijos darbuotojų efektyvų darbą.
  ]
  e [ Veikla projekte 
  | 
    Prižiūri organizacijoje naudojamas IS, analizuoja jų apkrovimą.
    Ieško būdų, kaip naudojamas IS padaryti efektyvesnėmis.
  ]
  e [ Motyvacija tobulinti įgūdžius 
  |
    Norėtų išmokti dirbti nauja sistema, kuri leistų automatizuoti
    didžiąją dalį darbo. Pats savarankiškai mokytis pradėti
    norėtų ir galėtų, jei tam pakaktų laiko darbo metu.
  ]
  %%%%%%%%%%%%%%%%%%%%%%%%%%%%%%%%%%%%%%%%%%%%%%%%%%%%%%%%%%%%%%%%
  hh [ Projekto informacija ]
  %%%%%%%%%%%%%%%%%%%%%%%%%%%%%%%%%%%%%%%%%%%%%%%%%%%%%%%%%%%%%%%%
  e [ Projekto tikslai 
  | 
    Paramos priemonių paraiškų nagrinėjimas yra nuolat
    besitęsiantis procesas. Vos ne kas mėnesį atsiranda naujų
    paraiškos priemonių, kurias organizacijos padaliniai turi būti
    pasiruošę apdoroti. Tam dažnai reikia atnaujinti esamas, arba
    įdiegti naujas informacines sistemas. Žadgailo tikslas – rasti
    laiko tarpus, kada jo prižiūrimas IS galima būtų atnaujinti.
  ]
  e [ Esamos situacijos problemos 
  | 
    Kadangi sistemas reikia nuolat atnaujinti, tai Žadgailas,
    pasinaudodamas MS Office skaičiuoklės moduliu, bando pats
    prognozuoti, kada sistemos bus mažiausiai apkrautos, kad galėtų
    nuspręsti, kada jas galima būtų atnaujinti. Deja, prognozavimas
    tokiu būdu yra sudėtinga ir labai daug laiko atimanti veikla.
  ]
  e [ Būsimos sistemos vizija 
  | 
    Žadgailas norėtų turėti įrankį, leidžianti \textbf{matyti konkrečios
    IS apkrovos svyravimus}. Taip pat jis pageidautų, kad su juo
    galima būtų greičiau nei per pusvalandį rasti \textbf{laiko intervalą,
    kurio metu būtų geriausia vykdyti sistemos atnaujinimo darbus}.
  ]
}

\newpage
\subsection{Panaudojamumo tikslai}
\xtable
{
  w [ 1 | 1 | 1 | 1 ]
  a [ p | p | p | p ]
  h [ Užduotis | Tikslas | Kriterijus | Sėkmės matas ]
  hh [ Riboto panaudojimo etapas ]
  %
  e [ Peržiūrėti konkrečios IS apkrovos svyravimus | Efektyvumas | Sugaištas laikas | < 60 min. ]
  e [ Surasti mažiausios apkrovos laiko intervalą | Efektyvumas | Sugaištas laikas | < 60 min. ]
  %
  hh [ Pilno panaudojimo etapas ]
  %
  e [ Peržiūrėti konkrečios IS apkrovos svyravimus | Efektyvumas | Sugaištas laikas | < 30 min. ]
  e [ Surasti mažiausios apkrovos laiko intervalą | Efektyvumas | Sugaištas laikas | < 30 min. ]
}
