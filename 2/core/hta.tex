\subsection{Užduotis „apkrovų skaičiavimas“}

\textbf{Norint įvertinti padalinių/IS apkrovas, reikia:}
\begin{enumerate}
  \item Prisijungti prie sistemos.
  
  \item Įvesti pradinius duomenis.
  \begin{enumerate}
    \item Importuoti duomenis iš failo.
    \item Įvesti reikalingą informaciją klaviatūra.
  \end{enumerate}
  
  \item Įvesti istorinius duomenis.
  \begin{enumerate}
    \item Importuoti istorinius duomenis iš failo.
    \item Įvesti istorinius duomenis klaviatūra.
  \end{enumerate}
  
  \item Skaičiuoti apkrovas.
  \item Filtruoti rezultatus.
  \begin{enumerate}
    \item Pasirinkti, kieno apkrovas rodyti - padalinių ar IS.
    \item Pasirinkti dominančius padalinius/IS.
    \item Nurodyti aktualų laiko intervalą.
    \item Pasirinkti norimą rodymo rėžimą: ar duomenis rodyti grafiškai, ar lentelės pavidalu.
  \end{enumerate}
\end{enumerate}

\vspace{1cm}
\textbf{Planai:}
\begin{enumerate}
  \item Atlikti 1-5. Jei pradiniai duomenys jau įvesti, tai 2 galima praleisti.
  Jei istoriniai duomenys jau įvesti, tai 3 galima praleisti.
  Jei rezultatų filtravimas nedomina, tai 5 galima praleisti.
  
  \item Atlikti 2.1, jei norimi duomenys yra saugomi faile. Atlikti 2.2, jei reikia papildyti
  pradinius duomenis arba jei visi norimi duomenys yra įvedami klaviatūra.
  
  \item Atlikti 3.1, jei istoriniai duomenys yra saugomi faile. Atlikti 3.1, jei reikia papildyti
  istorinius duomenis arba jei visi istoriniai duomenys yra įvedami klaviatūra.
  
  \item Atlikti 5.1-5.4 bet kuria tvarka. Bet kurį punktą galima praleisti, jei tokios filtravimo
  rūšies nereikia ir tenkina esami rezultatai.
\end{enumerate}

\subsection{Užduotis „apkrovų prognozavimas“}

\textbf{Prielaidos:} prisijungta prie sistemos, pradiniai duomenys įvesti.

\textbf{Norint apskaičiuoti prognozuojamas padalinių/IS apkrovas, reikia:}
\begin{enumerate}
  \item Įvesti planuojamas paramos priemonių apimtis.
  \begin{enumerate}
    \item Importuoti planuojamas apimtis iš failo.
    \item Įvesti reikalingus duomenis klaviatūra.
  \end{enumerate}
  
  \item Prognozuoti apkrovas.
  \item Filtruoti rezultatus (žr. užd. „apkrovų skaičiavimas“).
\end{enumerate}

\vspace{1cm}
\textbf{Planai:}
\begin{enumerate}
  \item Atlikti 1-3. 3 galima praleisti.
  \item Atlikti 1.1, jei norima planuojamus kiekius importuoti iš failo. Taip pat importuotus
  kiekius po to galima papildyti su 1.2 arba juos įvesti naudojantis vien tik 1.2.
\end{enumerate}

\subsection{Užduotis „laisviausių laiko intervalų paieška“}

\textbf{Prielaidos:} prisijungta prie sistemos, pradiniai duomenys įvesti.

\textbf{Norint apskaičiuoti mažiausiai apkrautus laiko intervalus tam tikros IS atnaujinimui ar
padalinio darbuotojų kvalifikacijos kėlimui/patalpų remontui, reikia:}
\begin{enumerate}
  \item Prognozuoti IS/padalinių apkrovas.
  \item Pasirinkti dominančią IS/padalinį.
  \item Pasirinkti laiko intervalą, kuriame ieškoma mažiausia apkrova.
  \item Pasirinkti kokio ilgio intervalo ieškoma.
  \item Pasirinkti, ar ieškomas intervalo ilgis keičiasi dinamiškai nuo dabartinio sezono.
  \item Ieškoti mažiausių apkrovų.
\end{enumerate}

\vspace{1cm}
\textbf{Planai:}
\begin{enumerate}
  \item Atlikti 1 (žr. užd. „apkrovų prognozavimas“). Atlikti 2-4 bet kokia tvarka ir tada 6,
  jei ieškoma fiksuoto ilgio laiko intervalo. Atlikti 2-3, 5 bet kokia tvarka ir tada 6, jei
  ieškomo intervalo ilgis keičiasi dinamiškai.
\end{enumerate}
