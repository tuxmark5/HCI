\subsection{Personos aprašas}
\xtableu
{
  a [ p | p ]
  w [ 1 | 3 ]
  %%%%%%%%%%%%%%%%%%%%%%%%%%%%%%%%%%%%%%%%%%%%%%%%%%%%%%%%%%%%%%%%
  hh [ Pagrindinė informacija ]
  %%%%%%%%%%%%%%%%%%%%%%%%%%%%%%%%%%%%%%%%%%%%%%%%%%%%%%%%%%%%%%%%
  e [ Trumpas asmenybės aprašas, veiklų siekiai
  |
    38 metų IT administratorius, Antanas. Siekia užtikrinti, kad
    bendradarbiai galėtų be trikdžių dirbti su naudojamomis
    sistemomis.
  ]
  e [ Veikla projekte
  |
    Rūpinasi sistemos veikimu, jos palaikymu. Prižiūri ir užtikrina, jog visi naudotojai
    turėtų tinkamus leidimus naudotis jiems reikiamu funkcionalumu.
  ]
  e [ Naudojamos IT
  |
    Raštams ir ataskaitoms ruošti naudoja rašyklės modulį iš
    Microsoft Office paketo. Bendrauja daugiausiai telefonu arba
    naudodamasis pašto programa Mozilla Thunderbird. Taip pat kasdien
    naudojasi interneto naršykle Mozilla Firefox. Be šių programų
    jis dar nuolat dirba su savo prižiūrimų IT priemonių administravimo 
    įrankiais.
  ]
  e [ Naudotojo tipas
  |
    Ekspertas.
  ]
  e [ Motyvacija tobulinti įgūdžius
  |
    Mokymasis dirbti nauja sistema yra privalomas, nes tai įeina į darbo pareigas. Tačiau nenori
    leisti laisvalaikio analizuodamas naujas sudėtingas technologijas, reikalingas darbui su sistema.
  ]
  e [ Prieinama parama
  |
    Yra pats atsakingas už sistemos veikimą bei pagalbą kitiems naudotojams, todėl operatyvios pagalbos
    darbo vietoje jam suteikti dažniausiai niekas negali.
  ]
  %%%%%%%%%%%%%%%%%%%%%%%%%%%%%%%%%%%%%%%%%%%%%%%%%%%%%%%%%%%%%%%%
  hh [ Projekto informacija ]
  %%%%%%%%%%%%%%%%%%%%%%%%%%%%%%%%%%%%%%%%%%%%%%%%%%%%%%%%%%%%%%%%
  e [ Projekto tikslai
  |
    Antano tikslas yra užtikrinti, kad prie konkrečių duomenų prieitų
    tik tie žmonės, kuriems tai leidžia jų užimamos pareigos. Taip
    pat jis suinteresuotas kuo greičiau pašalinti iškilusias problemas,
    kad jo kolegos galėtų efektyviai dirbti.
  ]
  e [ Būsimos sistemos vizija
  |
    Antanas norėtų turėti galimybę lengvai \textbf{peržiūrėti ir
    keisti naudotojų leidimus} naujoje sistemoje. Taip pat norėtų
    patogaus ir išsamaus \textbf{sistemos įvykių žurnalo}, kuriame
    galėtų operatyviai \textbf{matyti ir filtruoti įvykius} bei taip
    nustatyti galimus sistemos veikimo netikslumus.
  ]
}

\subsection{Panaudojamumo tikslai}
\xtable
{
  w [ 1 | 1 | 1 | 1 ]
  a [ p | p | p | p ]
  h [ Užduotis | Tikslas | Kriterijus | Sėkmės matas ]
  hh [ Įdiegimas ]
  ee
  [
    Kadangi organizacijoje jau yra naudojama ne viena informacinė
    sistema, iš kurių bent kelios naudoja reliacines duomenų bazes,
    tai administratorius jau yra pažįstamas su jų įdiegimu ir
    tvarkymu, todėl jam neturėtų kilti problemų įdiegiant ir paleidžiant
    sistemos serveriui skirtą dalį, pagal naudotojui skirtame vadove
    pateiktas instrukcijas. Dėl sistemos kliento įdiegimo, žr.
    pirmos personos aprašymą.
  ]
  hh [ Riboto panaudojimo etapas ]
  %
  e [ Pridėti naują naudotoją | Efektyvumas | Sugaištas laikas | < 30 min. ]
  e [ Pakeisti naudotojui suteiktas teises | Efektyvumas | Sugaištas laikas | < 30 min. ]
  e [ Ištrinti naudotoją | Efektyvumas | Sugaištas laikas | < 10 min. ]
  e [ Peržiūrėti paskutinių 10 įvykių, kurie įvyko prieš konkretų laiko
  momentą, informaciją | Efektyvumas | Sugaištas laikas | < 60 min. ]
  %
  hh [ Pilno panaudojimo etapas ]
  %
  e [ Pridėti naują naudotoją | Efektyvumas | Sugaištas laikas | < 20 min. ]
  e [ Pakeisti naudotojui suteiktas teises | Efektyvumas | Sugaištas laikas | < 30 min. ]
  e [ Ištrinti naudotoją | Efektyvumas | Sugaištas laikas | < 2 min. ]
  e [ Peržiūrėti paskutinių 10 įvykių, kurie įvyko prieš konkretų laiko
  momentą, informaciją | Efektyvumas | Sugaištas laikas | < 30 min. ]
}
