\subsection{Personos aprašas}
\xtableu
{
  a [ p | p ]
  w [ 1 | 3 ]
  %%%%%%%%%%%%%%%%%%%%%%%%%%%%%%%%%%%%%%%%%%%%%%%%%%%%%%%%%%%%%%%%
  hh [ Pagrindinė informacija ]
  %%%%%%%%%%%%%%%%%%%%%%%%%%%%%%%%%%%%%%%%%%%%%%%%%%%%%%%%%%%%%%%%
  e [ Trumpas asmenybės aprašas, veiklų siekiai
  |
    38-metų Sistemos administratorius, Antanas. Siekia užtikrinti, kad bendradarbiai
    galėtų be trigdžių dirbti su naudojamomis sistemomis.
    FIXME: Siekis/veiklos tinkamos?
  ]
  e [ Veikla projekte
  |
    Rūpinasi sistemos veikimu, jos palaikymu. Prižiūri ir užtikrina, jog visi vartotojai turėtų tinkamus leidimus
    naudotis jiems reikiamu funkcionalumu.
  ]
  e [ Naudojamos IT
  |
    FIXME: IT priemones sistemos palaikymui nera priklausomos nuo pacios sistemos pobudzio?
    Raštams ir ataskaitoms ruošti naudoja rašyklės modulį iš Microsoft Office paketo.
    Bendrauja daugiausiai telefonu arba naudodamasis pašto
    programa Mozilla Thunderbird. Taip pat kasdien naudojasi interneto naršykle Mozilla Firefox.
  ]
  e [ Naudotojo tipas
  |
    Ekspertas.
  ]
  e [ Motyvacija tobulinti įgūdžius
  |
    Mokymasis dirbti nauja sistema yra privalomas, nes tai įeina į darbo pareigas. Tačiau nenori
    leisti laisvalaikio analizuodamas naujas sudėtingas technologijas, reikalingas darbui su sistema.
  ]
  e [ Prieinama parama
  |
    Yra pats atsakingas už sistemos veikimą bei pagalbą kitiems vartotojams, todėl operatyvios pagalbos
    darbo vietoje jam suteikti dažniausia niekas negali.
  ]
  %%%%%%%%%%%%%%%%%%%%%%%%%%%%%%%%%%%%%%%%%%%%%%%%%%%%%%%%%%%%%%%%
  hh [ Projekto informacija ]
  %%%%%%%%%%%%%%%%%%%%%%%%%%%%%%%%%%%%%%%%%%%%%%%%%%%%%%%%%%%%%%%%
  e [ Projekto tikslai
  |
    TODO: suderinti su kitu personu aprasymais.
  ]
  e [ Esamos situacijos problemos
  |
    TODO: suderinti su kitu personu aprasymais:
  ]
  e [ Būsimos sistemos vizija
  |
    FIXME: Uzdekit bolds, nes nepamenu kaip ir neturiu laiko ieskoti, aciu.:)
    Antanas norėtų turėti galimybę lengvai [b]peržiūrėti ir keisti vartotojų leidimus[/b] naujoje sistemoje.
    Taip pat noretų patogaus ir išsamaus [b]sistemos įvykių katalogo[/b], kuriame galėtų operatyviai [b]matyti ir filtruoti
    įvykius[/b] bei taip nustatyti galimus sistemos veikimo netikslumus.
  ]
}

\subsection{Panaudojamumo tikslai}
\xtable
{
  w [ 1 | 1 | 1 | 1 ]
  a [ p | p | p | p ]
  h [ Užduotis | Tikslas | Kriterijus | Sėkmės matas ]
  hh [ Riboto panaudojimo etapas ]
  %
  e [ X | Y | Z | W ]
  ee [ Paaiškinimas ]
  %
  e [ X | Y | Z | W ]
  ee [ Paaiškinimas ]
}