\subsection{Personos aprašas}
\xtableu
{
  a [ p | p ]
  w [ 1 | 3 ]
  %%%%%%%%%%%%%%%%%%%%%%%%%%%%%%%%%%%%%%%%%%%%%%%%%%%%%%%%%%%%%%%%
  hh [ Pagrindinė informacija ]
  %%%%%%%%%%%%%%%%%%%%%%%%%%%%%%%%%%%%%%%%%%%%%%%%%%%%%%%%%%%%%%%%
  e [ Trumpas asmenybės aprašas, veiklų siekiai 
  | 
    35 metų vadybininkas, Tadas. Siekia kuo efektyviau išnaudoti turimus resursus ir praleisti
    kiek įmanoma mažiau laiko skirstant išteklius.
  ]
  e [ Veikla projekte 
  | 
    Prižiūri jam paskirtų organizacijos padalinių veiklą.
  ]
  e [ Naudojamos IT 
  | 
    Raštams ir ataskaitoms ruošti naudoja rašyklės modulį iš Microsoft Office paketo. Skaitiniams
    duomenims apdoroti naudoja skaičiuoklės modulį. Bendrauja daugiausiai naudodamasis pašto
    programa Mozilla Thunderbird. Taip pat kasdien naudojasi interneto naršykle Mozilla Firefox.
  ]
  e [ Naudotojo tipas 
  | 
    Vidutiniškai patyręs. 
  ]
  e [ Motyvacija tobulinti įgūdžius 
  |
    Norėtų išmokti dirbti nauja sistema, kuri leistų efektyviau atlikti paskirtus darbus, tačiau
    pats savarankiškai mokytis pradėti negali, nes neturi tam pakankamai laiko, o
    laisvalaikio darbui aukoti nenori.
  ]
  e [ Prieinama parama 
  | 
    Dažniausiai stengiasi išspręsti problemas pats. Kai pačiam
    nepavyksta – konsultuojasi su vietiniu IT specialistu.
  ]
  %%%%%%%%%%%%%%%%%%%%%%%%%%%%%%%%%%%%%%%%%%%%%%%%%%%%%%%%%%%%%%%%
  hh [ Projekto informacija ]
  %%%%%%%%%%%%%%%%%%%%%%%%%%%%%%%%%%%%%%%%%%%%%%%%%%%%%%%%%%%%%%%%
  e [ Projekto tikslai 
  | 
    Paramos priemonių paraiškų nagrinėjimas yra nuolat besitęsiantis procesas. Vos ne
    kas mėnesį atsiranda naujų paraiškos priemonių, kurias organizacijos padaliniai turi būti
    pasiruošę apdoroti. Tado tikslas – taip išbalansuoti paraiškų nagrinėjimą, kad kiekvienas
    jam pavaldus padalinys būtų apkrautas optimaliai.
  ]
  e [ Esamos situacijos problemos 
  | 
    Kadangi paramos priemonių paraiškų nagrinėjimas yra nuolatinė
    veikla, tai Tadui tenka  balansuoti padalinių apkrovas taip pat
    nuolatos, tačiau centralizuotos duomenų bazės nebuvimas reiškia,
    kad informacijos apie padalinių apkrovas jam pačiam tenka
    nuolat prašyti tuose padaliniuose dirbančių darbuotojų.
    Kad būtų išvengta padalinių perkrovų, Tadas, naudodamasis
    Microsoft Office skaičiuoklės moduliu, pats bando prognozuoti
    ateities apkrovas, tačiau tokią analizę rankomis atlikti yra
    sudėtinga, todėl jo galimybės eksperimentuoti su apkrovų
    duomenimis yra stipriai apribotos.
  ]
  e [ Būsimos sistemos vizija
  | 
    Tadas norėtų turėti įrankį, leidžiantį \textbf{kaupti apdorotų paramos priemonių paraiškų
    istorinę informaciją} ir kuris leistų tą informaciją panaudoti \textbf{padalinių apkrovų 
    skaičiavimui ir prognozavimui}.
    Kadangi Tadui tenka skaičiuoti apkrovas bent kartą per savaitę, tai jei Tadui naujasis 
    įrankis leistų apskaičiuoti 4–10 padalinių apkrovas bent per 30 minučių, 
    tai laikas, praleistas mokantis naudotis nauja sistema, būtų laikomas naudinga investicija.
    Taip pat, Tadas norėtų, jog sistema gebėtų \textbf{identifikuoti padalinių
    mažiausios apkrovos intervalus}, per kuriuos, pavyzdžiui, būtų galima kelti darbuotojų
    kvalifikaciją. Kad toks funkcionalumas būtų naudingas praktiškai, tai 
    reikia, kad skaičiavimus su sistema galima būtų atlikti greičiau
    nei per 10 minučių.
  ]
}

\subsection{Panaudojamumo tikslai}
\xtable
{
  w [ 1 | 1 | 1 | 1 ]
  a [ p | p | p | p ]
  h [ Užduotis | Tikslas | Kriterijus | Sėkmės matas ]
  %
  hh [ Įdiegimas ]
  ee 
  [ 
    Sistemos kliento įrašymo procedūra bus kuo panašesnė į
    standartinę programos įrašymo pasirinktoje operacinėje sistemoje
    procedūrą. Iškilus klausimams, bus galima persiskaityti
    atitinkamą skyrių naudotojui skirtame vadove, kuriame bus
    pateiktas kiekvienas įrašymo žingsnis su paaiškinimais.
  ]
  %
  hh [ Apmokymas ]
  ee 
  [ 
    Kadangi sistemos grafinė sąsaja bus kuriama taip, kad ji būtų kuo
    panašesnė į naudotojui įprastą Microsoft Excel programos sąsają,
    tai specialių apmokymų, kaip naudotis sistema, nereikės. Tačiau,
    jei vis dėlto iškils klausimų, tai visuomet bus galima perskaityti
    pateiktą sistemos naudojimo aprašymą.
  ]
  %
  hh [ Riboto panaudojimo etapas ]
  e [ Suskaičiuoti 5 padalinių apkrovas | Efektyvumas | Sugaištas laikas | < 60 min. ]
  e [ Prognozuoti 5 padalinių apkrovas | Efektyvumas | Sugaištas laikas | < 60 min. ]
  %
  hh [ Pilno panaudojimo etapas ]
  e [ Suskaičiuoti 5 padalinių apkrovas | Efektyvumas | Sugaištas laikas | < 30 min. ]
  e [ Prognozuoti 5 padalinių apkrovas | Efektyvumas | Sugaištas laikas | < 30 min. ]
  e [ Mažiausios apkrovos intervalų paieška | Efektyvumas | Sugaištas laikas | < 10 min. ]
  ee 
  [ 
    Esminė sistemos kūrimo priežastis – skaičiavimo efektyvumo padidinimas: visus skaičiuojamus
    parametrus jau ir dabar galima suskaičiuoti pasinaudojant Microsoft Excel, tačiau toks
    darbo pobūdis šiuo metu užima pernelyg daug laiko.
  ]
}

