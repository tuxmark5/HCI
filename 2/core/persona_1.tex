\subsection{Personos aprašas}
\xtableu
{
  a [ p | p ]
  w [ 1 | 3 ]
  %%%%%%%%%%%%%%%%%%%%%%%%%%%%%%%%%%%%%%%%%%%%%%%%%%%%%%%%%%%%%%%%
  hh [ Pagrindinė informacija ]
  %%%%%%%%%%%%%%%%%%%%%%%%%%%%%%%%%%%%%%%%%%%%%%%%%%%%%%%%%%%%%%%%
  e [ Trumpas asmenybės aprašas, veiklų siekiai 
  | 
    35-metų vadybininkas, Tadas. Siekia kuo efektyviau išnaudoti turimus resursus ir praleisti
    kiek įmanoma mažiau laiko skirstant išteklius.
  ]
  e [ Veikla projekte 
  | 
    Prižiūri jam paskirtų organizacijos padalinių veiklą.
  ]
  e [ Naudojamos IT 
  | 
    Raštams ir ataskaitoms ruošti naudoja rašyklės modulį iš Microsoft Office paketo. Skaitiniams
    duomenims apdoroti naudoja skaičiuoklės modulį. Bendrauja daugiausiai naudodamasis pašto
    programa Mozilla Thunderbird. Taip pat kasdien naudojasi interneto naršykle Mozilla Firefox.
  ]
  e [ Naudotojo tipas 
  | 
    Vidutiniškai patyręs. 
  ]
  e [ Motyvacija tobulinti įgūdžius 
  |
    Norėtų išmokti dirbti nauja sistema, kuri leistų efektyviau atlikti paskirtus darbus, tačiau
    pats savarankiškai mokytis pradėti negali, nes neturi tam pakankamai laisvo laiko, o
    laisvalaikio darbui aukoti nenori.
  ]
  e [ Prieinama parama 
  | 
    Dažniausiai stengiasi išspręsti problemas pats, bet kai laiko nėra, tai konsultuojasi
    su vietiniu IT specialistu.
  ]
  %%%%%%%%%%%%%%%%%%%%%%%%%%%%%%%%%%%%%%%%%%%%%%%%%%%%%%%%%%%%%%%%
  hh [ Projekto informacija ]
  %%%%%%%%%%%%%%%%%%%%%%%%%%%%%%%%%%%%%%%%%%%%%%%%%%%%%%%%%%%%%%%%
  e [ Projekto tikslai 
  | 
    TODO 
  ]
  e [ Esamos situacijos problemos 
  | 
    TODO 
  ]
  e [ Būsimos sistemos vizija 
  | 
    TODO  
  ]
}

\subsection{Panaudojamumo tikslai}
\xtable
{
  w [ 1 | 1 | 1 | 1 ]
  a [ p | p | p | p ]
  h [ Užduotis | Tikslas | Kriterijus | Sėkmės matas ]
  hh [ Riboto panaudojimo etapas ]
  %
  e [ X | Y | Z | W ]
  ee [ Paaiškinimas ]
  %
  e [ X | Y | Z | W ]
  ee [ Paaiškinimas ]
}

TODO: (reikalavimai)
* Panaudojamumo tikslai pateikti esminėms užduotims
* Panaudojamumo tikslai pateikti aktualiems naudojimo gyvavimo ciklo
etapams (nurodytas matavimo kriterijus ir sėkmės matas)
