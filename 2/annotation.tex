\xchapter{ANOTACIJA}

\xsmallsec{Informacija apie vykdytojus ir jų įnašą į darbą}
\xtable
{
  w [ 2  | 7 ]
  a [ p  | p ]
  h [ Vykdytojas | Įnašas ]
  %
  e [ Vytautas Astrauskas 
  | @begin{xenum} 
      @item Skyrelis „Klaidinantis SQLite klaidos pranešimas“
      @item Skyrelis „Klaidinanti Opera Mini mygtuko antraštė“ 
      @item Skyrelis „GMail laiškų keitimas“
      @item Skyrelis „Muzikos (ne)importavimas į iTunes“
      @item Pradinė dokumento struktūra
    @end{xenum}
  ]
  %
  e [ Martynas Budriūnas
  | @begin{xenum} 
      @item Skyrelis „Simbolių šalinimas Samsung GT-C5130S mobiliojo telefono Java ME programose“
      @item Skyrelis „Žodžio įtraukimas į T9 žodyną Samsung GT-C5130S mobiliajame telefone“
      @item Skyrelis „Skambučių priėmimas Samsung GT-C5130S mobiliajame telefone“
      @item Skyrelis „VU studentų atsiliepimų apie Erasmus studijas peržiūra“
    @end{xenum}
  ]
  %
  e [ Justinas Jucevičius 
  | @begin{xenum} 
      @item Skyrelis „„Recovery Toolbox for Word“ programos išjungimas“
      @item Skyrelis „Prisijungimas prie Yahoo! Mail“
      @item Skyrelis „Windows operacinės sistemos atnaujinimas“
      @item Skyrelis „„Assassin's Creed“ žaidimo paleidimas“
    @end{xenum}
  ]
  %
  e [ Egidijus Lukauskas 
  | @begin{xenum} 
      @item Skyrelis „Tekstinės žinutės keitimas iOS ver.3.2.3“
      @item Skyrelis „Failų peržiųra Nokia Symbian v52.0.101 operacinėje sistemoje“
      @item Skyrelis „Google Chrome (Chromium) optimizuota viršutinė naršyklės dalis“
    @end{xenum}
  ]
  %
  e [ Audrius Šaikūnas 
  | @begin{xenum} 
      @item Skyrelis „Eclipse fono spalvos konfigūracija“
      @item Skyrelis „Klaidinantis GCC klaidos pranešimas“ 
      @item Skyrelis „VMware Workstation“
      @item Skyrelis „XMonad“
      @item Atnaujinta dokumento struktūra
    @end{xenum}
  ]
}

\xsmallsec{Bibliografinis darbo aprašas}
Šiuo dokumentu siekiama formaliai aprašyti ir išanalizuoti
pastebėtus esamų sąsajų nepatogumus, paaiškinti koks panaudojamumo
principas buvo pažeistas ir kodėl. Taip pat šiuo darbu yra siekiama
atkreipti dėmesį į itin pasisekusias sąsajų realizacijas ir
paanalizuoti kodėl būtent jos konkrečiu atveju yra patogesnės už
kitokias galimas sąsajų realizacijas.

\xsmallsec{Darbo vadovas}
Šis darbas yra parengtas kaip žmogaus ir kompiuterio sąveikos pirmasis laboratorinis darbas
– „Pastebėti esamų interfeisų (ne)patogumai“, vadovaujant dėstytojai Kristinai Lapin.
