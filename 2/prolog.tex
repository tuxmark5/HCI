\xsmallsec{Programų sistemos pavadinimas}
Kuriamos sistemos pavadinimas yra „\systemname“.

\xsmallsec{Dalykinė sritis}
Paramos priemonių administravimas.

\xsmallsec{Probleminė sritis}
Padalinių ir informacinių sistemų apkrovų analizė ir prognozavimas.

% FIXME Ar tinka?
% Citata iš Čaplinsko: „Poskyryje „Probleminė sritis“
% aprašomi uždaviniai (problema), kuriuos privalo spręsti kuriamoji
% programų sistema.“


\xsmallsec{Naudotojų kvalifikaciniai reikalavimai}
% TODO Reikia kur nors pateikti naudotojų aprašymus (pavyzdžiui,
% kas per vienas yra IT vadovas nėra iš karto aišku ir dėstytoja
% gali prisikabinti):
% padalinio vadovas – žmogus atsakingas už padalinį;
% IT vadovas – žmogus atsakingas už informacinę sistemą;
% sistemos administratorius – žmogus atsakingas už sistemos įdiegimą ir
% priežiūrą;
% asistentas – žmogus atsakingas už istorinių duomenų į sistemą įvedimą.
\xtable
{
  w [ 2 | 2 | 4 ]
  a [ p | p | p ]
  h [ Naudotojas | Kvalifikacija | Pastabos ]
  %
  e [ Padalinio vadovas | Mokyklinis informatikos kursas
  | Turi turėti darbo su skaičiuokle pagrindus.
  ]
  e [ IT vadovas | Mokyklinis informatikos kursas
  | Turi turėti darbo su skaičiuokle pagrindus.
  ]
  e [ Sistemos administratorius | Informatikos bakalauras
  | Turi turėti darbo su skaičiuokle, programų ir duomenų bazės diegimo
  bei administravimo pagrindus.
  ]
  e [ Asistentas | Mokyklinis informatikos kursas
  | Turi turėti darbo su skaičiuokle pagrindus.
  ]
}

\xsmallsec{Darbo vadovas}
Šis darbas yra parengtas kaip žmogaus ir kompiuterio sąveikos
antrasis laboratorinis darbas – „\docname“, vadovaujant
dėstytojai Kristinai Lapin.

\xsmallsec{Naudoti dokumentai}
\xdoclist
{
  \xdocentry{A. Čaplinskas}{Laboratorinių ir kursinių darbų reikalavimai}{Vilnius, 2009, 36 psl.}
  \xdocentry{K. Moroz-Lapin}{Žmogaus ir kompiuterio sąveika}{Vilnius, 2008, 248 psl.}
  \xdocentryWeb{K. Moroz-Lapin}{Naudotojų poreikiai. Antrasis laboratorinis darbas}{2011}
  {2011-10-02}{http://uosis.mif.vu.lt/\~moroz/priemone/2\%20Naudotoju\%20poreikiai.pdf}
}
