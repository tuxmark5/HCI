\subsection{Užduočių vykdymo rezultatai}

Teisingai atliko pirmosios užduoties visas dalis tik vienas dalyvis,
o antrosios užduoties, dėl sistemos klaidos ir nerealizuoto
funkcionalumo, neįveikė niekas.

\subsubsection{1 užduoties 1 dalis}

Dėl prisijungimo nei vienam iš testuotojų nekilo problemų, turbūt todėl,
kad joje tereikėjo paspausti mygtuką „Prisijungti“.

\subsubsection{1 užduoties 2 dalis}

Su duomenų importavimu taip pat nei vienam iš testuotojų nekilo problemų
– visi praktiškai iš karto pastebėjo mygtuką „Importuoti“ ir sėkmingai
surado pradinių duomenų failą.

\subsubsection{1 užduoties 3 dalis}

Per kiek laiko kiekvienas iš testuotojų atliko šią dalį:
\xtable
{
  w [ 1 | 1 | 1 | 1 | 1 ]
  a [ p | p | p | p | p ]
  h [ A | B | C | D | E ]
  e [ 3:43 | 2:20 | 8:01 | 3:22 | 5:23 ]
}

Mažiausias atsakymo gavimo laikas: 3 minutės 22 sekundės.

Didžiausias atsakymo gavimo laikas: 8 minutės ir 1 sekundė.

Gavo teisingą atsakymą: 3 testuotojai.

Visiškai sėkmingai užduotį įvykdė (gavo teisingą atsakymą per reikiamą 
laiką) dalyviai: B.

Problemos su kuriomis susidūrė testuotojai:
\begin{itemize}
  \item nei vienas iš testuotojų nepastebėjo, kad yra kortelė apžvalga,
    kur viename grafike galima pamatyti visų padalinių apkrovas;
  \item maišė padalinių pavadinimus su jų identifikatoriais;
  \item paleidžiant programą ji pasileidžia neišdidinto lango veiksenoje
    ir nesimato mygtuko „rodyti“ skirto rodomo intervalo keitimui.
\end{itemize}

\subsubsection{1 užduoties 4 dalis}

Per kiek laiko kiekvienas iš testuotojų atliko šią dalį:
\xtable
{
  w [ 1 | 1 | 1 | 1 | 1 ]
  a [ p | p | p | p | p ]
  h [ A | B | C | D | E ]
  e [ 4:43 | 5:23 | 2:05 | 8:48 | 1:18 ]
}

Mažiausias atsakymo gavimo laikas: 1 minutė 18 sekundžių.

Didžiausias atsakymo gavimo laikas: 8 minutės ir 48 sekundės.

Gavo teisingą atsakymą: 3 testuotojai.

Visiškai sėkmingai užduotį įvykdė (gavo teisingą atsakymą per reikiamą 
laiką) dalyviai: C, E.

Problemos su kuriomis susidūrė testuotojai:
\begin{itemize}
  \item kai kurie testuotojai nepastebėjo, kad galima perjungti vaizdą
    iš stulpelių diagramos į lentelę ir bandė spėti, kuris iš stulpelių
    yra aukštesnis;
  \item kadangi sąlygoje nebuvo nurodyta nuo kada prasideda duomenys,
    tai visiems darant paiešką užkliuvo 10 metų intervalo ilgio
    apribojimas;
  \item nurodžius lygiai 10 metų ilgio intervalą, sistema parodydavo
    klaidos pranešimą, kad intervalas turi būti ne ilgesnis nei 10 metų;
  \item ne visiems pavyko nustatyti viso intervalo, kuriame egzistuoja
    duomenys, galus;
  \item ne visi iki galo suprato ką būtent turi padaryti;
  \item dviem testuotojams kilo problemų dėl to, kad užduotyje nebuvo
    nurodyta kokio ilgio intervalo reikia ieškoti.
\end{itemize}

\subsubsection{1 užduoties 5 dalis}

Per kiek laiko kiekvienas iš testuotojų atliko šią dalį:
\xtable
{
  w [ 1 | 1 | 1 | 1 | 1 ]
  a [ p | p | p | p | p ]
  h [ A | B | C | D | E ]
  e [ 3:41 | 3:05 | 1:55 | 2:02 | 2:53 ]
}

Mažiausias atsakymo gavimo laikas: 1 minutė 55 sekundės.

Didžiausias atsakymo gavimo laikas: 3 minutės ir 41 sekundė.

Gavo teisingą atsakymą: 3 testuotojai.

Visiškai sėkmingai užduotį įvykdė (gavo teisingą atsakymą per reikiamą 
laiką) dalyviai: D, E.

Problemos su kuriomis susidūrė testuotojai:
\begin{itemize}
  \item maišė informacines sistemas su padaliniais, greičiausiai todėl,
    kad iki galo nesuprato dalykinės srities;
  \item nei vienas iš testuotojų nepastebėjo, kad yra kortelė apžvalga,
    kur viename grafike galima pamatyti visų IS apkrovas;
  \item maišė IS pavadinimus su jų identifikatoriais.
\end{itemize}

\subsubsection{1 užduoties 6 dalis}

Per kiek laiko kiekvienas iš testuotojų atliko šią dalį:
\xtable
{
  w [ 1 | 1 | 1 | 1 | 1 ]
  a [ p | p | p | p | p ]
  h [ A | B | C | D | E ]
  e [ 0:45 | 0:52 | 1:58 | 2:28 | 2:12 ]
}

Mažiausias atsakymo gavimo laikas: 45 sekundės.

Didžiausias atsakymo gavimo laikas: 2 minutės ir 28 sekundė.

Gavo teisingą atsakymą: visi 5 testuotojai.

Visiškai sėkmingai užduotį įvykdė (gavo teisingą atsakymą per reikiamą 
laiką) dalyviai: A, B, C, D, E.

Problemos su kuriomis susidūrė testuotojai:
\begin{itemize}
  \item vienas testuotojas pasigedo galimybės pažymėti (atžymėti) visus
    padalinius;
  \item nei vienas iš testuotojų nepastebėjo, kad yra kortelė apžvalga,
    kur viename grafike galima pamatyti visų IS apkrovas.
\end{itemize}

\subsubsection{1 užduoties 7 dalis}

Per kiek laiko kiekvienas iš testuotojų atliko šią dalį:
\xtable
{
  w [ 1 | 1 | 1 | 1 | 1 ]
  a [ p | p | p | p | p ]
  h [ A | B | C | D | E ]
  e [ 2:46 | 2:52 | 3:48 | 8:50 | 3:09 ]
}

Mažiausias atsakymo gavimo laikas: 2 minutės 46 sekundės.

Didžiausias atsakymo gavimo laikas: 8 minutės ir 50 sekundžių.

Gavo teisingą atsakymą: visi 3 testuotojai.

Visiškai sėkmingai užduotį įvykdė (gavo teisingą atsakymą per reikiamą 
laiką) dalyviai: A, B, E.

Problemos su kuriomis susidūrė testuotojai:
\begin{itemize}
  \item turėjo problemų susiedami informacinės sistemos identifikatorių
    su jos pavadinimu;
  \item ne visiems pavyko nustatyti viso intervalo, kuriame egzistuoja
    duomenys, galus;
  \item dviem testuotojams kilo problemų dėl to, kad užduotyje nebuvo
    nurodyta kokio ilgio intervalo reikia ieškoti.
\end{itemize}

\subsection{2 užduoties 1 dalis}

Kadangi vykdant šią dalį pačioje pabaigoje lūždavo maketas, laikas
nebuvo matuojamas. Pastebėtas problemos:
\begin{itemize}
  \item daliai testuotojų nesusisieja mygtukas su pavadinimu „Intervalų
    paieška“ su funkcija „laisviausių laiko intervalų paieška“;
  \item dauguma testuotojų bandė daryti mažiausios apkrovos intervalo
    paiešką analogiškai tam, kaip jie ieškojo laikotarpio kada buvo
    didžiausias apkrovimas – kadangi sistemoje prognozavimas dar nėra
    realizuotas, tai „rankiniu“ būdu surasti intervalo jiems nepavyko.
\end{itemize}

\subsection{2 užduoties 2 dalis}

Kadangi ši dalis yra analogiška 1 daliai, tai ji nebuvo vykdoma.

\subsection{Dalyvių komentarai}

Kaip vertinate sistemą?
\begin{itemize}
  \item Iš pirmo žvilgsnio atrodo paprasta (mažai mygtukų), bet
    kai pradedi naudotis, pasirodo daug įvairiausių funkcijų.
  \item Padaryta patogiai, tačiau trūksta tam tikrų funkcijų darbo
    palengvinimui.
\end{itemize}

Ar susidūrėte su sunkumais, netikėtumais? Jei taip, tai kokiais?
\begin{itemize}
  \item Dalis funkcijų neveikia.
  \item Rodomo laikotarpio apribojimas iki 10 metų.
  \item Neišdidinus programos lango nesimato kai kurių svarbių mygtukų.
  \item Neradau kaip padaryti, kad rodytų palyginimą, ne taip, kad visus
    sudeda ir žiūrėk pats, o kad po apačia būtų išspausdinta daugiausiai
    dirbo tas ir tas padalinys.
\end{itemize}

Kas jums labiausiai patiko?
\begin{itemize}
  \item Mažai mygtukų.
  \item Įvairūs informacijos atvaizdavimo būdai.
  \item Lengva rasti istorinę informaciją, pagrindinius duomenis.
\end{itemize}

Kas jums labiausiai nepatiko?
\begin{itemize}
  \item Paslėpti mygtukai.
  \item Grafikų masteliai skirtingi. \emph{Pastaba: Testuotojas
    nerado galimybės peržiūrėti informaciją viename grafike.}
\end{itemize}

Kokių sistemos funkcijų pasigedote?
\begin{itemize}
  \item Pasirinkus grafiko stulpelį nerodo duomenis atitinkančios datos.
  \item Naudotojo dokumentacijos.
  \item Nėra mygtuko „pažymėti visus“, „atžymėti visus“ renkantis
    padalinius.
\end{itemize}
