Programų sistemos RSys interfeiso testavimą vykdė du komandos nariai -
Vytautas Astrauskas ir Martynas Budriūnas.

Prieš testavimą buvo patobulinta sistema iki tokio lygmens, kad būtų galima
įvykdyti vieną esminę užduotį - „apkrovų skaičiavimą“. Taip pat sistema
buvo specialiai papildyta iš anksto paruoštais duomenimis taip, kad išeitų testuoti
ir kitą esminę užduotį - „laisviausių laiko intervalų paiešką“. Kadangi
užduoties „apkrovų prognozavimas“ vykdymo eiga yra identiška užduočiai 
„apkrovų skaičiavimas“, tai ši užduotis su dalyviais nebuvo testuojama.

Testavimo dalyviams buvo pateiktas sistemos prototipas ir dvi minėtos užduotys,
kuriuos buvo išskaidytos į smulkesnius žingsnius, tokius kaip:
\begin{itemize}
  \item importuoti duomenis iš pateikto failo;
  \item rasti, kada labiausiai buvo apkrautas „padalinys 7“;
\end{itemize}

Testavimo dalyviams šios užduotys buvo pateikiamos užduoties lape, kurį galima
rasti priede nr. 4.

TODO: teigiami aspektai

Testavimo rezultatai labai nedžiugina, nes visi testavimo dalyviai rado
neigiamų sistemos aspektų. Labiausiai tikėtinas tokio rezultatyvumo paaiškinimas - 
ne iki galo suprasta dalykinė sritis bei vartotojo gido trūkumas.

TODO: rekomendacijos
