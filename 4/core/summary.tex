Programų sistemos RSys interfeiso testavimą vykdė du komandos nariai -
Vytautas Astrauskas ir Martynas Budriūnas.

Prieš testavimą buvo patobulinta sistema iki tokio lygmens, kad būtų galima
įvykdyti vieną esminę užduotį – „apkrovų skaičiavimą“.
Taip pat sistema buvo specialiai papildyta iš anksto paruoštais
duomenimis taip, kad išeitų testuoti ir kitą esminę užduotį –
„laisviausių laiko intervalų paiešką“. Kadangi užduoties
„apkrovų prognozavimas“ vykdymo eiga yra identiška užduočiai
„apkrovų skaičiavimas“, tai ši užduotis su dalyviais nebuvo
testuojama.

Testavimo dalyviams buvo pateiktas sistemos prototipas ir dvi minėtos
užduotys, kuriuos buvo išskaidytos į smulkesnius žingsnius, tokius
kaip:
\begin{itemize}
  \item importuoti duomenis iš pateikto failo;
  \item rasti, kada labiausiai buvo apkrautas „padalinys 7“.
\end{itemize}

Testavimo dalyviams šios užduotys buvo pateikiamos užduoties lape,
kurį galima rasti priede nr. 4.

Nors ir testavimo dalyviams sunkiai sekėsi iš pradžių susidoroti su
pateiktomis užduotimis, tačiau rezultatyvumas su kiekviena sekančia
užduotimi vis po truputį kilo. Tai leidžia manyti, jog sistemos
naudotojai pakankamai greitai įstengs sistemą įvaldyti. Tai ypač gerai
matosi palyginus pirmos užduoties 6 ir 3 dalių rezulatus.

Iš kitos pusės, testavimo rezultatai labai nedžiugina, nes visi
testavimo dalyviai rado neigiamų sistemos aspektų. Labiausiai
tikėtinas to paaiškinimas – ne iki galo suprasta dalykinė sritis
bei naudotojui skirtos dokumentacijos trūkumas.

Tačiau didelis problemų skaičius suteikia galimybę tobulinti sistemą. Kelios
rekomendacijos apie sistemos tobulinimo galimybes:
\begin{itemize}
  \item užvedus pelę ant diagramos stulpelio, sistema turėtų parodyti kokią
  	reikšmę minėtas stulpelis atitinka;
  \item automatiškai koreguoti stebimą laiko intervalą pagal turimus
    duomenis;
  \item kur įmanoma prie padalinių/sistemų indentifikatorių rodyti
    ir pilnus jų pavadinimus.
\end{itemize}

Plačiau apie rekomendacijas žr. skyrelyje
\nameref{section:recommendations}.
