\label{section:recommendations}

\xtable
{
  w [ 1 | 1 | 1 | 2 | 2 ]
  a [ p | p | p | p | p ]
  h [ Prioritetas | Sunkumo laipsnis | Dažnis | Problema | Rekomendacija ]
  e [ 6 | 2 | 4 |
    Ieškant didžiausios grafike nesimato, kuris iš stulpelių yra
    aukščiausias. |
    Padaryti, kad užvedus pelę ant stulpelio parodytų skaičių – reiškiantį
    jo dydį.
    ]
  e [ 6 | 3 | 3 |
    Mygtukas „Intervalų paieška“ nesusisieja su funkcija „laisviausių
    laiko intervalų paieška“. |
    Mygtuką pervadinti į „Laisviausių laiko intervalų paieška“.
    ]
  e [ 6 | 2 | 4 |
    Nesimato nuo kada „prasideda“ duomenys ir sistema meta klaidą 
    pasirinkus per didelį intervalą ($\geq 10$ metų). |
    Jei pradžios data buvo pasirinkta ankstesnė už duomenų „pradžią“,
    automatiškai ją nustatyti į duomenų „pradžios“ datą.
    ]
  e [ 6 | 2 | 4 |
    Pasirinkus stulpelį nerodo duomenis atitinkančios datos. |
    Padaryti, kad užvedus pelę ant stulpelio rodytų duomenis atitinkančią
    datą.
    ]
  e [ 6 | 2 | 4 |
    Nei vienas iš testuotojų nepastebėjo, kad yra kortelė apžvalga,
    kur viename grafike galima pamatyti visų padalinių apkrovas |
    Padaryti, kad pasirinkus rodyti intervalą išskaidytą į lygiai vieną
    dalį (pavyzdžiui, jei pasirenkame rodyti nuo 2010-02-01 iki
    2010-02-28 po mėnesį) sistema parodytų pranešimą su pasiūlymu
    pažiūrėti į „Apžvalgos“ kortelę.
    ]
  e [ 5 | 2 | 2 |
    Maišė padalinių (IS) pavadinimus ir identifikatorius. |
    Sąrašuose rodyti ne tik identifikatorius, bet ir pavadinimus. Taip
    pat, užvedus pelę ant padalinį ar IS reiškiančio elemento,
    rodyti paaiškinimus su identifikatoriumi ir pilnu pavadinimu.
    ]
  e [ 5 | 1 | 4 |
    Paleidžiant programą ji pasileidžia neišdidinto lango veiksenoje
    ir nesimato mygtuko „rodyti“ skirto rodomo intervalo keitimui. |
    Nustatyti, kad programa pasileistų išdidinto lango veiksenoje.
    ]
  e [ 3 | 1 | 2 |
    Trūksta galimybės, renkantis kuriuos padalinius (IS) rodyti,
    pažymėti (atžymėti) visus. |
    Pridėti mygtukus leidžiančius tai atlikti.
    ]
  e [ 3 | 1 | 2 |
    Trūksta galimybės, renkantis kokio intervalo duomenis rodyti,
    datos laukelyje nuspaudus klavišą „įvesti“ įvykdyti „Rodyti“. |
    Realizuoti šią galimybę.
    ]
}
