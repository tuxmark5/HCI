%Bendri panaudojamumo vertinimo klausimynai:
%\begin{itemize}
  %\item SUMI (Software Usability Measurement Inventory)
  %\item QUIS (Questionnaire for User Interaction Satisfaction)
%\end{itemize}
%Parengti savybių sąrašą (konspekto 96 psl.)


%Vertinimo kriterijai testuojant:
%\begin{itemize}
  %\item kiek kartų teko grįžti į pagrindinį meniu be reikalo;
  %\item kiek kartų buvo užuominų ar paraginimų;
  %\item kiek kartų buvo atverstas puslapis nurodantis puslapio žemėlapį
    %(struktūrą);
  %\item abejojimų skaičius (ir jų trukmė);
  %\item užduotys, kurios neatitinka sėkmės kriterijaus;
  %\item klaidų aprašas ir sunkumų nustatymas;
  %\item klaidų priežasčių nustatymas;
%\end{itemize}

\subsection{Testuojamos užduotys}

Testavimo metu buvo tikrinama, kaip sistemos naudotojai atlieka užduotis
„Apkrovų skaičiavimas“ (toliau bus ši užduotis vadinama užduotimi UA) ir 
„Laisviausių laiko intervalų paieška“ (užduotis UB).
Konkrečias užduočių formuluotes, kurios buvo pateiktos testuotojams
galima pamatyti TODO priede. Laikoma, kad testuojantysis visiškai sėkmingai
įvykdė užduotį, jei:
\begin{itemize}
  \item jis įvykdė visus nurodytus žingsnius (tvarka nesvarbi);
  \item pasinaudodamas sistema gavo teisingus rezultatus;
  \item nė karto nesikreipė pagalbos į asistentą.
\end{itemize}
TODO: DĖSTYTOJA: Ar teisingai suprasta, kas yra kriterijai ir kaip jie
turėtų būti aprašyti?

\xtableu
{
  a [ p | p | p ]
  w [ 3 | 2 | 1 ]
  h [ Použduotis | Kriterijus | Sėkmės matas ]
  hh [ Užduotis: „apkrovų skaičiavimas“ ]
  %
  er0 [ 3 | Prisijungti prie sistemos 
    | Atlikimo laikas | < 45 s. ]
  er1 [ 
    | Pagalbos kreipinių skaičius | = 0 ]
  er2 [ 
    | Sustojimų skaičius | = 0 ]
  %
  er0 [ 3 | Duomenų importavimas 
    | Atlikimo laikas | < 1 min. ]
  er1 [
    | Pagalbos kreipinių skaičius | = 0 ]
  er2 [ 
    | Sustojimų skaičius | = 0 ]
  %
  er0 [ 3 | Labiausiai apkrauto 2009 metų gruodžio mėnesio padalinio radimas 
    | Atlikimo laikas | < 2 min. ]
  er1 [
    | Pagalbos kreipinių skaičius | <= 1 ]
  er2 [ 
    | Sustojimų skaičius | < 3 ]
  %
  er0 [ 3 | Intervalo, kada buvo labiausiai apkrautas padalinys 7 radimas 
    | Atlikimo laikas | < 2 min. ]
  er1 [
    | Pagalbos kreipinių skaičius | <= 1 ]
  er2 [ 
    | Sustojimų skaičius | < 3 ]
  %
  er0 [ 3 | Labiausiai apkrautos 2011 metų rugsėjo menesį sistemos radimas 
         | Atlikimo laikas | < 2 min. ]
  er1 [ | Pagalbos kreipinių skaičius | <= 1 ]
  er2 [ | Sustojimų skaičius | < 3 ]
  %
  er0 [ 3 | Labiausiai apkrauto 2010 metų vasario mėnesį padalinio radimas
      (iš 1, 4, 5, 10, 11 ir 12 padalinių) 
    | Atlikimo laikas | < 5 min. ]
  er1 [
    | Pagalbos kreipinių skaičius | <= 1 ]
  er2 [ 
    | Sustojimų skaičius | < 3 ]
  %
  er0 [ 3 | Radimas, kada buvo labiausiai apkrauta Patikros užduočių vykdymo IS 
    | Atlikimo laikas | < 5 min. ]
  er1 [
    | Pagalbos kreipinių skaičius | <= 1 ]
  er2 [ 
    | Sustojimų skaičius | < 3 ]
  %
  hh [ Užduotis: „laisviausių laiko intervalų paieška“ ]
  %
  er0 [ 3 | Mažiausios apkrovos 15 dienų intervalo radimas padaliniui 
      4 laikotarpyje 2012-01-01 – 2012-06-01 
    | Atlikimo laikas | < 4 min. ]
  er1 [
    | Pagalbos kreipinių skaičius | <= 1 ]
  er2 [ 
    | Sustojimų skaičius | < 3 ]
  %
  er0 [ 3 | Mažiausios apkrovos 15 dienų intervalo radimas padaliniui 
      9 laikotarpyje 2012-01-01 – 2012-06-01 
    | Atlikimo laikas | < 4 min. ]
  er1 [
    | Pagalbos kreipinių skaičius | <= 1 ]
  er2 [ 
    | Sustojimų skaičius | < 3 ]
}

\subsection{Metodas}

TODO: DĖSTYTOJA: Ar teisingai suprasta, kas čia turi būti aprašyta.

Testavimo vykdymo eiga:
\begin{enumerate}
  \item testuojantysis pasirašo sutikimą;
  \item testuojantysis užpildo klausimyną apie jo patirtį;
  \item testuojančiajam duodama užduotis ir jis ją bando atlikti
    pasinaudodamas programa, garsiai komentuodamas ką bando daryti;
    vykdytojas tuo metu sėdi šalia, žymisi pastabas bei per kiek
    laiko testuojantysis įvykdė užduoties dalį (galimas ir nuotolinis
    būdas, kai testuotojas pats žymi pastabas ir laikus);
  \item testuojantysis užpildo klausimyną apie naudojimosi programa
    įspūdžius.
\end{enumerate}

% Mano gauti atsakymai:
% \begin{verbatim}
% 3: „Padalinys 1“
% 4: 2008-05 (10245,5), 2010-03 (10152,5), 2010-05 (10137), 2011-06 (10290) – norimas atsakymas: 2011-06-01 – 2011-07-01;
% 5: IS1, Finansų apskaitos ir valdymo;
% 6: „Padalinys 12“;
% 7: 2009-04 (35295), 2010-03 (35913,5), 2011-05 (34999) – norimas atsakymas: 2010-03-01 – 2010-04-01;
% 1: 2012-02-13 – 2012-02-28;
% 2: 2012-02-18 – 2012-03-04;
% \end{verbatim}

\subsection{Aplinka}

TODO: DĖSTYTOJA: Ar teisingai suprasta, kas čia turi būti aprašyta.

Keturi žmonės sistemą testavo kavinėje: vienas sėdi su vykdytoju prie
kompiuterio, kiti tuo tarpu sėdi netoliese ir šnekasi apie pašalinius
dalykus.

Vienas žmogus testavo sistemą namie, visą reikiamą įrangą įsirašęs į
savo kompiuterį.

\subsection{Dalyviai}

TODO: DĖSTYTOJA: Ar teisingai suprasta, kas čia turi būti aprašyta.

Visi dalyviai yra skirtingų profesijų atstovai, atitinkantys
„Naudotojų kvalifikaciniai reikalavimai“ skyrelyje aprašytus reikalavimus
vadovams ir asistentui. 4 iš 5 dalyvių su dalykine sritimi ir kuriamos
sistemos paskirtimi buvo trumpai supažindinti prieš testavimą, o
penktasis tai žinojo jau iš anksčiau.
