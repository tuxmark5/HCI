%Bendri panaudojamumo vertinimo klausimynai:
%\begin{itemize}
  %\item SUMI (Software Usability Measurement Inventory)
  %\item QUIS (Questionnaire for User Interaction Satisfaction)
%\end{itemize}
%Parengti savybių sąrašą (konspekto 96 psl.)


%Vertinimo kriterijai testuojant:
%\begin{itemize}
  %\item kiek kartų teko grįžti į pagrindinį meniu be reikalo;
  %\item kiek kartų buvo užuominų ar paraginimų;
  %\item kiek kartų buvo atverstas puslapis nurodantis puslapio žemėlapį
    %(struktūrą);
  %\item abejojimų skaičius (ir jų trukmė);
  %\item užduotys, kurios neatitinka sėkmės kriterijaus;
  %\item klaidų aprašas ir sunkumų nustatymas;
  %\item klaidų priežasčių nustatymas;
%\end{itemize}

\subsection{Testuojamos užduotys}

Testavimo metu buvo tikrinama, kaip sistemos naudotojai atlieka užduotis
„Apkrovų skaičiavimas“ ir „Laisviausių laiko intervalų paieška“.
Konkrečias užduočių formuluotes, kurios buvo pateiktos testuotojams
galima pamatyti TODO priede. Laikoma, kad testuojantysis visiškai sėkmingai
įvykdė užduotį, jei:
\begin{itemize}
  \item jis įvykdė visus nurodytus žingsnius (tvarka nesvarbi);
  \item pasinaudodamas sistema gavo teisingus rezultatus;
  \item nė karto nesikreipė pagalbos į asistentą.
\end{itemize}
TODO: DĖSTYTOJA: Ar teisingai suprasta, kas yra kriterijai ir kaip jie
turėtų būti aprašyti?

\subsection{Metodas}

Testavimo eiga:
\begin{enumerate}
  \item \verb|git pull|, sukompiliuojame naujausią versiją arba:
    \begin{verbatim}
    wget http://astrauskas.lt/RSys
    wget http://astrauskas.lt/libkdchart.so.2
    chmod 755 RSys
    export LD_LIBRARY_PATH=.
    ./RSys
    \end{verbatim}
  \item įsirašome \verb|ffmpeg| ir \verb|motion|;
  \item pritaikome sau scenarijus esančius ŽKS saugykloje
    \verb|4/tools| kataloge (ypač atkreipkit dėmesį į \verb|motion.conf|
    nustatymą \verb|target_dir|);
  \item atsispausdiname sutikimą, pirmą ir antrą klausimynus ir užduoties
    aprašymą;
  \item duodame pasirašyti sutikimą (\verb|sutikimas.odt|);
  \item duodame užpildyti klausimyną (\verb|klausimynas.odt|);
  \item paleidžiame \verb|record-camera.sh|;
  \item paleidžiame \verb|record-screen.sh|;
  \item paleidžiame RSys;
  \item pasodiname testuotoją ir duodame jam užduotį;
  \item vykdymo metu užsirašinėjam ko klausinėja, kaip reaguoja į programą
    ir t.t. (prisiminkit, kad garsas nėra įrašomas);
  \item kai baigia, išjungiame \verb|record-camera.sh| ir
    \verb|record-screen.sh|;
  \item duodame antrąjį klausimyną (\verb|klausimynas (po testavimo).odt|);
  \item kai baigia jį pildyti, išjungiame RSys.
\end{enumerate}

Mano gauti atsakymai:
\begin{verbatim}
3: „Padalinys 1“
4: 2008-05 (10245,5), 2010-03 (10152,5), 2010-05 (10137), 2011-06 (10290) – norimas atsakymas: 2011-06-01 – 2011-07-01;
5: IS1, Finansų apskaitos ir valdymo;
6: „Padalinys 12“;
7: 2009-04 (35295), 2010-03 (35913,5), 2011-05 (34999) – norimas atsakymas: 2010-03-01 – 2010-04-01;
1: 2012-02-13 – 2012-02-28;
2: 2012-02-18 – 2012-03-04;

Aš pats testą atlikau per ~23 minutes.

\end{verbatim}

\subsection{Aplinka}

\subsection{Dalyviai}
