\xsmallsec{Programų sistemos pavadinimas}
Kuriamos sistemos pavadinimas yra „\systemname“.

\xsmallsec{Dalykinė sritis}
Paramos priemonių administravimas.

\xsmallsec{Probleminė sritis}
\begin{enumerate}
  \item Padalinių ir informacinių sistemų apkrovų analizė ir prognozavimas.
  \item Padalinių/IS apkrovų vizualizacija.
  \item Mažiausių padalinių/IS apkrovos intervalų paieška.
\end{enumerate}

\xsmallsec{Naudotojų kvalifikaciniai reikalavimai}
\xtable
{
  w [ 2 | 2 | 4 ]
  a [ p | p | p ]
  h [ Naudotojas | Kvalifikacija | Pastabos ]
  %
  e [ Padalinio vadovas | Mokyklinis informatikos kursas
  | Turi turėti darbo su skaičiuokle pagrindus.
  ]
  e [ IT vadovas | Mokyklinis informatikos kursas
  | Turi turėti darbo su skaičiuokle pagrindus.
  ]
  e [ Sistemos administratorius | Informatikos bakalauras
  | Turi turėti darbo su skaičiuokle, programų ir duomenų bazės diegimo
  bei administravimo pagrindus.
  ]
  e [ Asistentas | Mokyklinis informatikos kursas
  | Turi turėti darbo su skaičiuokle pagrindus.
  ]
}

\xsmallsec{Darbo vadovas}
Šis darbas yra parengtas kaip žmogaus ir kompiuterio sąveikos
trečiasis laboratorinis darbas – „\docname“, vadovaujant
dėstytojai Kristinai Lapin.

\xsmallsec{Naudoti dokumentai}
\xdoclist
{
  \xdocentry{K. Moroz-Lapin}{Žmogaus ir kompiuterio sąveika}{Vilnius, 2008, 248 psl.}
  \xdocentryWeb{K. Moroz-Lapin}{Eskizinis maketas ir jo vertinimas. Trečiasis laboratorinis darbas}{2011}
  {2011-11-04}{http://uosis.mif.vu.lt/\~moroz/priemone/3\%20Interfeiso\%20eskizas\%20ir\%20jo\%20vertinimas.pdf}
}

\xsmallsec{Naudoti įrankiai}
\xtableu
{
  w [ 1 | 3 ]
  a [ p | p ]
  h [ Pavadinimas | Aprašymas, Nuroroda ]
  %
  e [ XeTeX | Dokumentų procesorius. \newline \url{http://scripts.sil.org/xetex} ]
  e [ Qt Creator | Integruota kūrimo aplinka. \newline \url{http://qt.nokia.com/products/developer-tools/} ]
  e [ Git | Versijų kontrolės sistema. \newline \url{http://git-scm.com/} ]
}
