\xchapter{ANOTACIJA}

\xsmallsec{Bibliografinis darbo aprašas}
Šiuo darbu siekiama sukurti eskizinį prototipą, kuriame būtų matoma esminių
užduočių pagrindinių langų struktūra, informacijos pateikimo būdai ir galimi
šių užduočių atlikimo scenarijai. Taip pat vertinant šį eskizą KLM ir pažintinės
peržvalgos metodais siekiama atrasti galimus eskizo trūkūmus dar nepradėjus
implementuoti interfeiso taip užtikrinant, kad implementuotas interfeisas
turės kiek įmanoma mažesnį defektų skaičių ir gerai realizuotus panaudojamumo
principus.

\xsmallsec{Informacija apie vykdytojus ir jų įnašą į darbą}
\xtable
{
  w [ 3  | 7 ]
  a [ p  | p ]
  h [ Vykdytojas | Įnašas ]
  %
  e [ Vytautas Astrauskas
  | @begin{xenum} 
      @item Dokumento peržiūra
      @item Skyrelis „Esminių užduočių pažintinė peržvalga“
    @end{xenum}
  ]
  %
  e [ Martynas Budriūnas
  | @begin{xenum} 
      @item Dokumento peržiūra
    @end{xenum}
  ]
  %
  e [ Justinas Jucevičius 
  | @begin{xenum} 
      @item TODO
    @end{xenum}
  ]
  %
  e [ Egidijus Lukauskas 
  | @begin{xenum} 
      @item Visi interfeisų maketai.
    @end{xenum}
  ]
  %
  e [ Audrius Šaikūnas 
  | @begin{xenum} 
      @item Skyrelis „Prototipas“ (tekstinė dalis)
      @item Skyrelis „Esminių užduočių vertinimas, remiantis KLM metodu“
    @end{xenum}
  ]
}

\xsmallsec{Kontaktai}
\xtableu
{
  w [ 1 | 1 ]
  a [ p | p ]
  h [ Vykdytojas | El. paštas ]
  %
  e [ Vytautas Astrauskas | \url{Vytautas.Astrauskas&Amif.stud.vu.lt} ]
  e [ Martynas Budriūnas  | \url{Martynas.Budriunas&Amif.stud.vu.lt} ]
  e [ Justinas Jucevičius | \url{Justinas.Jucevicius&Amif.stud.vu.lt} ]
  e [ Egidijus Lukauskas  | \url{Egidijus.Lukauskas&Amif.stud.vu.lt} ]
  e [ Audrius Šaikūnas    | \url{tuxmarkv&Agmail.com} ]
}
