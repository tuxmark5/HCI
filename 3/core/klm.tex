\subsection{Užduotis „Apkrovų skaičiavimas“}
\textbf{Sąlygos:} prisijungta prie sistemos, duomenys importuoti.
\klm
{
  klm1 [ Pasirinkti rėžimą (IS ar padalinių) | HMP | pelė ]  
  klm1 [ Pažymėti dominančius (~5) padalinius/IS | MPMPMPMPMP | pelė ]  
  klm1 [ Paspausti ant intervalo pradžios įvedimo lauko | MP | pelė ]
  klm1 [ Įvesti pradžios datą (metus ir mėnesį) | HM6K | pelė ]
  klm1 [ Paspausti ant intervalo pabaigos įvedimo lauko | MHP | pelė ]
  klm1 [ Įvesti pabaigos datą (metus ir mėnesį) | HM6K | pelė ]
  klm1 [ Paspausti „Rodyti“ | HMP | pelė ]
}
{
  klm1 [ Pasirinkti rėžimą (IS ar padalinių) | MK | pelė ]  
  klm1 [ Pažymėti dominančius (~5) padalinius/IS | HMPMPMPMPMP | pelė ]  
  klm1 [ Paspausti ant intervalo pradžios įvedimo lauko | MP | pelė ]
  klm1 [ Įvesti pradžios datą (metus ir mėnesį) | HM6K | pelė ]
  klm1 [ Paspausti TAB | MK | pelė ]
  klm1 [ Įvesti pabaigos datą (metus ir mėnesį) | HM6K | pelė ]
  klm1 [ Paspausti ENTER | MK | pelė ]
}

\subsection{Užduotis „Apkrovų prognozavimas“}
\textbf{Sąlygos:} prisijungta prie sistemos, duomenys importuoti.
\klm
{
  klm1 [ Pasirinkti kortelę „Planuojami kiekiai“ | HMP | pelė ]  
  %
  klm1 [ Aktyvuoti priemonės langelį | MP | pelė ]  
  klm1 [ Įvesti 1-ąją priemonę | HM4K | pelė ]
  klm1 [ Aktyvuoti kiekio langelį | HMP | pelė ] 
  klm1 [ Įvesti 1-ąjį kiekį | HM2K | pelė ] 
  klm1 [ Aktyvuoti intervalo prad. langelį | HMP | pelė ] 
  klm1 [ Įvesti 1-ojo intervalo pradžią | HM8K | pelė ] 
  klm1 [ Aktyvuoti intervalo pab. langelį | HMP | pelė ] 
  klm1 [ Įvesti 1-ojo intervalo pabaigą | HM8K | pelė ] 
  %
  klm1 [ Aktyvuoti priemonės langelį | HMP | pelė ]  
  klm1 [ Įvesti 2-ąją priemonę | HM4K | pelė ]
  klm1 [ Aktyvuoti kiekio langelį | HMP | pelė ] 
  klm1 [ Įvesti 2-ąjį kiekį | HM2K | pelė ] 
  klm1 [ Aktyvuoti intervalo prad. langelį | HMP | pelė ] 
  klm1 [ Įvesti 2-ojo intervalo pradžią | HM8K | pelė ] 
  klm1 [ Aktyvuoti intervalo pab. langelį | HMP | pelė ] 
  klm1 [ Įvesti 2-ojo intervalo pabaigą | HM8K | pelė ] 
  %
  klm1 [ Aktyvuoti priemonės langelį | HMP | pelė ]  
  klm1 [ Įvesti 3-ąją priemonę | HM4K | pelė ]
  klm1 [ Aktyvuoti kiekio langelį | HMP | pelė ] 
  klm1 [ Įvesti 3-ąjį kiekį | HM2K | pelė ] 
  klm1 [ Aktyvuoti intervalo prad. langelį | HMP | pelė ] 
  klm1 [ Įvesti 3-ojo intervalo pradžią | HM8K | pelė ] 
  klm1 [ Aktyvuoti intervalo pab. langelį | HMP | pelė ] 
  klm1 [ Įvesti 3-ojo intervalo pabaigą | HM8K | pelė ] 
  %
  klm1 [ Aktyvuoti paramos administravimo lauką | HMP | x ]
  klm1 [ Įvesti administravimo sąnaudas | HM2K | x ]
  klm1 [ Aktyvuoti paramos administravimo lauką | HMP | x ]
  klm1 [ Įvesti administravimo sąnaudas | HM2K | x ]
  klm1 [ Aktyvuoti paramos administravimo lauką | HMP | x ]
  klm1 [ Įvesti administravimo sąnaudas | HM2K | x ]
}
{
  klm1 [ Aktyvuoti kortelę „Planuojami kiekiai“ | M2K | pelė ]  
  %
  klm1 [ Aktyvuoti priemonės langelį | HMP | pelė ]  
  klm1 [ Įvesti 1-ąją priemonę | HM4K | pelė ]
  klm1 [ Spausti TAB | MK | pelė ] 
  klm1 [ Įvesti 1-ąjį kiekį | M2K | pelė ] 
  klm1 [ Spausti TAB | MK | pelė ] 
  klm1 [ Įvesti 1-ojo intervalo pradžią | M8K | pelė ] 
  klm1 [ Spausti TAB | MK | pelė ] 
  klm1 [ Įvesti 1-ojo intervalo pabaigą | M8K | pelė ] 
  %
  klm1 [ Spausti ENTER | MK | pelė ]  
  klm1 [ Įvesti 2-ąją priemonę | M4K | pelė ]
  klm1 [ Spausti TAB | MK | pelė ] 
  klm1 [ Įvesti 2-ąjį kiekį | M2K | pelė ] 
  klm1 [ Spausti TAB | MK | pelė ] 
  klm1 [ Įvesti 2-ojo intervalo pradžią | M8K | pelė ] 
  klm1 [ Spausti TAB | MK | pelė ] 
  klm1 [ Įvesti 2-ojo intervalo pabaigą | M8K | pelė ] 
  %
  klm1 [ Spausti ENTER | MK | pelė ]  
  klm1 [ Įvesti 2-ąją priemonę | M4K | pelė ]
  klm1 [ Spausti TAB | MK | pelė ] 
  klm1 [ Įvesti 2-ąjį kiekį | M2K | pelė ] 
  klm1 [ Spausti TAB | MK | pelė ] 
  klm1 [ Įvesti 2-ojo intervalo pradžią | M8K | pelė ] 
  klm1 [ Spausti TAB | MK | pelė ] 
  klm1 [ Įvesti 2-ojo intervalo pabaigą | M8K | pelė ] 
}

\subsection{Užduotis „Laisviausių laiko intervalų paieška“}
\textbf{Sąlygos:} prisijungta prie sistemos, duomenys importuoti.
\klm
{
  klm1 [ Paspausti ant intervalo pradžios įvedimo lauko | HMP | pelė ]
  klm1 [ Įvesti pradžios datą (metus ir mėnesį) | HM6K | pelė ]
  klm1 [ Paspausti ant intervalo pabaigos įvedimo lauko | MHP | pelė ]
  klm1 [ Įvesti pabaigos datą (metus ir mėnesį) | HM6K | pelė ]
  klm1 [ Paspausti „Rodyti“ | HMP | pelė ]
  klm1 [ Paspausti „Ieškoti intervalo“ | MP | pelė ]
  klm1 [ Pasirinkti ar atsižvelgiama į sezoniškumą | MP | pelė ]
  klm1 [ Įvesti reikiamo intervalo ilgį | MH2K | pelė ]
  klm1 [ Paspausti „Ieškoti mažiausiai apkrauto intervalo“ | MP | pelė ]
}
{
  klm1 [ Paspausti ant intervalo pradžios įvedimo lauko | HMP | pelė ]
  klm1 [ Įvesti pradžios datą (metus ir mėnesį) | HM6K | pelė ]
  klm1 [ Paspausti TAB | MK | pelė ]
  klm1 [ Įvesti pabaigos datą (metus ir mėnesį) | M6K | pelė ]
  klm1 [ Paspausti ENTER | MK | pelė ]
  klm1 [ Paspausti „Ieškoti intervalo“ | HMP | pelė ]
  klm1 [ Pasirinkti ar atsižvelgiama į sezoniškumą | MP | pelė ]
  klm1 [ Įvesti reikiamo intervalo ilgį | MH2K | pelė ]
  klm1 [ Paspausti ENTER | MK | pelė ]
}

\subsection{Použduotis „Prisijungti prie sistemos“}
\klm
{
  klm1 [ Pasiekti pelę | H | ]
  klm1 [ Paspausti ant naudotojo vardo įvedimo lauko | MP | ]
  klm1 [ Įvesti 8 raidžių naudotojo vardą | HM8K | ]
  klm1 [ Paspausti ant slaptažodžio įvedimo lauko | HMP | ]
  klm1 [ Įvesti 8 raidžių slaptažodį | HM8K | ]
  klm1 [ Paspausti prisijungti | HMP | ]
}
{
  klm1 [ Su TAB klavišu pereiti į naudotojo įvedimo lauką | MKK | ]
  klm1 [ Įvesti 8 raidžių naudotojo vardą | M8K | ]
  klm1 [ Pereiti į slaptažodžio įvedimo lauką | MK | ]
  klm1 [ Įvesti 8 raidžių slaptažodį | M8K | ]
  klm1 [ Paspausti ENTER | MK | ]
}

\subsection{Použduotis „Importuoti duomenis“}
\textbf{Sąlygos:} prisijungta prie sistemos.
\klm
{
  klm1 [ Pasiekti pelę | H | pelė ]
  klm1 [ Pasirinkti meniu punktą „Sistema“ | MP | pelė ]
  klm1 [ Pasirinkti meniu punktą „Importuoti“ | MP | pelė ]
  klm1 [ Pasirinkti norimą katalogą | MPMP | pelė ]
  klm1 [ Pasirinkti norimą failą | MP | pelė ]
}
{
  klm1 [ Paspausti importavimo klavišų kombinaciją | MKK | pelė ]
  klm1 [ Pasirinkti meniu punktą „Importuoti“ | HMP | pelė ]
  klm1 [ Pasirinkti norimą katalogą | MPMP | pelė ]
  klm1 [ Pasirinkti norimą failą | MP | pelė ]
}

\subsection{Našumo kėlimo galimybės}
\begin{itemize}
  \item Laisviausių intervalų paieškos pagreitinimui galima sukurti papildomų klavišų kombinacijų, taip būtų išvengtą navigavimo tarp skirtingų formos elementų.
  \item Navigacijai tarp kortelių taip pat galima sukurti sparčiųjų klavišų kombinacijas, taip
  paspartinant naudojimąsi sistema.
\end{itemize}
