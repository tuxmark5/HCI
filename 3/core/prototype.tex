\subsection{Užduoties „Apkrovų skaičiavimas“ scenarijus}

\newcommand{\ri}[1]{(\ref{#1} pav.)}

\begin{enumerate}
  \item \textbf{Prisijungti prie sistemos.} \ref{f:main_connect} paveiksle 
    parodytas prisijungimo sąsaja.
  \item Suvedus naudotojo vardą ir slaptažodį į minėtą sąsają,
    yra parodoma pagrindinė sąsaja \ri{f:main} su sėkmingo 
    prisijungimo pranešimu \ri{f:status_connected}.
  \item \textbf{Importuoti duomenis.} Duomenis importuoti galima
    pasirinkus meniu punktą „Sistema“ ir jame paspaudus
    „Importuoti“ \ri{f:menu_sistema}.
  \item Atsidarius standartiniam OS failų pasirinkimo dialogui, jame
    \textbf{pasirinkti importuojamą failą} \ri{f:open_file}.
  \item Importavus duomenis bus parodytas pranešimas, nurodantis kiek ir
    kokių duomenų buvo importuota \ri{f:status_imported}.
  \item Toliau reikia \textbf{pasirinkti dominantį rėžimą}:
    informacinių sistemų arba padalinių. Tai galima padaryti per
    įrankių juostą \ri{f:toolbar} arba per meniu „Rodyti“
    \ri{f:menu_rodyti}.
  \item Pasirinkus dominantį rėžimą, pasikeičia įrankių juostos
    būsena. Pasirinkus rėžimą „Informacinės sistemos“, atitinkamas
    įrankių juostos mygtukas įsispaudžia \ri{f:toolbar_is}. Pasirinkus
    antrąjį rėžimą, pirmasis mygtukas atsispaudžia ir pasikeičia
    antrojo būsena \ri{f:toolbar_padaliniai}.
  \item Vos pakeitus rėžimą atitinkamai pasikeičia sistemos
    skaičiavimo rėžimas ir atnaujinama grafikuose ir lentelėse matoma
    informacija.
  \item \textbf{Pasirinkti dominantį laiko intervalą.} Tai galima
    padaryti įrankių juostoje pakeičiant intervalo pradžios ir pabaigos
    laukus \ri{f:toolbar} ir po to paspaudžiant mygtuką „Rodyti“,
    esantį šalia intervalo pradžios/pabaigos įvedimo laukų.
  \item Tai padarius, vėl bus atnaujinti grafikai ir juose bus rodomi
    duomenys tik iš pageidauto intervalo.
  \item Jei sąsajoje vaizduojamas duomenų kiekis yra per didelis,
    galima \textbf{išjungti/įjungti individualių IS/padalinių apkrovų
    skaičiavimą ir vizualizavimą}. Tai galima padaryti pagrindinės
    sąsajos dešinėje pusėje panaikinant atitinkamų informacinių
    sistemų/padalinių žymėjimą arba pažymint atitinkamas informacines
    sistemas/padalinius \ri{f:main}.
  \item Jei ir toliau netenkina kurio nors padalinio/IS apkrovų
    vizualizavimas, galima \textbf{pakeisti rodymo rėžimą
    individualiai kiekvienam grafikui} užvedus ant jo pelės žymeklį.
    Tai padarius, ant atitinkamo grafiko atsiras rėžimo pasirinkimo
    mygtukai \ri{f:diag_mover}, kuriais galima pakeisti šio
    padalinio/IS apkrovos pateikimo būdą.
\end{enumerate}

Galimas pavyzdys, kaip atrodytų apkrovos skaičiavimo rezultatas, pateiktas
paveiksle \ref{f:output}

\subsection{Užduoties „Apkrovų prognozavimas“ scenarijus}

Atliekant šią užduotį yra daroma prielaida, jog jau yra
\textbf{prisijungta prie sistemos} ir reikiami \textbf{duomenys jau yra
importuoti}. Jei tai nėra padaryta, reikia atlikti atitinkamus
žingsnius, kurie yra pateikti užduoties „Apkrovų skaičiavimas“
scenarijuje.

\begin{enumerate}
  \item Pirmiausia kairėje sąsajos pusėje reikia \textbf{pasirinkti
    kortelę „Planuojami kiekiai“} \ri{f:main}.
  \item Ten reikia \textbf{suvesti planuojamus administruojamų paramos
    priemonių kiekius ir intervalus bei jų administravimo kaštus}.
  \item Tai padarius, automatiškai bus atnaujinami grafikai ir
    lentelės, kuriuose matysis pageidaujami pasikeitimai dešiniojoje
    pagrindinės sąsajos \ri{f:main} pusėje.
\end{enumerate}

\subsection{Užduoties „Laisviausių laiko intervalų paieška“ scenarijus}

Atliekant šią užduotį yra daroma prielaida, jog jau yra
\textbf{prisijungta prie sistemos} ir reikiami \textbf{duomenys jau
yra importuoti}. Jei tai nėra padaryta, reikia atlikti atitinkamus
žingsnius, kurie yra pateikti užduoties „Apkrovų skaičiavimas“
scenarijuje.

\begin{enumerate}
  \item Tam, kad būtų atlikta prasmingą laisviausių laiko intervalų
    paieška, pirmiausia rekomenduojama atlikti apkrovų prognozavimą.
  \item Tada reikia \textbf{paspausti įrankių juostoje \ri{f:toolbar}
    esantį mygtuką „Ieškoti intervalo“}.
  \item Tai padarius, pagrindinės sąsajos pranešimų sekcijoje
    atsiras forma, leidžianti konfigūruoti intervalo paieškos
    parametrus \ri{f:main_season0}.
  \item Tada galima \textbf{pasirinkti, ar ieškomo intervalo ilgis
    priklausys nuo sezoniškumo}. Šią savybę galima keisti
    pasinaudojant dviem akutėmis, esančiomis intervalo paieškos
    formoje \ri{f:main_season0}.
  \item Pakeitus sezoniškumo parametrą atitinkamai pasikeis formos
    turinys. Pasirinkus nepriklausymą nuo sezoniškumo, bus matoma forma,
    pavaizduota \ref{f:main_season0} paveikslėlyje, o pasirinkus 
    priklausymą – forma, matoma \ref{f:main_season1} paveikslėlyje.
  \item Suvedus atitinkamus formos parametrus ir \textbf{paspaudus
    mygtuką „Ieškoti mažiausiai apkrauto intervalo“}, grafikuose bus
    užrašyti ir paryškinti rasti intervalai.
\end{enumerate}
