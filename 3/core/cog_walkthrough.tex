\subsection{Užduotis „Apkrovų skaičiavimas“}

\xtableu{
  w [ 1 | 20 | 20 | 20 ]
  a [ p | p | p | p ]
  h [
    |
    Žingsnis |
    Ar naudotojas žino (mato), ką ir kaip jam veikti? |
    Ar naudotojui suprantamas sistemos atsakas? ]
  %
  ei [
    Prisijungti prie sistemos. |
    Taip, paleidus programą, pradiniame lange yra sritis „Prisijungimas“,
    su laukais, skirtais įvesti prisijungimo duomenis bei mygtukas
    „Prisijungti“. |
    Taip, parodomas pranešimas „Sėkmingai prisijungta!“ bei mygtukas
    „Prisijungti“ tampa neaktyvus, o „Atsijungti“ – aktyvus. ]
  ei [
    Importuoti pradinius duomenis iš failo. |
    Ne, įrankių juostoje nėra mygtuko „importuoti“ bei meniu juostoje
    nesimato jokio meniu su užrašu „importavimas“, „duomenys“ ar „failas“,
    bet pasirinkus vienintelį neprieštaraujantį meniu „Sistema“, jame
    pamatoma komanda „Importuoti“. |
    Taip, parodomas standartinis failo atvėrimo dialogas.
    ]
  e [||
    Taip, suradus norimą failą renkamasi vienintelį tinkamą pasirinkimą
    „Atverti“. |
    Taip, būsenos juostoje parodomas pranešimas kiek ir kokių duomenų
    buvo importuota.
    ]
  ei [
    Įvesti pradinius duomenis klaviatūra. |
    Taip, tarp kortelių yra viena su pavadinimu „Informacinės sistemos“. |
    Ją nuspaudus parodomas informacinių sistemų sąrašas.
    ]
  e [||
    Ne, nėra akivaizdu, kad paskutinė tuščia eilutė yra skirta naujų
    duomenų įvedimui. |
    Taip, įvedus duomenis atsiranda dar viena tuščia eilutė.
    ]
  ei [
    Importuoti istorinius duomenis iš failo. |
    Ne, įrankių juostoje nėra mygtuko „importuoti“ bei meniu juostoje
    nesimato jokio meniu su užrašu „importavimas“, „duomenys“ ar „failas“,
    bet pasirinkus vienintelį neprieštaraujantį meniu „Sistema“, jame
    pamatoma komanda „Importuoti“. |
    Taip, parodomas standartinis failo atvėrimo dialogas.
    ]
  e [||
    Taip, suradus norimą failą renkamasi vienintelį tinkamą pasirinkimą
    „Atverti“. |
    Taip, būsenos juostoje parodomas pranešimas kiek ir kokių duomenų
    buvo importuota.
    ]
  ei [
    Įvesti pradinius duomenis klaviatūra. |
    Taip, tarp kortelių yra viena su pavadinimu „Istoriniai duomenys“. |
    Ją nuspaudus parodomas istorinių duomenų sąrašas.
    ]
  e [||
    Ne, nėra akivaizdu, kad paskutinė tuščia eilutė yra skirta naujų
    duomenų įvedimui. |
    Taip, įvedus duomenis atsiranda dar viena tuščia eilutė.
    ]
  ei [
    Skaičiuoti apkrovas. |
    Taip, dešinėje lango pusėje, apačioje yra kortelė su užrašu „Apkrova“. |
    Taip, parodomos įvairios diagramos rodančios apkrovas.
    ]
  ei [
    Filtruoti rezultatus. Pasirinkti, kieno apkrovas rodyti –
    padalinių ar IS. |
    Taip, yra matomas meniu punktas „Rodyti“, kuriame galima pasirinkti
    analizuojamą sritį „Informacinės sistemos“ arba „Padaliniai“. |
    Taip, dešinėje lango pusėje sąrašas pasikeitė į pasirinktųjų elementų
    sąrašą.
    ]
  ei [
    Filtruoti rezultatus. Pasirinkti dominančius padalinius/IS. |
    Taip, dešinėje matomas padalinių / IS sąrašas, kuriame galima pažymėti
    norimus matyti padalinius / IS. |
    Taip, pažymėjus (atžymėjus) varnelę atitinkamo padalinio / IS 
    grafikas atsiranda (dingsta).
    ]
  ei [
    Filtruoti rezultatus. Nurodyti aktualų laiko intervalą. |
    Taip, dešinėje matoma frazė „Filtruoti nuo … iki …“ ir du laukai
    leidžiantys keisti datą.
    TODO: Pridėti nuotrauką ir antraštes laukams. |
    Taip, keičiant datą, keičiasi grafikai / lentelės.
    ]
  ei [
    Filtruoti rezultatus. Pasirinkti norimą rodymo rėžimą: ar
    duomenis rodyti grafiškai, ar lentelės pavidalu. |
    Taip, užvedus pelę ant grafiko / lentelės atsiranda pasirinkimo
    mygtukai. |
    Taip, nuspaudus ant atitinkamo mygtuko grafikas (lentelė) pasikeičia
    į lentelę (grafiką).
    ]
}

Rasti defektai:
\begin{itemize}
  \item Nėra akivaizdu, kad „importuoti“ reikia ieškoti meniu „Sistema“
    (\ref{fig:meniu_sistema}). TODO: Pasiūlyti pataisymą.
  \item Nėra akivaizdu, kad paskutinė tuščioji eilutė skirta naujiems
    duomenims (\ref{fig:visas_langas_IS_su_prognoze}). Galimas pataisymas:
    tuščiosios eilutės langelio fone pasvirusiu šriftu įrašyti
    „Pridėti naują IS sistemą“. (TODO: Ekrano nuotrauka.)
\end{itemize}

\ximage{fig:meniu_sistema}{Meniu „Sistema“.}{layout/images/Screens/Sistema_meniu.png}
\ximage{fig:visas_langas_IS_su_prognoze}{Informacinių sistemų įvedimo langas.}{layout/images/Screens/visas_langas_IS_su_prognoze.png}

\subsection{Užduotis „Apkrovų prognozavimas“}

\subsection{Užduotis „Laisviausių laiko intervalų paieška“}
