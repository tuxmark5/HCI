\subsection{Užduotis „Apkrovų skaičiavimas“}

\label{sec:cw_apkrovu_skaiciavimas}

\xtableu{
  w [ 1 | 20 | 20 | 20 ]
  a [ p | p | p | p ]
  h [
    |
    Žingsnis |
    Ar naudotojas žino (mato), ką ir kaip jam veikti? |
    Ar naudotojui suprantamas sistemos atsakas? ]
  %
  ei [
    Prisijungti prie sistemos. |
    Taip, paleidus programą, iš karto matoma sritis „Prisijungimo
    duomenys“, kurioje yra laukai, į kuriuos  galima vesti
    prisijungimo informaciją bei yra mygtukas
    „Prisijungti“. |
    Taip, parodomas pranešimas „Sėkmingai prisijungta!“ bei visos
    kortelės tampa aktyviomis.
    ]
  ei [
    Importuoti pradinius duomenis iš failo. |
    Ne, įrankių juostoje nėra mygtuko „importuoti“ bei meniu juostoje
    nesimato jokio meniu su užrašu „importavimas“, „duomenys“ ar „failas“,
    bet pasirinkus vienintelį neprieštaraujantį meniu „Sistema“, jame
    pamatoma komanda „Importuoti“. |
    Taip, parodomas standartinis failo atvėrimo dialogas.
    ]
  e [||
    Taip, suradus norimą failą renkamasi vienintelį tinkamą pasirinkimą
    „Atverti“. |
    Taip, parodomas pranešimas kiek ir kokių duomenų buvo importuota.
    ]
  ei [
    Įvesti pradinius duomenis klaviatūra. |
    Taip, tarp kortelių yra viena su pavadinimu „IS“. |
    Ją nuspaudus parodomas informacinių sistemų sąrašas.
    ]
  e [||
    Ne, nėra akivaizdu, kad paskutinė tuščia eilutė yra skirta naujų
    duomenų įvedimui. |
    Taip, įvedus duomenis atsiranda dar viena tuščia eilutė.
    ]
  ei [
    Importuoti istorinius duomenis iš failo. |
    Ne, įrankių juostoje nėra mygtuko „importuoti“ bei meniu juostoje
    nesimato jokio meniu su užrašu „importavimas“, „duomenys“ ar „failas“,
    bet pasirinkus vienintelį neprieštaraujantį meniu „Sistema“, jame
    pamatoma komanda „Importuoti“. |
    Taip, parodomas standartinis failo atvėrimo dialogas.
    ]
  e [||
    Taip, suradus norimą failą renkamasi vienintelį tinkamą pasirinkimą
    „Atverti“. |
    Taip, parodomas pranešimas kiek ir kokių duomenų buvo importuota.
    ]
  ei [
    Įvesti pradinius duomenis klaviatūra. |
    Taip, tarp kortelių yra viena su pavadinimu „Istoriniai duomenys“. |
    Ją nuspaudus parodomas istorinių duomenų sąrašas.
    ]
  e [||
    Ne, nėra akivaizdu, kad paskutinė tuščia eilutė yra skirta naujų
    duomenų įvedimui. |
    Taip, įvedus duomenis atsiranda dar viena tuščia eilutė.
    ]
  ei [
    Skaičiuoti apkrovas. |
    Taip, dešinėje lango pusėje, apačioje yra kortelė su užrašu
    „Apkrova ir prognozės“. |
    Taip, kiekvienam padaliniui parodoma po jo apkrovos informaciją.
    ]
  ei [
    Filtruoti rezultatus. Pasirinkti, kieno apkrovas rodyti –
    padalinių ar IS. |
    Taip, įrankių juostoje yra matomi du mygtukai „Informacinės sistemos“
    ir „Padaliniai“. |
    Taip, dešinėje lango pusėje sąrašas pasikeitė į pasirinktųjų elementų
    sąrašą.
    ]
  ei [
    Filtruoti rezultatus. Pasirinkti dominančius padalinius / IS. |
    Taip, dešinėje matomas padalinių / IS sąrašas, kuriame galima pažymėti
    norimus matyti padalinius / IS. |
    Taip, pažymėjus (atžymėjus) varnelę atitinkamo padalinio / IS 
    grafikas atsiranda (dingsta).
    ]
  ei [
    Filtruoti rezultatus. Nurodyti aktualų laiko intervalą. |
    Taip, įrankių juostoje matoma „Rodyti nuo … iki …“ bei mygtukas
    „Rodyti“. |
    Taip, keičiant datą, keičiasi grafikai / lentelės.
    ]
  ei [
    Filtruoti rezultatus. Pasirinkti norimą rodymo rėžimą: ar
    duomenis rodyti grafiškai, ar lentelės pavidalu. |
    Taip, užvedus pelę ant grafiko / lentelės atsiranda pasirinkimo
    mygtukai. |
    Taip, nuspaudus ant atitinkamo mygtuko grafikas (lentelė) pasikeičia
    į lentelę (grafiką).
    ]
}

\subsection{Užduotis „Apkrovų prognozavimas“}

\label{sec:cw_apkrovu_prognozavimas}

\textbf{Pastaba:} kadangi prisijungimas prie sistemos, pradinių
duomenų įvedimas bei rezultatų filtravimas jau buvo nagrinėti
\nameref{sec:cw_apkrovu_skaiciavimas} skyrelyje, tai jie praleisti.

\xtableu{
  w [ 1 | 20 | 20 | 20 ]
  a [ p | p | p | p ]
  h [
    |
    Žingsnis |
    Ar naudotojas žino (mato), ką ir kaip jam veikti? |
    Ar naudotojui suprantamas sistemos atsakas? ]
  %
  ei [
    Importuoti planuojamas apimtis iš failo. |
    Ne, įrankių juostoje nėra mygtuko „importuoti“ bei meniu juostoje
    nesimato jokio meniu su užrašu „importavimas“, „duomenys“ ar „failas“,
    bet pasirinkus vienintelį neprieštaraujantį meniu „Sistema“, jame
    pamatoma komanda „Importuoti“. |
    Taip, parodomas standartinis failo atvėrimo dialogas.
    ]
  e [||
    Taip, suradus norimą failą renkamasi vienintelį tinkamą pasirinkimą
    „Atverti“. |
    Taip, parodomas pranešimas kiek ir kokių duomenų buvo importuota.
    ]
  ei [
    Įvesti planuojamas apimtis klaviatūra. |
    Taip, tarp kortelių yra viena su pavadinimu „Planuojami kiekiai“. |
    Ją nuspaudus parodomas planuojamų kiekių sąrašas.
    ]
  e [||
    Ne, nėra akivaizdu, kad paskutinė tuščia eilutė yra skirta naujų
    duomenų įvedimui. |
    Taip, įvedus duomenis atsiranda dar viena tuščia eilutė.
    ]
  ei [
    Prognozuoti apkrovas. |
    Taip, dešinėje lango pusėje, apačioje yra kortelė su užrašu
    „Apkrova ir prognozės“. |
    Taip, parodomos diagramos su esamomis apkrovomis bei jų prognozėmis 
    pasirinktiems padaliniams / informacinėms sistemoms.
    ]
  }

\subsection{Užduotis „Laisviausių laiko intervalų paieška“}

\textbf{Pastaba:} kadangi prisijungimas prie sistemos bei pradinių
duomenų įvedimas jau buvo nagrinėti
\nameref{sec:cw_apkrovu_skaiciavimas} skyrelyje, o informacinių
sistemų / padalinių apkrovų prognozavimas
\nameref{sec:cw_apkrovu_prognozavimas} skyrelyje, tai jie praleisti.

\xtableu{
  w [ 1 | 20 | 20 | 20 ]
  a [ p | p | p | p ]
  h [
    |
    Žingsnis |
    Ar naudotojas žino (mato), ką ir kaip jam veikti? |
    Ar naudotojui suprantamas sistemos atsakas? ]
  %
  ei [
    Pasirinkti dominančią IS/padalinį. |
    Taip, įrankių juostoje matomas mygtukas Informacinės sistemos
    (Padaliniai). |
    Taip, dešinėje pusėje vietoj padalinių sąrašo parodomas informacinių
    sistemų sąrašas arba atvirkščiai.
    ]
  e [||
    Taip, informacinių sistemų (padalinių) sąraše galima pažymėti norimas
    sistemas (padalinius). |
    Taip, pažymėjus (atžymėjus) sistemą (padalinį) atitinkantis
    grafikas atsiranda (pradingsta).
    ]
  ei [
    Pasirinkti laiko intervalą, kuriame ieškoma mažiausia apkrova. |
    Taip, viršuje matomas pasirinkimas „Rodyti nuo … iki …“ bei
    mygtukas „Rodyti“. |
    Taip, pasikeičia grafikų rodomas laiko intervalas.
    ]
  ei [
    Pasirinkti, ar ieškomas intervalo ilgis keičiasi dinamiškai nuo
    dabartinio sezono. |
    Taip, įrankių juostoje yra mygtukas „Ieškoti intervalo“. |
    Taip, atsiranda papildomi laukai, skirti įvesti informacijai.
    ]
  e [||
    Taip, yra pasirinkimas ar atsižvelgti į sezoniškumą ar
    neatsižvelgti. |
    ]
  ei [
    Pasirinkti, kokio ilgio intervalo ieškoma. |
    Taip, yra laukelis(-iai) į kurį(-iuos) galima įvesti dienų kiekį. |
    ]
  ei [
    Ieškoti mažiausių apkrovų. |
    Taip, yra mygtukas „Ieškoti mažiausiai apkrauto intervalo“. |
    Taip, diagramose pasikeičia atitinkamo intervalo spalva.
    ]
  }

\subsection{Rasti defektai}

\begin{itemize}
  \item Nėra akivaizdu, kad „importuoti“ reikia ieškoti meniu „Sistema“
    (\ref{fig:meniu_sistema}). TODO: Pasiūlyti pataisymą.
  \item Nėra akivaizdu, kad paskutinė tuščioji eilutė skirta naujiems
    duomenims (\ref{fig:visas_langas_IS_su_prognoze}). Galimas pataisymas:
    tuščiosios eilutės langelio fone pasvirusiu šriftu įrašyti
    „Pridėti naują IS sistemą“. (TODO: Ekrano nuotrauka.)
\end{itemize}

\ximage{fig:meniu_sistema}{Meniu „Sistema“.}{layout/images/Screens/Sistema_meniu.png}
\ximage{fig:visas_langas_IS_su_prognoze}{Informacinių sistemų įvedimo langas.}{layout/images/Screens/visas_langas_IS_su_prognoze.png}
