\begin{xcase}{Simbolių šalinimas Samsung GT-C5130S mobiliojo telefono
              Java ME programose}
  \xcgoal
  {
    Kaip ir daugelis kitų mobiliųjų telefonų, Samsung GT-C5130S gali vykdyti
    Java Platform, Micro Edition platformos programas. Tokių programų
    teksto laukų redagavimui telefono virtuali mašina suteikia standartinę
    sąsają (\ref{fig:case_m_1_1} paveikslas). Rašant tekstą dažnai tenka jį
    taisyti, pavyzdžiui, ištrinant  netinkamą simbolį.

    \ximage{fig:case_m_1_1}{Teksto redagavimas}{martynas/images/case_1_1.png}
  }
  
  \xctools
  {
    Norint pašalinti simbolį, reikia atlikti tokius veiksmus:
    \begin{enumerate}
      \item Kairiuoju funkciniu mygtuku pasirinkti „Meniu“.
        (\ref{fig:case_m_1_2} paveikslas)
      \item Atsidariusiame meniu pasirinkti „Valyti“.
        (\ref{fig:case_m_1_3} paveikslas)
      \item Atsidariusiame žemesnio lygio meniu pasirinkti veiksmą „Valyti“.
        (\ref{fig:case_m_1_4} paveikslas)
    \end{enumerate}

    \ximagerow{fig:case_m_1}{Simbolio šalinimo procedūra}{3}
    {
      \ximagecell{fig:case_m_1_2}{Pasirinkus „Meniu“}{martynas/images/case_1_2.png}
      \ximagecell{fig:case_m_1_3}{Pasirinkus „Valyti“}{martynas/images/case_1_3.png}
      \ximagecell{fig:case_m_1_4}{Antrąkart pasirinkus „Valyti“}{martynas/images/case_1_4.png}
    }
  }
  
  \xcresult
  {
    Ištrinamas prieš žymeklį buvęs simbolis.
  }
  
  \xcprinciples
  {
    \xpentry{Darna}{Pažeistas}
    {
      Šiame telefone rašant tekstą kitose sąsajose (pavyzdžiui, rašant
      žinutę ar telefono numerį) funkcija „Valyti“ vykdoma nuspaudus
      dešinįjį funkcinį mygtuką. Šioje sąsajoje dešinysis mygtukas atlieka
      funkciją „Atšaukti“, kuri išjungia teksto lauko redagavimą ir
      ištrina visą anksčiau įvestą tekstą.
    }
    \xpentry{Našumas}{Pažeistas}
    {
      Simbolio šalinimas yra dažnai naudojama funkcija. Tai, kad jai
      įvykdyti reikia mažiausiai 3 mygtukų paspaudimų stipriai mažina
      sąsajos naudojimo našumą.
    }
  }
  
  \xcthoughts
  {
    Labai keista, kad šios sąsajos programuotojai nesiėmė standartinės, kone
    visuose telefonuose taikomos praktikos, ir taip apsunkino tokią paprastą
    simbolio šalinimo procedūrą. Ji ne tik užima santykinai daug laiko, bet
    ir kelia pavojų, kad dėl įpročių naudojantis kitomis telefono įvesties
    sąsajomis per klaidą bus ištrintas visas parašytas tekstas.
  }
\end{xcase}
