\begin{xcase}{VU studentų atsiliepimų apie Erasmus studijas peržiūra}
  \xcgoal
  {
    Tinklalapyje \url{http://www.erasmus.tprs.vu.lt/ataskaita/atsiliepimai/}
    pateikiama prieiga prie VU studentų atsiliepimų apie Erasmus studijas
    partnerinėse institucijose duomenų bazės. Norima pasižiūrėti jos duomenis.
  }
  
  \xctools
  {
    Pasirinkus šalį, universitetą bei studentą pasirodo
    \ref{fig:case_4_erasmus} paveiksle pavaizduotas langas. Atrodo, tarsi
    atitinkami studijų įvertinimai turėtų būti pateikti lentelėje, tačiau
    joje nėra duomenų – galbūt jie dar nesuvesti. „Suomija“, „Helsinki“ bei
    datos šiame paveiskle primena nuorodas, tačiau yra paprastas tekstas.

    \ximage{fig:case_4_erasmus}{Informacijos peržiūros langas}{martynas/images/case_4_erasmus.png}
  }
  
  \xcresult
  {
    Peržiūrint kitų studentų anketas, taip pat nepavyko rasti daugiau jokių
    kitų duomenų – visur būdavo tik tokios pačios lentelių antraštės. Tik
    gerokai vėliau sužinojau, jog tos antraštės yra nuorodos į puslapius
    atitinkamo pobūdžio klausimais ir atsakymais.
  }
  
  \xcprinciples
  {
    \xpentry{Darna}{Pažeistas}
    {
      Įprastai saityno nuorodos atskiriamos pagal melsvą šriftą ir
      pabraukimą. Šiuo atveju nuorodos yra parašytos juodu šriftu.
    }
  }
  
  \xcthoughts
  {
    <MINTYS>
  }
\end{xcase}
