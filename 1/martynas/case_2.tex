\begin{xcase}{Žodžio įtraukimas į T9 žodyną Samsung GT-C5130S mobiliajame
              telefone}
  \xcgoal
  {
    Kaip ir daugelis kitų mobiliųjų telefonų, kuriuose vienas klavišas
    atitinka skirtingas raides, Samsung GT-C5130S galima rašyti tekstines
    žinutes naudojant T9\footnote{Žr. http://t9.com} teksto nuspėjimą. Kai
    programa negali nuspėti norimo parašyti žodžio, siūloma jį įtraukti į
    žodyną įvedant įprastiniu būdu paraidžiui (kiekvieną klavišą spaudžiant
    po kelis kartus, kol pasirodys tinkama raidė).
  }
  
  \xctools
  {
    Pavyzdžiui, žinutėje norime parašyti žodį „neprirašyk“. Paspaudus
    klavišus 6, 3, 7, 7, 4, 7 ir 2, telefonas ekrane rodo „nepripa“ ir,
    nustatęs, kad tinkamo žodžio neras, siūlo jį pridėti
    (\ref{fig:case_m_2_1} paveikslas). Pasirinkus „Pridėti žodį“,
    atsidariusiame dialoge įvedame „neprirašyk“ ir spaudžiame
    „Įtraukti“ (\ref{fig:case_m_2_2} paveikslas).
  }

  \ximagerow{fig:case_m_2}{Žodžio įterpimas T9}{3}
  {
    \ximagecell{fig:case_m_2_1}{T9 nepavyksta nuspėti žodžio}{martynas/images/case_2_1.png}
    \ximagecell{fig:case_m_2_2}{Žodžio įterpimas}{martynas/images/case_2_2.png}
    \ximagecell{fig:case_m_2_3}{Po įterpimo}{martynas/images/case_2_3.png}
  }
  
  \xcresult
  {
    Žinutėje vietoj norimo žodžio atsirado žodis „nepripaneprirašyk“ (\ref{fig:case_m_2_3} paveikslas).
  }
  
  \xcprinciples
  {
    \xpentry{Našumas}{Pažeistas}
    {
      Tokioje žinutės rašymo T9 sąsajoje, naudotojui kaskart įvedus nežinomą
      žodį ir jį įtraukus vedant paraidžiui, teks trinti simbolius.
    }
  }
  
  \xcthoughts
  {
    Šios našumo ir naudojimo malonumo problemos buvo galima išvengti
    naudojant įprastą žinučių rašymo T9 sąsajos realizaciją, kada įtraukiant
    žodį jis pakeičia tai, kas buvo parašyta prieš iškviečiant šią funkciją,
    o ne yra prirašomas gale.
  }
\end{xcase}
