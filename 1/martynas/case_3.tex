\begin{xcase}{Skambučių priėmimas Samsung GT-C5130S mobiliajame telefone}
  \xcgoal
  {
    Atsiliepti į telefono skambutį.
  }
  
  \xctools
  {
    Skambučio pradžioje kairysis mygtukas susietas su funkcija „Priimti“, o
    dešinysis – „Atmesti“ (\ref{fig:case_m_3_calling} paveikslas). Jeigu nuspaudžiamas
    koks nors skaičių klaviatūros mygtukas, pasirodo skaičių įvedimo laukelis,
    neturintis jokio kito funkcionalumo, negu „Valyti“ (dešinysis mygtukas) ir
    „Atmesti“ (kairysis mygtukas; \ref{fig:case_m_3_number} paveikslas). Išvalius
    visus įvestus skaičius, šis „Valyti“ mygtukas persivadina į „Atgal“
    (\ref{fig:case_m_3_empty} paveikslas), kurį nuspaudus grįžtama į pirminį vaizdą.

    \ximagerow{fig:case_m_3}{Skambučio priėmimas}{3}
    {
      \ximagecell{fig:case_m_3_calling}{Skambutis}{martynas/images/case_3_calling.png}
      \ximagecell{fig:case_m_3_number}{Nuspaudus mygtuką 5}{martynas/images/case_3_number.png}
      \ximagecell{fig:case_m_3_empty}{Išvalius laukelį}{martynas/images/case_3_empty.png}
    }
  }
  
  \xcresult
  {
    Jei naudojantis telefonu kas nors skambina, dažnai skambutis atmetamas,
    nors ir norima atsiliepti.
  }
  
  \xcprinciples
  {
    \xpentry{Darna}{Pažeistas}
    {
      Įprastai „Atmesti“ funkciją atlieka dešinysis mygtukas, tačiau
      nuspaudus kokį nors skaičių ji persikelia į kairę. Telefono
      skambėjimo metu pagrindinis dėmesys turėtų būti priėmimo ir atmetimo
      funkcijomis.
    }
  }
  
  \xcthoughts
  {
    Skambučio priėmimo metu atsirandantis skaičių įvedimo laukelis yra
    visiškai bereikšmis. Su jame įrašytais skaičiais negalima nieko kito
    nuveikti, negu kad juos ištrinti. Jei skambinantysis padeda ragelį,
    tas laukas irgi dingsta.

    Jei naudojantis telefonu (pavyzdžiui, rašant žinutę) kas nors skambina,
    pasirodžius skambučiui ir norint paspausti „Priimti“, dažnai per klaidą
    atmetamas skambutis, kadangi telefonas dar nebūna spėjęs parodyti
    skaičių įvesties laukelio, tačiau mygtukų funkcijos jau būna
    apsikeitusios.
  }
\end{xcase}
