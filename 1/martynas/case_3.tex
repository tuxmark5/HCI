\begin{xcase}{Skambučių priėmimas Samsung GT-C5130S mobiliajame telefone}
  \xcgoal
  {
    Norime atsiliepti telefono skambutį.
  }
  
  \xctools
  { % TODO: iliustracijos
  }
  
  \xcresult
  {
    Skambučio pradžioje kairysis mygtukas susietas su funkcija „Priimti“, o
    dešinysis – „Atmesti“. Jeigu nuspaudžiamas koks nors skaičių klaviatūros
    mygtukas, pasirodo skaičių įvedimo laukelis neturintis jokio kito
    funkcionalumo, negu „Valyti“ (dešinysis mygtukas) ir „Atmesti“ (kairysis
    mygtukas). Išvalius visus įvestus skaičius, šis „Valyti“ mygtukas
    persivadina į „Atgal“, kurį nuspaudus grįžtama į pirminį vaizdą.
  }
  
  \xcprinciples
  {
    \xpentry{Darna}{Pažeistas}
    {
      Įprastai „Atmesti“ funkciją atlieka kairysis mygtukas, tačiau
      nuspaudus kokį nors skaičių ji persikelia į dešinę. Telefono
      skambėjimo metu pagrindinis dėmesys turėtų būti priėmimo ir atmetimo
      funkcijomis.
    }
  }
  
  \xcthoughts
  {
    Skambučio priėmimo metu atsirandantis skaičių įvedimo laukelis yra
    visiškai bereikšmis. Su jame įrašytais skaičiais negalima nieko kito
    nuveikti, negu kad juos ištrinti. Jei skambinantysis padeda ragelį,
    tas laukas irgi dingsta.

    Jei naudojantis telefonu (pavyzdžiui, rašant žinutę) kas nors skambina,
    pasirodžius skambučiui ir norint paspausti „Priimti“, dažnai per klaidą
    atmetamas skambutis, kadangi telefonas dar nebūna spėjęs parodyti
    skaičių įvesties laukelio, tačiau mygtukų funkcijos jau būna
    apsikeitusios.
  }
\end{xcase}
