\begin{xcase}{Google Chrome (Chromium) optimizuota viršutinė naršyklės dalis}
  \xcgoal
  	Prieš keletą metų interneto naršyklės, ypač jų viršutinė dalis, kur yra adreso
	juosta bei nustatymai, užimdavo nemažą ekrano dalį (dažniausia daugiau negu 120px).
	Tai tapo dar aktualiau išpopuliarėjaus 
	plačiaekraniams monitoriams bei nešiojamiems kompiuteriams, nes yra palyginti 
	mažesnis ekrano aukštis su ankstesniais modeliais. Chromium naršyklės
	\footnote{Projekto svetainė: \url{http://www.chromium.org/}.} kūrėjai pritaikė 
	daug naujovių, kad lango apimtis būtų kuo mažesnė.
  }
  
  \xctools
  {
    Chromium kūrėjai perkėlė puslapių langus (angl. Tabs) į prieš tai dažniausia 
	Windows operacinėje sistemoje nenaudojamą erdvę - viršutinę lango antraštę.
	Taip pat visos nustatymų ir papildomos funkcijos perkeltos į vieną atsidarantį
	langelį pažymėtą mechaninio rakto simboliu (pav. @ref{fig:case_3_chrome_settings}).
	
	\ximagerow{fig:case_3_chromium}{Chromium ir FireFox 3 langų palyginimas}{3}
    {
      \ximagecell{fig:case_3_chrome}{Google Chrome naršyklė}{egidijus/images/case_3_chrome.png}
	  \ximagecell{fig:case_3_firefox3}{FireFox 3.0 naršyklė}{egidijus/images/case_3_firefox3.png}
	  \ximagecell{fig:case_3_chrome_settings}{Google Chrome naršyklės nustatymų meniu}{egidijus/images/case_3_chrome_settings.png}
    }
  }
  
  \xcresult
  {
  	Pritaikius naują Chromium išdėstymą, naršyklės viršutinė dalis pastebimai sumažėjo.
	Dėl to atsirado daugiau vietos atverto puslapio regimąjam turiniui ir vartotojams 
	mažiau reikia slankioti puslapio aukščio reguliatorių (angl. Scrollbar).
  }
  
  \xcprinciples
  {
    \xpentry{Robastiškumas (matomumas)}{Įgyvendintas}{
		Pritaikius minimalistinę sąsają, vartotojas mato tik jam reikalingiausius 
		naršymui įrankius: internetinio adreso juostą bei atidarytų puslapių
		langus. Tačiau visos reikalingos, bet rečiau naudojamos, funkcijos
		išlieka. Taip vartotojui pateikiamas maksimalus regimasis puslapių 
		plotas.
	}
  }
  
  \xcthoughts
  {
    Išnaudojus optimaliai ekrano plotą, buvo pasiektas rezultatas, kuris tenkina mano,
	kaip interneto naršyklės vartotojo, poreikius: greitas naršymas, didelis puslapio
	matomumo plotas, jokių regimų nenaudojamų mygtukų.
  }
\end{xcase}
