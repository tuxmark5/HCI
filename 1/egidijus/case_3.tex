\begin{xcase}{Google Chrome (Chromium) optimizuota viršutinė naršyklės dalis}
  \xcgoal
  {
    Prieš keletą metų interneto naršyklės, ypač jų viršutinė
    dalis, kur yra adreso juosta bei nustatymai, užimdavo nemažą
    ekrano dalį (dažniausia daugiau negu 120 pikselių). Tai tapo dar
    aktualiau išpopuliarėjus  plačiaekraniams monitoriams bei
    nešiojamiems kompiuteriams, nes ekrano aukščio ir pločio
    santykis tampa vis mažesnis lyginant su senesniais modeliais.
    Chromium naršyklės \footnote{Projekto svetainė:
    \url{http://www.chromium.org/}.} kūrėjai pritaikė  daug
    naujovių, kad lango apimtis būtų kuo mažesnė.
  }
  
  \xctools
  {
    Chromium kūrėjai perkėlė puslapių korteles (angl. Tabs) į
    prieš tai dažniausiai Windows operacinėje sistemoje nenaudojamą
    erdvę – lango antraštę. Taip pat visos nustatymų ir papildomos
    funkcijos perkeltos į vieną atsidarantį langelį pažymėtą
    mechaninio rakto simboliu (pav. \ref{fig:case_3_chrome_settings}).
	
    \ximagerow{fig:case_3_chromium}{Chromium ir Firefox 3 langų palyginimas}{3}
    {
      \ximagecell{fig:case_3_chrome}{Google Chrome naršyklė}{egidijus/images/case_3_chrome.png}
      \ximagecell{fig:case_3_firefox3}{FireFox 3.0 naršyklė}{egidijus/images/case_3_firefox3.png}
      \ximagecell{fig:case_3_chrome_settings}{Google Chrome naršyklės nustatymų meniu}{egidijus/images/case_3_chrome_settings.png}
    }
  }
  
  \xcresult
  {
    Pritaikius naują Chromium išdėstymą, naršyklės viršutinė dalis
    pastebimai sumažėjo. Dėl to atsirado daugiau vietos atverto puslapio
    regimajam turiniui ir naudotojams mažiau reikia slankioti puslapio
    slinkties juostą (angl. scrollbar).
  }
  
  \xcprinciples
  {
    \xpentry{Robastiškumas (matomumas)}{Įgyvendintas}
    {
      Pritaikius minimalistinę sąsają, naudotojas mato tik jam
      reikalingiausius naršymui įrankius: internetinio adreso juostą
      bei atidarytų puslapių langus. Tačiau visos kitos reikalingos,
      bet rečiau naudojamos, funkcijos išlieka. Taip naudotojui
      pateikiamas maksimalus regimasis puslapių  plotas.
    }
  }
  
  \xcthoughts
  {
    Išnaudojus optimaliai ekrano plotą, buvo pasiektas rezultatas,
    kuris tenkina interneto naršyklės naudotojo poreikius: greitas
    naršymas, didelis puslapio matomumo plotas, jokių nenaudojamų
    mygtukų nuolat matomoje sąsajos dalyje.
  }
\end{xcase}
