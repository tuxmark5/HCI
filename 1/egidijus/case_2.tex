\begin{xcase}{Failų peržiųra Nokia Symbian v52.0.101 operacinėje sistemoje}
  \xcgoal
  {
    Norėta peržiūrėti failus, esančius telefono atminties kortelėje. 
    Buvo naudojama Nokia Symbian v52.0.101 \footnote{Nokia Symbian puslapis: \url{http://symbian.nokia.com/}.} 
    operacinės sistemos failų tvarkyklės programa (File Manager). 
  }
  
  \xctools
  {
    Nokia Symbian operacinės sistemos failų tvarkyklė leidžia naviguoti po
	katalogus, esančius telefono vidinėje bei išorinėse atmintinėse. Taip 
	pat, suradus norimą failą, jį galima tiesiogiai atidaryti peržiūrai.
  }
  
  \xcresult
  {
    Atminties kortelėje radus dainą ir paspaudus ant jos pavadinimo atsidaro muzikos
    grotuvo langas bei daina pradeda groti. Tada paspaudžiamas muzikos grotuve mygtukas 
    Atgal („Back“) (pav. \ref{fig:case_2_music_player}), kad uždaryti grotuvo langą 
    ir grįžti į failų tvarkyklę. Tačiau, užsidarius muzikos grotuvui, grįžtama ne į
    failų tvarkyklės langą, bet į pagrindinį meniu (pav. \ref{fig:case_2_home_menu}). 
    Tik įdėmiau įsižiūrėjus galima pastebėti, jog failų tvarkyklė liko aktyvi fone 
    (pav. \ref{fig:case_2_background}) ir į ją reikia vėl persijungti.
	
    \ximagerow{fig:case_2_nokia}{Failo peržiūra Nokia Symbian su File Manager}{4}
    {
      \ximagecell{fig:case_2_file_manager}{Failų tvarkyklė}{egidijus/images/case_2_file_manager.jpg}
      \ximagecell{fig:case_2_music_player}{Atsidaro muzikos grotuvas}{egidijus/images/case_2_music_player.png}
      \ximagecell{fig:case_2_home_menu}{Grįžtama į pagrindinį meniu}{egidijus/images/case_2_home_menu.jpg}
      \ximagecell{fig:case_2_background}{Failų tvarkyklė lieka fone}{egidijus/images/case_2_background.jpg}
    }
  }
  
  \xcprinciples
  {
    \xpentry{Nuspėjamumas}{Pažeistas}
    {
      Muzikos grotuvo lange paspaudus mygtuką Atgal(„Back“) (pav. @ref{fig:case_2_music_player}) buvo tikėtasi grįžti 
      į prieš tai buvusį aktyvų failų tvarkyklės langą. Tačiau buvo atidarytas 
      pagrindinio meniu langas (pav. @ref{fig:case_2_home_menu}).
    }
  }
  
  \xcthoughts
  {
    Greičiausiai tai yra programavimo klaida, kai nesuderinami procesų tarpusavio gryžimo keliai
    tarp dviejų atskirai veikiančių programų.
  }
\end{xcase}
