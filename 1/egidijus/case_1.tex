\begin{xcase}{Tekstinės žinutės keitimas iOS ver.3.2.3}
  \xcgoal
  	Naudojant iPhone 3Gs 
	iOS 4.3.5)\footnote{Produkto puslapis: \url{http://www.apple.com/iphone}.}
	buvo mėginama parašyti trumpąją žinutę, bet surinkus tekstą, pastebėta 
	viename žodžių padaryta rašybos klaida. Todėl norėta grįžti i blogo 
	simbolio vietą bei ištaisyti klaidą.
	
	\ximage{fig:case_1_iPhone_sms_window}{iPhone SMS rašymo langas}{egidijus/images/case_1_iPhone_sms_screen.png}
  }
  
  \xctools
  {
    Žinutės rašymo lange yra matomas jau surinktas tekstas bei galima 
    naviguoti tarp žodžių pirštu paspaudžiant norimoje vietoje. 
  }
  
  \xcresult
  {
    Paspaudus norimoje vietoje, rašymo žymeklis atsiranda arba žodžio
    pabaigoje, arba pradžioje. Kadangi nėra jokių klavišų navigavimui tarp
    atskirų pavienių raidžių, tai, norint ištaisyti klaidą 
	įrašytą atsitiktinę raidę), reikia ištrinti visą žodžio dalį, o ne tik 
    klaidingą simbolį, bei perrašyti trūkstamą žodžio dalį pakartotinai. Tai 
    užima papildomai laiko bei dėmesio sutelkimo.
  }
  
  \xcprinciples
  {
    \xpentry{Nuspėjamumas}{Pažeistas}{
		Buvo tikėtasi, jog paspaudus ant žodžio toje vietoje, kur padaryta klaida,
		ten atsiras ir rašymo žymeklis bei bus galima ištrinti nereikalingą vieną
		simbolį.
	}
    \xpentry{Lankstumas>}{Pažeistas}{
		Vartotojui nėra palikta galimybė pasirinkti redaguoti visą parašyta žodį
		ar tik jo dalį.
	}
  }
  
  \xcthoughts
  {
    iOS mobiliojo telefono operacinė sistema turi įdiegtus daugelio kalbų
    žodynus bei gramatikos tikrinimą. Todėl tekstą rašant anglų kalba, 
    klaidos dažniausia automatiškai ištaisomos bei nėra poreikio taisyti 
    žodžių vidurinę dalį. Dėl to, gali būti, jog kūrėjai nenorėjo daryti 
    papildomų mygtukų bei užimti vietą ekrane. 
    Tačiau rašant lietuviškai, automatinis teksto taisymas nėra pakankamai 
    gerai ištobulintas, kad jį būtų patogu naudoti. Vartotojai dažnai šią
    funkciją išjungia, rašydami padaro klaidų ir prisireikia jas taisyti.
  }
\end{xcase}
