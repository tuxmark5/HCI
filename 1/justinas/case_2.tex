\begin{xcase}{Prisijungimas prie Yahoo! Mail}
  \xcgoal
  {
    Naudojant naršyklę prisijungti prie Yahoo! Mail neatsinaujinant į
    naujesnę pašto versiją.
  }
  \xctools
  {
    \ref{fig:case_2_yahoo} paveikslėlyje parodytas prašymo atsinaujinti langas.

    \ximage{fig:case_2_yahoo}{Yahoo! Mail atsinaujinimo langas}{justinas/images/case_2.jpg}
  }
  \xcresult
  {
    Kadangi nebuvo norima atsinaujinti į naujesnę versiją, o „Don't Upgrade“ arba „Upgrade Later“ 
    mygtuko nesimato, teko perskaityti visą tekstą, kad būtų surasta, ką reikia paspausti, kad 
    išliktų senoji versija.
  }
  \xcprinciples
  {
    \xpentry{Apibendrinimas}{Pažeistas}
    {
      Dauguma programų aiškiai pateikia pasirinkimą naudotojui, ar
      šis nori atsinaujinti naudojamą programą/paslaugą dabar ar
      vėliau.
    }
    \xpentry{Dialogo iniciatyva}{Pažeistas}
    {
      Didžioji dauguma programų aiškiai nurodo galimas pasirinktis.
    }
  }
  \xcthoughts
  {
    Tikėtina, kad buvo norima priversti naudotoją atsinaujinti. Norint
    išvengti tokių nesupratimų, turėtų būti visi įmanomi pasirinkimai 
    rodomi aiškiai.
  }
\end{xcase}
