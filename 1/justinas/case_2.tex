\begin{xcase}{Prisijungimas prie Yahoo! Mail}
  \xcgoal
  {
    Naudojant naršyklę prisijungti prie Yahoo! Mail neatsinaujinant į naujesnę versiją.
  }
  \xctools
  {
    \ref{fig:case_2} paveikslėlyje parodytas prašymo atsinaujinti langas.

    \ximage{fig:case_2}{Yahoo! Mail atsinaujinimo langas}{justinas/images/case_2.jpg}
  }
  \xcresult
  {
    Kadangi nebuvo norima atsinaujinti į naujesnę versiją, o „Don't Upgrade“  arba „Upgrade Later“ mygtuko nesimato teko perskaityti visą tekstą, kad surasti, kur reikia paspausti, kad išliktų senoji versija.
  }
  \xcprinciples
  {
	\xpentry{Dialogo iniciatyva}{Pažeistas}
	{
	  Didžioji dauguma programų aiškiai nurodo galimas pasirinktis.
	}
  }
  \xcthoughts
  {
    Tikėtina, kad buvo norima priversti vartotoją atlikti tam tikrą veiksmą. Norint išvengti tokių nesupratimų, turėtų būti visi įmanomi pasirinkimai rodomi aiškiai.
  }
\end{xcase}