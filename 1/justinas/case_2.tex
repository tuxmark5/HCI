\begin{xcase}{Prisijungimas prie Yahoo! Mail}
  \xcgoal
  {
    % kieno versiją? pašto? reiktų aiškiau nurodyti.
    Naudojant naršyklę prisijungti prie Yahoo! Mail neatsinaujinant į naujesnę versiją.
  }
  \xctools
  {
    \ref{fig:case_2} paveikslėlyje parodytas prašymo atsinaujinti langas.

    \ximage{fig:case_2}{Yahoo! Mail atsinaujinimo langas}{justinas/images/case_2.jpg}
  }
  \xcresult
  {
    Kadangi nebuvo norima atsinaujinti į naujesnę versiją, o „Don't Upgrade“ arba „Upgrade Later“ 
    mygtuko nesimato, teko perskaityti visą tekstą, kad surasti, ką reikia paspausti, kad 
    išliktų senoji versija.
  }
  \xcprinciples
  {
    \xpentry{Apibendrinimas}{Pažeistas}
    {
      Dauguma programų aiškiai pateikia pasirinkimą vartotojui, ar šis nori atsinaujinti
      naudojamą programą/paslaugą dabar ar vėliau.
    }
    \xpentry{Dialogo iniciatyva}{Pažeistas}
    {
      Didžioji dauguma programų aiškiai nurodo galimas pasirinktis.
    }
  }
  \xcthoughts
  {
    % Kokį veiksmą?
    Tikėtina, kad buvo norima priversti vartotoją atlikti tam tikrą veiksmą. Norint išvengti 
    tokių nesupratimų, turėtų būti visi įmanomi pasirinkimai rodomi aiškiai.
  }
\end{xcase}
