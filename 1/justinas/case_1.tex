\begin{xcase}{„Recovery Toolbox for Word“ programos išjungimas}
  \xcgoal
  {
    Išjungti „Recovery Toolbox for Word“ programą.
  }
  \xctools
  {
    \ref{fig:case_1} paveikslėlyje parodytas programos išjungimo patvirtinimo dialogas, kuris atrodo kaip įspėjimas, kad nebaigta naudotis programa. Programa sufleruoja spausti „Yes“ mygtuką norint užbaigti darbą.

    \ximage{fig:case_1}{Programos langas bei išjungimo patvirtinimo dialogas}{justinas/images/case_1.jpg}
  }
  \xcresult
  {
    Kadangi nebuvo norima užbaigti darbo prieš išjungiant programą, buvo pasirinktas mygtukas „No“, po kurio paspaudimo nieko neįvyko. Procesas kartotas keletą kartų iki tol, kol buvo nuspręsta išbandyti „Yes“ mygtuką, kad pavyktų atlikti tai, kas buvo norima.
  }
  \xcprinciples
  {
    \xpentry{Nuspėjamumas}{Pažeistas}
    {
      Buvo tikėtasi, kad programa siūlo baigti darbą prieš ją išjungiant, rodydama tokį patvirtinimo dialogą.
    }
	\xpentry{Atpažįstamumas}{Pažeistas}
	{
	  Didžioji dauguma programų įspėja, kad nebaigtas darbas rodydamos panašų dialogą, tik klausdamos ar norima išsaugoti prieš uždarant.
	  Taip pat daugumoje programų išjungimo patvirtinimo dialogo tekstas būna panašus į „Do you really want to quit?“.
	}
  }
  \xcthoughts
  {
    Tikėtina, kad buvo norima pabrėžti programos galimybes dažniau naudojant žodį „Recovery“. Norint išvengti tokių nesupratimų, turėtų būti vengiama dviprasmiškų pranešimų.
  }
\end{xcase}