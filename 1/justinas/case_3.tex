\begin{xcase}{Windows operacinės sistemos atnaujinimas}
  \xcgoal
  {
    Nukelti kompiuterio perkrovimą po atnaujinimų vėlesniam laikui.
  }
  \xctools
  {
    \ref{fig:case_3} paveikslėlyje parodytas pranešimas, jog norint, kad atnaujinimai įsigaliotų, 
    reikia perkrauti kompiuterį.

    \ximage{fig:case_3}{Pranešimo apie reikiamą kompiuterio perkrovimą dialogas}{justinas/images/case_3.jpg}
  }
  \xcresult
  {
    Kadangi tuo metu buvo intensyviai dirbama, kompiuterio perkrovimą reikėjo nukelti vėlesniam 
    laikui, todėl buvo spaudžiamas mygtukas „Restart Later“. Ta pati lentelė pasirodydavo kas 5 
    minutes taip trukdydama darbui iki tol, kol buvo nuspręsta nespausti nė vieno mygtuko, o 
    tiesiog paslėpti lentelę, kad ši netrukdytų dirbti.
  }
  \xcprinciples
  {
    \xpentry{Daugiagijiškumas}{Pažeistas}
    {
      Didžioji dauguma programų apie kažkokius pasikeitimus, įvykusius naudotojui dirbant ne su 
      tomis programomis, praneša netrukdydamos naudotojui dirbti ir neprašydamos staigaus atsako.
    }
  }
  \xcthoughts
  {
    Tikėtina, kad buvo norima pasirūpinti, jog naudotojas nepamirštų
    apie atliktus atnaujinimus ir kompiuterio perkrovimo būtinumą.
    Norint išvengti tokių nesusipratimų reiktų pranešti tokio
    pobūdžio informaciją netrikdant naudotojo ir neblaškant jo
    dėmesio į nekritinio pobūdžio sistemos įvykius.
  }
\end{xcase}
