\begin{xcase}{„Assassin's Creed“ žaidimo paleidimas}
  \xcgoal
  {
    Paleisti žaidimą „Assassin's Creed“.
  }
  \xctools
  {
    \ref{fig:case_4} paveikslėlyje parodytas pranešimas, kad kažkokia programa jau naudoja 
    garso tvarkyklę, todėl yra siūloma išjungti bet kokias programas, kurios gali ją naudoti, 
    arba žaisti be garso.

    \ximage{fig:case_4}{Įspėjimas apie naudojamą garso tvarkyklę}{justinas/images/case_4.jpg}
  }
  \xcresult
  {
    Kadangi vis dėl to norima buvo žaisti su garsu, buvo bandoma išjungti bet kokias programas 
    potencialiai galinčias naudoti garso tvarkyklę. Išjungus didžiąją dalį programų ir vis tiek 
    susidūrus su  ta pačia problema buvo ieškomas kitas sprendimas, kol galiausiai paaiškėjo, 
    kad kompiuteryje garso tvarkyklė išvis nebuvo įdiegta.
  }
  \xcprinciples
  {
    \xpentry{Nuspėjamumas}{Pažeistas}
    {
      Buvo tikėtasi, kad kaip programa ir praneša, klaida įvyko, nes garso tvarkyklė jau 
      buvo naudojama.
    }
    \xpentry{Sintezavimas}{Pažeistas}
    {
      Nebuvo įmanoma įvertinti, kurie atlikti veiksmai iššaukė tokią klaidą, kadangi pranešama 
      klaida neatitiko realybės.
    }
    \xpentry{Atpažįstamumas}{Pažeistas}
    {
      Buvo manoma, kad problema pažįstama ir žinomas jos sprendimo būdas, tačiau sprendimo 
      būdas buvo visiškai kitoks, nei įprasta.
    }
  }
  \xcthoughts
  {
    Tikėtina, kad buvo netinkamai įvertintos visos priežastys, dėl kurių gali kilti tokia 
    problema. Norint išvengti tokių nesusipratimų, jeigu priežastis nėra tiksliai žinoma, 
    reikėtų pranešti abstrakčiau, kad naudotojas neatmestų gero problemos sprendimo būdo, 
    kaip netinkamo.
  }
\end{xcase}
