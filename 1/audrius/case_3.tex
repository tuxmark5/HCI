\begin{xcase}{VMware Workstation}
  \xcgoal
  {
    Sukurti virtualią mašiną ir ja pasinaudoti.
  }
  
  \xctools
  {
    VMware Workstation interfeisas (\ref{fig:vmware_1}) yra pakankamai nuspėjamas. Sudėtingesnių elementų 
    paaiškinimus galima pamatyti informacinių pranešimų pagalba (tooltips), o sudėtingos
    operacijos perkeltos į vedlius, kur kiekvienas žingsnis (pvz. \ref{fig:vmware_2}) pakankamai detaliai 
    paaiškintas su nuoroda į pagalbinę dokumentaciją.
    \ximage{fig:vmware_1}{Pagrindinis VMware Workstation interfeisas}{audrius/images/vmware_1.png}
    \ximage{fig:vmware_2}{Antrasis naujos VM kūrimo vedlio žingsnis}{audrius/images/vmware_2.png}
  }
  
  \xcresult
  {
    Pirmą kartą naudojantis virtualizacijos produktu neskaitant dokumentacijos, pagalbinės
    informacijos ir panašaus pobūdžio instrukcijų puikiai pavyko susikurti virtualią mašiną
    ir sėkmingai ją eksplotuoti.
  }
  
  \xcprinciples
  {
    \xpentry{Apibendrimas}{Įgyvendintas}
    {
      VMware Workstation sistemoje yra visi standartiniai meniu (File, Edit, View, Help), kurių
      struktūra yra panaši į gerai žinomų teksto redagavimo priemonių struktūrą.
    }
    \xpentry{Atpažįstamumas}{Įgyvendintas}
    {
      Sudėtingiausiai operacijai – virtualios mašinos kūrimui yra pateiktas vedlys, kurio
      pagalba greitai ir efektyviai galima susikurti naują virtualią mašiną.
    }
    \xpentry{Matomumas}{Įgyvendintas}
    {
      Pagrindiniame naudotojo interfeise sudėtos tik esminės funkcijos reikalingos virtualių
      mašinų naudojimui, visos kitos perkeltos į atitinkamus dialogus ir meniu.
    }
    \xpentry{Nuspėjamumas}{Įgyvendintas}
    {
      Kiekvieno interfeiso elemento funkcijos yra pavaizduotos prasmingomis piktogramomis, meniu
      įrašai nėra visai trumpi, bet yra informatyvūs.
    }
    \xpentry{Sintezavimas}{Įgyvendintas}
    {
      Pagal įrankių juostos mygtukų išvaizdą buvo galima nuspėti jų vykdomas funkcijas ir 
      poveikį sistemai.
    }
  }
  
  \xcthoughts
  {
    VMware Workstation – tai produktas pagrinde skirtas programuotojams. Aukštą interfeiso
    panaudojamumo lygį paaiškina didelė produkto kaina bei didelis produkto amžius.
  }
\end{xcase}
