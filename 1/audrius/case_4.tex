\begin{xcase}{XMonad}
  \xcgoal
  {
    Atsidaryti daugiau negu 2 notepad (ar kokios kitos programos) langus taip, kad visi 
    tilptų į ekraną.
  }
  
  \xctools
  {
    Langų valdyklės pradinė būsena minimali (pav. \ref{fig:xmonad_1a}), niekaip nesufleruojanti apie galimas elgesio
    anomalijas.
    \ximagerow{fig:xmonad_1}{XMonad interfeisas}{2}
    {
      \ximagecell{fig:xmonad_1a}{Pradinė būsena}{audrius/images/xmonad_1.png}
      \ximagecell{fig:xmonad_1b}{Tikėtinas rezultatas}{audrius/images/xmonad_3.png}
    }
  }
  
  \xcresult
  {
    Rezultatas – stulbinantis. Ne tik langų rankiniu būdu ekrane nereikėjo dėlioti, bet šie
    ir proporcingai pasiskirstė po ekraną taip, kad nei vienas iš jų nepersidengtų \ref{fig:xmonad_2}.
    \ximage{fig:xmonad_2}{4 atidaryti langai su XMonad langų valdykle}{audrius/images/xmonad_2.png}
  }
  
  \xcprinciples
  {
    \xpentry{Užduočių perkėlimas}{Įgyvendintas}
    {
      Naudotojas turi galimybę pozicionuoti langus ir rankiniu būdu, tačiau tai daug efektyviau
      atliekama automatiškai pasinaudojus XMonad galimybėmis.
    }
    \xpentry{Nuspėjamumas}{Pažeistas}
    {
      Tikėtasi, jog langai persidengs (pav. @ref{fig:xmonad_1b}).
    }
  }
  
  \xcthoughts
  {
    Nors ir nuspėjamumo panaudojimo principas buvo pažeistas, tačiau tai yra „malonus 
    siurprizas“, nes rankinis langų tvarkymas – tai operacija, kurią modernios operacinės
    sistemos tikrai turėtu automatizuoti.
  }
\end{xcase}
