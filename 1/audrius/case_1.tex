\begin{xcase}{Eclipse fono spalvos konfigūracija}
  \xcgoal
  {
    Pakeisti kodo redaktoriaus fono spalvą.
  }
  
  \xctools
  {
    Senesnėse versijose jokių indikacijų apie papildomų spalvinių parametrų egzistavimą nebuvo.
    Tik vėlesnėse Eclipse platformos versijose buvo įdėtos nuorodos su komentaru į visus 3
    spalvų konfigūracijos dialogus.
    %
    \ximage{fig:eclipse_1}{Sintaksės spalvų konfigūravimo langas}{audrius/images/eclipse_1.png}
    \ximage{fig:eclipse_2}{Anotacijų išvaizdos konfigūravimo langas}{audrius/images/eclipse_3.png}
    \ximage{fig:eclipse_3}{Redaktoriaus spalvų konfigūravimo langas}{audrius/images/eclipse_4.png}
  }
  
  \xcresult
  {
    Pirmą kartą pabandžius keisti fono spalvą, atitinkamų nustatymų iškart rasti nepavyko.
    Tik paieškojus internete pavyko surasti užuominą, jog Eclipse platforma skaido spalvinius
    nustatymus į 3 dialogus.
  }
  
  \xcprinciples
  {
    \xpentry{Apibendrinimas}{Pažeistas}
    {
      Visi kiti konfigūracijos parametrai yra grupuojami į fiksuotus vienetus pagal savo pobūdį 
      ir semantiką.
    }
    \xpentry{Darna}{Pažeistas}
    {
      Fono spalva yra laikoma išimtimi ir yra perkelta į skirtingą nustatymų grupę (pav. @ref{fig:eclipse_3}).
    }
    \xpentry{Nuspėjamumas}{Pažeistas}
    {
      Tikėtasi rasti visus spalvinius parametrus vienoje vietoje (pav. @ref{fig:eclipse_1}).
      Visuose kituose kodo redaktoriuose spalviniai parametrai yra sudedami į vieną 
      dialogą/dialogų grupę.
    }
  }
  
  \xcthoughts
  {
    Eclipse platformos autoriai sugrupavo spalvinius parametrus vien pagal semantiką: 
    sintaksinių elementų spalvų parametrai yra keičiami per pirmąjį dialogą (pav. \ref{fig:eclipse_1}), redaktoriaus
    elementų per antrąjį (pav. \ref{fig:eclipse_3}), kodo anotacijos - per trečiąjį (pav. \ref{fig:eclipse_2}). Nors ir struktūriškai toks variantas
    yra matematiniu požiūriu idealesnis, tačiau naudotojui toks išdėstymas yra pakankamai
    nenuspėjamas.
  }
\end{xcase}
