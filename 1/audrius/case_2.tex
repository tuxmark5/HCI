\begin{xcase}{Klaidinantis GCC klaidos pranešimas}
  \xcgoal
  {
    Pagal pateiktą klaidos pranešimą (\ref{fig:gcc}) greitai ir efektyviai pašalinti iškilusią problemą ir
    toliau tęsti GCC kompiliavimą.
    \ximage{fig:gcc}{Kompiliavimo metu gautas klaidos pranešimas}{audrius/images/gcc.png}
  }
  
  \xctools
  {
    Klaidos pranešime buvo nurodyta, jog kažko sistemoje nebuvo rasta, tačiau ko konkrečiai – 
    neįvardinta.
  }
  
  \xcresult
  {
    Ne tik iš klaidos pranešimo nepavyko suprasti, kas atsitiko blogai, tačiau nepavyko 
    surasti ir internete šiai klaidai pašalinti reikalingos informacijos.
    Klaida buvo galiausiai netiesiogiai pašalinta pradedant GCC kompiliavimo procesą iš naujo
    vadovaujantis internete rastu gidu-aprašymu, kaip tai tiksliai reikia atlikti GCC kompiliavimą. 
    Pasirodo, papildomai reikėjo parsisiųsti ir sukompiliuoti BINUTILS paketą prieš pradedant GCC 
    kompiliavimo procesą.
  }
  
  \xcprinciples
  {
    \xpentry{Darna}{Pažeistas}
    {
      GCC kompiliavimo procesą vykdo speciali kompiliavimo sistema Automake, kuri yra atsakinga
      už klaidų aptikimą ir prasmingų pranešimų atspausdinimą. Šiuo atveju ne tik Automake 
      sistema nepateikė informatyvaus klaidos pranešimo, bet ir naptiko pačios klaidos.
    }
    \xpentry{Nuspėjamumas}{Pažeistas}
    {
      Iš pateikto klaidos pranešimo visiškai nepavyko nustatyti galimos klaidos priežasties.
    }
  }
  
  \xcthoughts
  {
    GCC kompiliavimas yra pakankamai sudėtinga procedūra, kurios
    paprasti naudotojai (ir net programuotojai) dažniausiai patys
    neatlikinėja. Labiausiai tikėtina, jog GCC autoriai laikė tai
    savaime suprantamu dalyku, kad prieš GCC kompiliavimą reikia
    įdiegti ir BINUTILS paketą, tačiau tai nepaaiškina, kodėl
    nebuvo pateiktas joks prasmingas klaidos pranešimas kompiliavimo
    pradžioje. Aprašomas klaidos pranešimas buvo pateiktas praėjus
    net 20 minučių nuo kompiliavimo pradžios.
  }
\end{xcase}
