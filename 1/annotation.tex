\xchapter{ANOTACIJA}

\xsmallsec{Informacija apie vykdytojus ir jų įnašą į darbą}
\xtable
{
  w [ 2  | 7 ]
  a [ p  | p ]
  h [ Vykdytojas | Įnašas ]
  %
  e [ Vytautas Astrauskas 
  | @begin{xenum} 
      @item Klaidinantis SQLite klaidos pranešimas 
      @item Klaidinanti Opera Mini mygtuko antraštė 
      @item GMail laiškų keitimas 
      @item Muzikos (ne)importavimas į iTunes 
      @item Pradinė dokumento struktūra
    @end{xenum}
  ]
  %
  e [ Martynas Budriūnas
  | @begin{xenum} 
      @item Simbolių šalinimas Samsung GT-C5130S mobiliojo telefono Java ME programose
      @item Žodžio įtraukimas į T9 žodyną Samsung GT-C5130S mobiliajame telefone
      @item Skambučių priėmimas Samsung GT-C5130S mobiliajame telefone
      @item VU studentų atsiliepimų apie Erasmus studijas peržiūra
    @end{xenum}
  ]
  %
  e [ Justinas Jucevičius 
  | @begin{xenum} 
      @item „Recovery Toolbox for Word“ programos išjungimas
      @item Prisijungimas prie Yahoo! Mail
      @item Windows operacinės sistemos atnaujinimas
      @item „Assassin's Creed“ žaidimo paleidimas
    @end{xenum}
  ]
  %
  e [ Egidijus Lukauskas 
  | @begin{xenum} 
      @item Tekstinės žinutės keitimas iOS ver.3.2.3
      @item Failų peržiųra Nokia Symbian v52.0.101 operacinėje sistemoje
      @item Google Chrome (Chromium) optimizuota viršutinė naršyklės dalis
    @end{xenum}
  ]
  %
  e [ Audrius Šaikūnas 
  | @begin{xenum} 
      @item Eclipse fono spalvos konfigūracija 
      @item Klaidinantis GCC klaidos pranešimas 
      @item VMware Workstation 
      @item XMonad
      @item Atnaujinta dokumento struktūra
    @end{xenum}
  ]
}

\xsmallsec{Bibliografinis darbo aprašas}
Šiuo dokumentu siekiama formaliai aprašyti ir išanalizuoti pastebėtus esamų sąsajų nepatogumus,
paaiškinti koks panaudojamumo principas buvo pažeistas ir kodėl. Iš kitos pusės, antrasis šio
dokumento tikslas yra dokumentuoti interfeisus, kuriuose ne tik minėtų problemų nėra, tačiau kur
ir panaudojamumo projektavimo principai yra realizuoti idealiai konkrečiame kontekse.

\xsmallsec{Darbo vadovas}
Šis darbas yra parengtas kaip žmogaus ir kompiuterio sąveikos pirmasis labulatorinis darbas
– „Pastebėti esamų interfeisų (ne)patogumai“, vadovaujant dėstytojai Kristinai Lapin.
