\begin{xcase}{GMail laiškų keitimas}
  \xcgoal
  {
    Elektroniniu paštu pasiklausti, kokio tipo brūkšnius reikia vartoti.
  }
  
  \xctools
  {
    \ref{fig:case_3_send} paveikslėlyje parodyta, kaip GMail atvaizdavo
    išsiųstąjį laišką.

    \ximage{fig:case_3_send}{Nusiųsto laiško fragmentas}{vytautas/images/case_3_send.png}
  }
  
  \xcresult
  {
    \ref{fig:case_3_got} paveikslėlyje parodyta, kaip GMail atvaizdavo
    gautąjį laišką.
    
    \ximage{fig:case_3_got}{Gautojo laiško fragmentas}{vytautas/images/case_3_got.png}
  }
  
  \xcprinciples
  {
    \xpentry{Nuspėjamumas}{Pažeistas}
    {
      Buvo tikėtasi, kad kaip laiškas atrodo išsiųstas,
      taip jis atrodys ir gautas.
    }
    \xpentry{Darna}{Pažeistas}
    {
      Beveik visada siunčiamų laiškų tekstas išlieka nepakeistas,
      nepriklausomai nuo siuntėjo ir gavėjo naudojamų įrenginių, taip
      pat apie kitokio tipo laiško pakeitimus (pavyzdžiui, kad nerodomi
      vaizdai) yra pranešama.
    }
  }
  
  \xcthoughts
  {
    Tikėtina, kad laiško konvertavimas buvo atliktas, norint jį geriau
    atvaizduoti telefono ekrane. Norint išvengti dėl konvertavimo
    atsirandančių nesusipratimų, derėtų bent jau parodyti pranešimą, kad
    laiško turinys buvo pakeistas.
  }
\end{xcase}
