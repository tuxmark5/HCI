\begin{xcase}{Muzikos (ne)importavimas į iTunes}
  \xcgoal
  {
    Pridėti keletą naujų muzikos albumų į iTunes muzikos grotuvo biblioteką.
  }
  
  \xctools
  {
    \ref{fig:case_4_add_to_library_menu},
    \ref{fig:case_4_add_to_library_dialog} ir
    \ref{fig:case_4_processing} paveikslėliuose pateikta veiksmų seka, kuri
    buvo atlikta siekiant pridėti naujus albumus į biblioteką.

    \ximage{fig:case_4_add_to_library_menu}{Funkcijos pasirinkimas}{vytautas/images/case_4_add_to_library_menu.png}
    \ximage{fig:case_4_add_to_library_dialog}{Norimų pridėti albumų pasirinkimas}{vytautas/images/case_4_add_to_library_dialog.png}
    \ximage{fig:case_4_processing}{Įkėlimas}{vytautas/images/case_4_processing.png}
  }
  
  \xcresult
  {
    Patikrinus paaiškėjo, kad buvo pridėti tik patys pirmieji albumai.
    Pakartojus procedūrą, nebuvo pridėtas nei vienas naujas albumas. 
  }
  
  \xcprinciples
  {
    \xpentry{Nuspėjamumas}{Pažeistas}
    {
      Buvo tikėtasi, kad į biblioteką bus pridėti visi failai, kuriuos buvo nurodyta pridėti.
    }
  }
  
  \xcthoughts
  {
    Greičiausiai programa importuoja visus failus iš eilės tol, kol suranda
    tokį, kurį jau turi bibliotekoje. Tai paaiškintų, kodėl pirmąjį kartą
    buvo pridėti keli pirmieji albumai, o antrąjį – nieko. Tokiu atveju
    turbūt paprasčiausias ir geriausias sprendimas būtų tiesiog paklausti
    naudotojo: „Šis failas bibliotekoje jau yra. Ką norite daryti? Pridėti,
    ignoruoti ar nutraukti?“.
  }
\end{xcase}
