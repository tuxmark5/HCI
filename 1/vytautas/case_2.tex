\begin{xcase}{Klaidinanti Opera Mini mygtuko antraštė}
  \xcgoal
  {
    Parsisiųsti failą iš interneto, naudojant mobilųjį telefoną.
  }
  
  \xctools
  {
    \ref{fig:case_2} paveikslėliuose pateikta naršyklės pateikiamų dialogų
    seka, kai yra bandoma parsisiųsti failą. Paskutiniame dialoge
    (\ref{fig:case_2_final_dialog}) tampa neaišku kas atsitiko: parašyta,
    jog „Atsiuntimas baigtas“, bet tuo pačiu tėra vienintelis
    pasirinkimas „Atsisakyti“ ir jis yra virš dešiniojo funkcinio
    mygtuko (virš kairiojo mygtuko paprastai būna pasirinkimai tokie,
    kaip „Gerai“, „Ok“, „Taip“, o virš dešiniojo „Atšaukti“,
    „Ne“ ir „Išeiti“).
    
    \ximagerow{fig:case_2}{Failo parsisiuntimas Opera Mini naršyklėje}{3}
    {
      \ximagecell{fig:case_2_save_dialog}{Failo išsaugojimas}{vytautas/images/case_2_save_dialog.png}
      \ximagecell{fig:case_2_in_progress}{Siunčiama}{vytautas/images/case_2_in_progress.png}
      \ximagecell{fig:case_2_final_dialog}{Failas parsiųstas}{vytautas/images/case_2_final_dialog.png}
    }
  }
  
  \xcresult
  {
    Pasirinkus vienintelį galima pasirinkimą „Atsisakyti“ ir paskui patikrinus
    atsisiųstąjį failą, paaiškėjo, kad su juo viskas yra gerai.
  }
  
  \xcprinciples
  {
    \xpentry{Nuspėjamumas}{Pažeistas}
    {
      Buvo neaišku kas
      atsitiks nuspaudus vienintelį galimą mygtuką. Taip pat buvo pažeistas
      ir darnos principas: kadangi failo atsiuntimas pavyko, tai apie tai
      pranešančio dialogo uždarymui neturėtų būti naudojamas tas pats
      mygtukas, kuris įprastai yra skirtas operacijų atšaukimui.
    }
  }
  
  \xcthoughts
  {
    Greičiausiai, šis pranešimas yra tiesiog programavimo klaida: mygtukui
    nurodytas neteisingas tipas.
  }
\end{xcase}
