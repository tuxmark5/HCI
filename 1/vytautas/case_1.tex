\begin{xcase}{Klaidinantis SQLite klaidos pranešimas}
  \xcgoal
  {
    Iš taikomosios programos, naudojančios 
    Django\footnote{Projekto svetainė: \url{https://www.djangoproject.com/}.} 
    karkasą, įrašyti duomenis į
    SQLite\footnote{Projekto svetainė: \url{http://sqlite.org/}.}
    duomenų bazę.
  }
  %
  \xctools
  {
    Nustatymų faile buvo prašoma nurodyti failų sistemos kelią iki duombazės 
    failo. Kadangi programa turėjo veikti GNU/Linux sistemoje, tai taip
    pat buvo patikrinta, ar naudotojas, kuris paleidžia programą, turi
    teisę skaityti ir rašyti nurodytą failą.
  }
  %
  \xcresult
  {
    Programa lūžo, su klaidos pranešimu nurodytu
    \ref{fig:case_1_sqlite_error} paveiksle. Kadangi visi su duombaze susiję
    nustatymai tėra tik kelias iki failo, tai reakcija buvo dar kartą
    patikrinti ar programa pasiekia failą ir ar gali jį skaityti ir rašyti.
    Kaip vėliau paaiškėjo, su pačiu duomenų bazės failu viskas yra gerai.
    Duomenų bazė bando sukurti failą, informacijai apie tranzakcijas
    saugoti, tame pačiame kataloge, kuriame yra ir pats duombazės failas,
    bet to jai padaryti nepavyksta, nes naudotojas, paleidęs programą,
    neturi tesės rašyti į tą katalogą.
    
    \ximage{fig:case_1_sqlite_error}{Klaidinantis SQLite klaidos pranešimas}{vytautas/images/case_1_sqlite_error.png}
  }
  %
  \xcprinciples
  {
    \xpentry{Darna}{Pažeistas}
    {
      Unix tipo operacinėse
      sistemose yra įprasta, kad pranešimas nurodantis, jog nepavyko
      atverti failo, reiškia jog nepavyko atverti paties failo, o ne
      kad nepavyko atlikti kokios nors (kad ir su tuo susijusios)
      procedūros.
    }
  }
  %
  \xcthoughts
  {
    Galimas paaiškinimas, kodėl SQLite grąžina būtent tokį klaidos
    pranešimą, būtų tai, kad tranzakcijų failas yra kuriamas bandant
    atidaryti duomenų bazės failą ir įvykusios atidarymo metu klaidos
    nėra diferencijuojamos. Vienas iš paprasčiausių sprendimų būtų
    tiksliai nurodyti priežastį, kodėl būtent nepavyko atverti
    duomenų bazės.
  }
\end{xcase}
